% !TeX program = lualatex
\documentclass[protokoll]{dvd}

\KOMAoptions{
    headwidth = 18cm,
    footwidth = 18cm,
}
\usepackage{graphicx}
\begin{document}

\title{Styrelsemöte 2}
\subtitle{2024}
\author{Styrelsen}
\date{2024-02-05}


\textbf{Datum:} \csname @date\endcsname\\
\textbf{Tid:} 12:10\\
\textbf{Plats:} Styrelserummet\\
\textbf{Styrelsemedlemmar:}
\begin{närvarande_förtroendevalda}
    \förtroendevald{Ordförande}{Samuel Hammersberg}{Ja}
    \förtroendevald{Vice ordförande}{Tim Persson}{Ja}
    \förtroendevald{Kassör}{Lukas Gartman}{Ja}
    \förtroendevald{SAMO}{Josefin Kokkinakis}{Ja}
    \förtroendevald{Sekreterare}{Gustav Dalemo}{Ja}
\end{närvarande_förtroendevalda}

%\textbf{Övriga medlemmar:}
%\medlem{Ingen}
\section{Öppnande av möte}

Mötet öppnades av Samuel Hammersberg kl 12:09

\section{Runda bordet}

Runda bordet innebär att varje person berättar hur de känner sig.
Man kan till exempel berätta att man är stressad på grund av en inlämning, irriterad på sin granne, eller bara väldigt glad därför att man ligger i fas med plugget.

\section{Formalia}

\subsection{Styrelsens beslutbarhet}

\blockquote[7 kap. 5 \S~första stycket i stadgan][]{%
    Styrelsen är endast beslutsmässig då samtliga styrelsemedlemmar har fått kallelsen till styrelsemötet och minst hälften av styrelsemedlemmarna är närvarande.
    Ordförande eller vice ordförande måste vara närvarande när beslut tas.
}

\subsubsection*{Förslag}

\begin{attsatser}
    \item Styrelsen har uppnått kraven i 7 kap. 5 § första stycket i stadgan och är därmed beslutbar.
\end{attsatser}
\subsubsection*{Beslut}
\begin{attsatser}
    \item förslaget till beslut bifalles
\end{attsatser}


\subsection{Fastställande av mötesschema}

För att styrelsen ska kunna fatta ett styrelsebeslut eller protokollföra en diskussion behöver punkten i mötesschemat där styrelsen ska fatta beslut vara inlagd eller föras in i mötesschemat senast vid den här punkten.

\subsubsection*{Förslag}

\begin{attsatser}
    \item mötesschemat fastställs utan några förändringar.
\end{attsatser}
\subsubsection*{Beslut}
\begin{attsatser}
    \item förslaget till beslut bifalles
\end{attsatser}

\subsection{Val av mötessekreterare}
Gustav Dalemo tar sig an uppdraget
\subsubsection*{Förslag}
\begin{attsatser}
    \item Gustav Dalemo väljs till mötessekreterare
\end{attsatser}
\subsubsection*{Beslut}
\begin{attsatser}
    \item förslaget till beslut bifalles
\end{attsatser}

\subsection{Val av protokolljusterare}

Protokolljusterare har till uppgift att kontrollera att protokollet i slutändan reflekterar de faktiska besluten och diskussionerna som fördes under mötet.
Utöver protokolljusteraren så ska mötesordförande och mötessekreteraren signera protokollet.
Vid styrelsemöten ska det endast vara en justerare.
Mötesordförande och mötessekreteraren kan inte vara justerare.

\subsubsection*{Förslag}
\begin{attsatser}
    \item Lukas Gartman väljs till protokolljusterare
\end{attsatser}
\subsubsection*{Beslut}
\begin{attsatser}
    \item förslaget till beslut bifalles
\end{attsatser}

\section{Rapport}
\subsection{Ordförande}
Klar med formuläret och ska skicka ut det till PAs så får dem skicka ut det. Det ska bli intressant att höra vad folk tycker om utbildningen.

\subsection{Kassör}
Det kom in en bankgiro-betalning på 905kr från en Rune Johansson. Oklart vem det är och vart dem kommer ifrån. En mysko inbetalning. Vi behöver nog ringa banken och kolla upp detta och förmodligen betala tillbaks.
Börjat betala ut pengar till data-tjej.

\subsection{Vice-ordförande}
Inbjudan till komittémötet men fick frågan om att byta datum. Svårt att skicka mail med långsam dator.

\subsection{SAMO}
Incidenthantering har kollats på lite, men lite oklart hur mycket man ska skriva. Vi diskuterar detta vidare på dedigerad diskussionspunkt.
UGAIT-mötet diskuterade IT-fakultetens nedläggning. Pratade mycket om Lindholmsrelaterade ämnen som inte påverkar oss direkt. Dem hade gjort klart rapport om trivsel/studiemiljö om brist på platser etc.

\subsection{Mötessekreterare}
Gjort en liten latex template med TODO-list så det blir enklare för mötessekreterare att anteckna.

\newpage

\section{Diskussionspunkter}

\subsection*{IT-fakulteten}
Beslutet att den ska stängas var redan bestämt, men rektorn behöver mandat. Rapporten var dålig. Vi borde pusha för att folk ska skriva upp att detta behöver göras om eftersom mycket är oklart kring vår framtid och hurvida vi hamnar hos NatSex eller ej.

\subsection*{Backen}
Vi har fått en reserverad plats men det var för ett par år sedan så vi måste nog kontakta kåren igen. Förslagsvis är detta ett uppdrag som ska ges till DVRK/Mega6.

\subsection*{Incidenthantering}
Incidenthanteringsdokumentet är färdigt. Vi föreslår att utvärdera själva dokumentet ska vara en del av dokumentet själv.

\subsection*{Lindholms-flytt}
Studierummet på våning 4 i Patriciabyggnaden ska vi få tillgång till.
Diskussionen bör startas mellan OOPsex och oss för framtida samarbeten.

\subsection*{Bilder och hemsida}
Vi vill fixa hemsidan och ta bilder så det blir färdigt innan sökperioden.

\newpage
\section{Avslutande av möte}

\subsection{Nästa möte} 
2024-02-12 VI SKA TA BILDER!

\subsection{Mötets avslutande}
Mötet avslutades kl. 13:04

\styrelsesignaturer

\end{document}
