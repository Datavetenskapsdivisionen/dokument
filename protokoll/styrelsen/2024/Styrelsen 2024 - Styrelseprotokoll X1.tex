% !TeX program = lualatex
\documentclass[protokoll]{dvd}

\KOMAoptions{
    headwidth = 18cm,
    footwidth = 18cm,
}
\usepackage{graphicx}
\begin{document}

\title{Styrelsemöte 1}
\subtitle{2024}
\author{Styrelsen}
\date{2024-01-22}


\textbf{Datum:} \csname @date\endcsname\\
\textbf{Tid:} 12:10\\
\textbf{Plats:} Styrelserummet\\
\textbf{Styrelsemedlemmar:}
\begin{närvarande_förtroendevalda}
    \förtroendevald{Ordförande}{Samuel Hammersberg}{Ja}
    \förtroendevald{Vice ordförande}{Tim Persson}{Ja}
    \förtroendevald{Kassör}{Lukas Gartman}{Ja}
    \förtroendevald{SAMO}{Josefin Kokkinakis}{Ja}
    \förtroendevald{Sekreterare}{Gustav Dalemo}{Ja}
\end{närvarande_förtroendevalda}

%\textbf{Övriga medlemmar:}
%\medlem{Ingen}
\section{Öppnande av möte}

Mötet öppnades av Samuel Hammersberg kl 12:10

\section{Runda bordet}

Runda bordet innebär att varje person berättar hur de känner sig.
Man kan till exempel berätta att man är stressad på grund av en inlämning, irriterad på sin granne, eller bara väldigt glad därför att man ligger i fas med plugget.

\section{Formalia}

\subsection{Styrelsens beslutbarhet}

\blockquote[7 kap. 5 \S~första stycket i stadgan][]{%
    Styrelsen är endast beslutsmässig då samtliga styrelsemedlemmar har fått kallelsen till styrelsemötet och minst hälften av styrelsemedlemmarna är närvarande.
    Ordförande eller vice ordförande måste vara närvarande när beslut tas.
}

\subsubsection*{Förslag}

\begin{attsatser}
    \item Styrelsen har uppnått kraven i 7 kap. 5 § första stycket i stadgan och är därmed beslutbar.
\end{attsatser}
\subsubsection*{Beslut}
\begin{attsatser}
    \item förslaget till beslut bifalles
\end{attsatser}


\subsection{Fastställande av mötesschema}

För att styrelsen ska kunna fatta ett styrelsebeslut eller protokollföra en diskussion behöver punkten i mötesschemat där styrelsen ska fatta beslut vara inlagd eller föras in i mötesschemat senast vid den här punkten.

\subsubsection*{Förslag}

\begin{attsatser}
    \item mötesschemat fastställs utan några förändringar.
\end{attsatser}
\subsubsection*{Beslut}
\begin{attsatser}
    \item förslaget till beslut bifalles
\end{attsatser}

\subsection{Val av mötessekreterare}
Gustav Dalemo tar sig an uppdraget
\subsubsection*{Förslag}
\begin{attsatser}
    \item Gustav Dalemo väljs till mötessekreterare
\end{attsatser}
\subsubsection*{Beslut}
\begin{attsatser}
    \item förslaget till beslut bifalles
\end{attsatser}

\subsection{Val av protokolljusterare}

Protokolljusterare har till uppgift att kontrollera att protokollet i slutändan reflekterar de faktiska besluten och diskussionerna som fördes under mötet.
Utöver protokolljusteraren så ska mötesordförande och mötessekreteraren signera protokollet.
Vid styrelsemöten ska det endast vara en justerare.
Mötesordförande och mötessekreteraren kan inte vara justerare.

\subsubsection*{Förslag}
\begin{attsatser}
    \item Tim Persson väljs till protokolljusterare
\end{attsatser}
\subsubsection*{Beslut}
\begin{attsatser}
    \item förslaget till beslut bifalles
\end{attsatser}

\section{Rapport}
\subsection{Ordförande}
Fortsatt arbeta på enkäten och inbjuden till möte i slutet av februari och mars.

\subsection{Kassör}
Inget att rapportera, bokföringen inte klart så årsredovisning inte klar. Detta står högt på priolistan.

\subsection{(Blivande) Vice-ordförande}
Monaden är i behov av renovering.

\subsection{(Blivande) SAMO}
Gerdes har godkänt att köpa in vattenkokare, kaffebryggare och termosar till Monaden.
Det är bra att veta exakt hur mycket som går till detta.
Oklart hur man ska göra med datatjej-betalningen men Gerdes har inte svarat. Divisionen kan lägga ut så länge.

\subsection{(Blivande) Mötessekreterare}
Inget att rapportera.

\newpage

\section{Beslutspunkter}

\subsubsection*{Nya poster}
Vi ska dela ut roller för 2024.
\subsubsection*{Förslag till beslut:}
\begin{attsatser}
    \item Josefin Kokkinakis väljs som SAMO verksamhetsår 2024.
    \item Tim Persson väljs till vice-ordförande verksamhetsår 2024.
    \item Gustav Dalemo väljs som sekreterare verksamhetsår 2024.
    \item Lukas Gartman till kassör verksamhetsår 2024.
\end{attsatser}

\subsubsection*{Beslut}
\begin{attsatser}
    \item Attsatserna bifalles
\end{attsatser}

\subsubsection*{Kontinuerliga ordförande möten}
Det vore fördelaktigt att arrangera exempelvis ett månatligt lunchmöte där samtliga kommittéordförande deltar. Det är positivt att det är öppet för alla medlemmar att delta.
\subsubsection*{Förslag till beslut:}
\begin{attsatser}
    \item Styrelsen ska hålla möten med samtliga kommiteeordföranden minst en gång per läsperiod. Ordförandens närvaro uppmanas, och övriga komittemedlemmar är välkomna.
\end{attsatser}

\subsubsection*{Beslut}
\begin{attsatser}
    \item Attsatsen bifalles
\end{attsatser}


\section{Diskussionspunkter}

\subsection*{Incidentrapportering}
SAMO fixar till nästa vecka hur man ska göra.

\subsection*{Mötesprotokoll tidigare år}
Vi ska fixa ihop protokollen så att allt är på samma ställe. Just nu är det lite utspritt.

\subsubsection*{Vilka är styrelsen}
Vice ordförande kan tänka sig fortsätta arbetet.


\newpage
\section{Avslutande av möte}

\subsection{Nästa möte}
2024-02-05

\subsection{Mötets avslutande}
Mötet avslutades kl. 12:57

\styrelsesignaturer

\end{document}
