% !TeX program = lualatex
\documentclass[protokoll]{dvd}

\KOMAoptions{
    headwidth = 18cm,
    footwidth = 18cm,
}
\usepackage{graphicx}
\begin{document}

\title{Styrelsemöte 3}
\subtitle{2024}
\author{Styrelsen}
\date{2024-02-22}


\textbf{Datum:} \csname @date\endcsname\\
\textbf{Tid:} 12:13\\
\textbf{Plats:} Styrelserummet\\
\textbf{Styrelsemedlemmar:}
\begin{närvarande_förtroendevalda}
    \förtroendevald{Ordförande}{Samuel Hammersberg}{Ja}
    \förtroendevald{Vice ordförande}{Tim Persson}{Ja}
    \förtroendevald{Kassör}{Lukas Gartman}{Ja, på länk}
    \förtroendevald{SAMO}{Josefin Kokkinakis}{Ja}
    \förtroendevald{Sekreterare}{Gustav Dalemo}{Ja}
\end{närvarande_förtroendevalda}

\begin{närvarande_medlemmar}
    \medlem{Alva Johansson}
\end{närvarande_medlemmar}

%\textbf{Övriga medlemmar:}
%\medlem{Ingen}
\section{Öppnande av möte}

Mötet öppnades av Samuel Hammersberg kl 12:13

\section{Runda bordet}

Runda bordet innebär att varje person berättar hur de känner sig.
Man kan till exempel berätta att man är stressad på grund av en inlämning, irriterad på sin granne, eller bara väldigt glad därför att man ligger i fas med plugget.

\section{Formalia}

\subsection{Styrelsens beslutbarhet}

\blockquote[7 kap. 5 \S~första stycket i stadgan][]{%
    Styrelsen är endast beslutsmässig då samtliga styrelsemedlemmar har fått kallelsen till styrelsemötet och minst hälften av styrelsemedlemmarna är närvarande.
    Ordförande eller vice ordförande måste vara närvarande när beslut tas.
}

\subsubsection*{Förslag}

\begin{attsatser}
    \item Styrelsen har uppnått kraven i 7 kap. 5 § första stycket i stadgan och är därmed beslutbar.
\end{attsatser}
\subsubsection*{Beslut}
\begin{attsatser}
    \item förslaget till beslut bifalles
\end{attsatser}


\subsection{Fastställande av mötesschema}

För att styrelsen ska kunna fatta ett styrelsebeslut eller protokollföra en diskussion behöver punkten i mötesschemat där styrelsen ska fatta beslut vara inlagd eller föras in i mötesschemat senast vid den här punkten.

\subsubsection*{Förslag}

\begin{attsatser}
    \item mötesschemat fastställs utan några förändringar.
\end{attsatser}
\subsubsection*{Beslut}
\begin{attsatser}
    \item förslaget till beslut bifalles
\end{attsatser}

\subsection{Val av mötessekreterare}
Gustav Dalemo tar sig an uppdraget
\subsubsection*{Förslag}
\begin{attsatser}
    \item Gustav Dalemo väljs till mötessekreterare
\end{attsatser}
\subsubsection*{Beslut}
\begin{attsatser}
    \item förslaget till beslut bifalles
\end{attsatser}

\subsection{Val av protokolljusterare}

Protokolljusterare har till uppgift att kontrollera att protokollet i slutändan reflekterar de faktiska besluten och diskussionerna som fördes under mötet.
Utöver protokolljusteraren så ska mötesordförande och mötessekreteraren signera protokollet.
Vid styrelsemöten ska det endast vara en justerare.
Mötesordförande och mötessekreteraren kan inte vara justerare.

\subsubsection*{Förslag}
\begin{attsatser}
    \item Alva Johansson väljs till protokolljusterare
\end{attsatser}
\subsubsection*{Beslut}
\begin{attsatser}
    \item förslaget till beslut bifalles
\end{attsatser}

\section{Rapport}
\subsection{Ordförande}
Var på möte igår angående campusflytt.
Ett beslut är taget men vi får inte veta vad det är, ty det är sekretessbelagt.

Har jobbat på enkäten och kontaktat ansvarig för festanmälan.

Det verkar som SEM går bra. Dem har en antagningspoäng på 22,5 poäng.\\
60\% av studenterna är kvinnor.\\
Har bett om statistik om utbildningen så att vi kan få förbättringspunkter.

\subsection{Kassör}
Väntar på svar av Gerdes om en eskning.

\subsection{Vice-ordförande}
Inget att rapportera.

\subsection{SAMO}
Inget nytt sen sist.

\subsection{Mötessekreterare}
Inget att rapportera.

\newpage

\section{Beslutspunkter}

\subsubsection*{Incidenthantering}
Alla medlemmar bör läsa igenom den och det är bra om det läggs till som ett officiellt styrdokument.

\subsubsection*{Förslag till beslut:}
\begin{attsatser}
    \item Det ska röstas om att lägga in indidenthanteringen\\ som ett officiellt styrdokument i nästa stämma.
\end{attsatser}

\subsubsection*{Beslut}
\begin{attsatser}
    \item Attsatserna bifalles
\end{attsatser}

\subsubsection*{Förslag till beslut:}
\begin{attsatser}
    \item Preliminärt ha nästa stämma 2024-03-21. Vi ska ha kostymer och peruker.
\end{attsatser}

\subsubsection*{Beslut}
\begin{attsatser}
    \item Attsatserna bifalles
\end{attsatser}

\subsubsection*{Orbi-betalning}
Vi vill ha ett bättre betalningsystem i Orbi. Vi vill sätta upp orbibetalning via oss.
Betalningen går först genom Göta och dem står för transaktionsavgifterna på Orbi. 
Sen får vi en summa inbetalt på ett och samma konto och sen får vi betala ut dem till respektive komitte.

\subsubsection*{Förslag till beslut:}
\begin{attsatser}
    \item Gå vidare med att ansluta datavetenskapsdivisionen till Orbi-pay via Göta.
\end{attsatser}

\subsubsection*{Beslut}
\begin{attsatser}
    \item Attsatserna bifalles
\end{attsatser}


\section{Diskussionspunkter}

\subsection*{Backen}
Vi kommer få kontakta chalmers studentkår igen. Innan det borde vi sätta ihop en arbetsgrupp.
Detta kan tas upp på kommiteemöte. Tim tar sig an att ta upp detta där.

\subsection*{Studentenkät}
Vi har fått in 48 svar.
Den finns länkad här:
\href{https://drive.google.com/file/d/1CCt1Z3ZoE3Ne9fgUs6aXHUFfAUod3lLN/view?usp=drive_link}{Enkät}

\subsection*{Festanmälan}
Har haft kontakt som anvarar om det på chalmers.

\subsection*{Plan för upprustning av Monaden}
Vi skjuter detta till nästa möte pga tidsbrist.

\subsection*{DVD aktiva}
Det finns delade åsikter om hurvida DVD-aktiva kanalen. Finns både för och nackdelar.
Vi lutar åt att ha kvar den men ska uppmana fler att använda den stora kanalen för
saker som kan röra alla.


\newpage
\section{Avslutande av möte}

\subsection{Nästa möte} 
2024-03-04

\subsection{Mötets avslutande}
Mötet avslutades kl. 13:07

\styrelsesignaturer

\end{document}
