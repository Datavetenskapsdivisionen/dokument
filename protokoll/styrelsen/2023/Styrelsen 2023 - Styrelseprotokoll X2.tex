\documentclass[protokoll]{dvd}

\KOMAoptions{
    headwidth = 18cm,
    footwidth = 18cm,
}
\usepackage{graphicx}
\begin{document}

\title{Styrelsemöte 5}
\subtitle{2023}
\author{Styrelsen}
\date{2023-08-22}


\textbf{Datum:} \csname @date\endcsname\\
\textbf{Tid:} 14:30\\
\textbf{Plats:} Monaden\\
\textbf{Styrelsemedlemmar:}
\begin{närvarande_förtroendevalda}
    \förtroendevald{Ordförande}{Samuel Hammersberg}{Ja}
    \förtroendevald{Vice ordförande}{Vakant}{Nej}
    \förtroendevald{Kassör}{Lukas Gartman}{Ja}
    \förtroendevald{SAMO}{Tekla Siesjö}{Nej}
    \förtroendevald{Sekreterare}{Vakant}{Nej}
\end{närvarande_förtroendevalda}

\textbf{Övriga medlemmar:} \\
    \medlem{Albin Otterhäll}


\section{Öppnande av möte}

Mötet öppnades av Samuel Hammersberg kl 14:30

\section{Runda bordet}

Runda bordet innebär att varje person berättar hur de känner sig.
Man kan till exempel berätta att man är stressad på grund av en inlämning, irriterad på sin granne, eller bara väldigt glad därför att man ligger i fas med plugget.

\section{Formalia}

\subsection{Styrelsens beslutbarhet}

\blockquote[7 kap. 5 \S~första stycket i stadgan][]{%
    Styrelsen är endast beslutsmässig då samtliga styrelsemedlemmar har fått kallelsen till styrelsemötet och minst hälften av styrelsemedlemmarna är närvarande.
    Ordförande eller vice ordförande måste vara närvarande när beslut tas.
}

\subsubsection*{Förslag}

\begin{attsatser}
    \item Styrelsen har uppnått kraven i 7 kap. 5 § första stycket i stadgan och är därmed beslutbar.
\end{attsatser}
\subsubsection*{Beslut}
\begin{attsatser}
    \item förslaget till beslut bifalles
\end{attsatser}


\subsection{Fastställande av mötesschema}

För att styrelsen ska kunna fatta ett styrelsebeslut eller protokollföra en diskussion behöver punkten i mötesschemat där styrelsen ska fatta beslut vara inlagd eller föras in i mötesschemat senast vid den här punkten.

\subsubsection*{Förslag}

\begin{attsatser}
    \item mötesschemat fastställs utan några förändringar.
\end{attsatser}
\subsubsection*{Beslut}
\begin{attsatser}
    \item förslaget till beslut bifalles
\end{attsatser}

\subsection{Val av mötessekreterare}
Eftersom att styrelsen för nuvarande inte har en sekreterare tar Lukas Gartman på sig jobbet.
\subsubsection*{Förslag}
\begin{attsatser}
    \item Lukas Gartman väljs till mötessekreterare
\end{attsatser}
\subsubsection*{Beslut}
\begin{attsatser}
    \item förslaget till beslut bifalles
\end{attsatser}

\subsection{Val av protokolljusterare}

Protokolljusterare har till uppgift att kontrollera att protokollet i slutändan reflekterar de faktiska besluten och diskussionerna som fördes under mötet.
Utöver protokolljusteraren så ska mötesordförande och mötessekreteraren signera protokollet.
Vid styrelsemöten ska det endast vara en justerare.
Mötesordförande och mötessekreteraren kan inte vara justerare.

\subsubsection*{Förslag}
\begin{attsatser}
    \item Albin Otterhäll väljs till protokolljusterare
\end{attsatser}
\subsubsection*{Beslut}
\begin{attsatser}
    \item förslaget till beslut bifalles
\end{attsatser}

\section{Rapport}
\subsection{Ordförande}
Samuel Hammersberg har fixat access för att låsa upp dörren för Tim Persson, då han ska kunna låsa upp dörren under mottagningen.
Utöver detta har han varit med och hjälpt

\subsection{Kassör}
Lukas Gartman har kontaktat instutionen för att få reda på status på när mottagningspengarna kommer in.
Tydligen har det varit strul men pengarna ska komma in den 22:e eller den 23:e augusti.

\newpage

\section{Beslutspunkter}

\subsubsection*{Komplettering av beslut på stämma}
Under förra styrelsemötet bestämdes det att Lukas Gartman och Samuel Hammersberg skulle representera
divisionen gentemot banken \textbf{var för sig}. Detta gjordes då inte tydligt nog och beslutet behövs tas igen.
\textbf{var för sig}.

\subsubsection*{Förslag till beslut:}
\begin{attsatser}
    \item Samuel Hammersberg och Lukas Gartman representerar divisionen \textbf{var för sig}
\end{attsatser}

\subsubsection*{Beslut}
\begin{attsatser}
    \item förslaget till beslut bifalles
\end{attsatser}

\begin{center}
    \includegraphics[scale=0.5]{{"./cring"}.png}
\end{center}

\newpage
\section{Avslutande av möte}

\subsection{Nästa möte}
Styrelsen har preliminärt bokat in 31:a 10:00 för nästa möte.

\subsection{Mötets avslutande}
Mötet avslutades kl. 15:00

\styrelsesignaturer

\end{document}
