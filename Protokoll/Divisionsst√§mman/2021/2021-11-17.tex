\documentclass[protokoll]{dvd}

\KOMAoptions{
	headwidth = 18cm,
	footwidth = 18cm,
}

\begin{document}

\title{Divisionsstämmans möte 3}
\subtitle{2021}
\author{Divisionsstämman}
\date{2021-11-17}

\textbf{Datum:} \csname @date\endcsname\\
\textbf{Tid:} 17:17\\
\textbf{Plats:} Monaden\\
\textbf{Styrelsemedlemmar:}
\begin{närvarande_förtroendevalda}
	\förtroendevald{Ordförande}{Albin Otterhäll}{Ja}
	\förtroendevald{Vice ordförande}{Samuel Hammersberg}{Ja}
	\förtroendevald{SAMO}{Tekla Siesjö}{Ja}
	\förtroendevald{Kassör}{Morgan Thowsen}{Ja}
	\förtroendevald{Sekreterare}{Sebastian Selander}{Ja}
\end{närvarande_förtroendevalda}

\textbf{Närvarande övriga medlemmar:}
\begin{närvarande_medlemmar}
    \medlem{Lucas Gyllensvaan}[Lämnade efter 4.19 §]
    \medlem{Anthon Wirback}
    \medlem{Josefin Kokkinakkis}
    \medlem{Benjamin Sannholm}
    \medlem{Emmie Berger}
    \medlem{Kristoffer Gustafsson}
    \medlem{Miranda Jernberg}[Lämnade efter 4.20 §]
    \medlem{Jeffrey Wolff}
    \medlem{Alexander Lisborg}[Anlände efter 4.11 §]
\end{närvarande_medlemmar}

\section{Öppnande av möte}

Mötet öppnades av Albin Otterhäll kl 17:32

\newpage
\section{Formalia}


\subsection{Divisionsstämmans beslutbarhet}

5 kap. i Stadgan och 1 kap. 1--5 §§ i styrdokumentet \emph{Regler för Divisionsstämman} definierar regler Divisionstämman.

Den 2 november kallade styrelsen till extrastämma genom att skicka ut mejl via \verb|utskick@dvet.se|; lägga upp en notis på \url{https://dvet.se}; samt satte upp en lapp i Monaden.
Man skapade även ett event på Facebook och länkade eventet i Facebookgruppen \emph{Monaden (GU Datavetenskap)}.
Det lades upp även notiser på Discordservern \emph{MonadenDV}

Dessa möteshandlingar ska ha skickats ut under måndagen den 15 november.

\subsubsection*{Förslag till beslut}

\begin{attsatser}
	\item Divisionsstämman har uppnått kraven i Stadgan och i \emph{Regler för Divisionsstämman} för att få hålla möte, och är därmed beslutbar.
\end{attsatser}

\subsubsection*{Beslut}

\begin{attsatser}
	\item attsatsen bifalles.
\end{attsatser}

\newpage

\subsection{Fastställande av mötesschema}

För att Divisionsstämman ska kunna fatta ett beslut eller protokollföra en diskussion behöver punkten i mötesschemat där stämman ska fatta beslut vara inlagd eller föras in i mötesschemat senast vid den här punkten.

Inga interpellationer, eller enklare frågor inkom till styrelsen.


\subsubsection*{Förslag till beslut}

\begin{attsatser}
	\item Mötesschemat fastställs utan några förändringar.
\end{attsatser}

\subsubsection*{Beslut}

\begin{attsatser}
	\item attsatsen bifalles.
\end{attsatser}

\subsection{Val av mötesordförande}

Mötesordförande har till uppgift att leda Divisionsstämmans sammankomst.
Hen ansvarar för att mötesformalia sköts.

\subsubsection*{Förslag till beslut}

\begin{attsatser}
	\item Albin Otterhäll väljs till mötesordförande.
\end{attsatser}

\subsubsection*{Beslut}

\begin{attsatser}
	\item attsatsen bifalles.
\end{attsatser}

\newpage

\subsection{Val av vice mötesordförande}

Vice mötesordförande hjälper mötesordförande med att hålla talarlistan, och att alla får komma till tals.

\subsubsection*{Förslag till beslut}

\begin{attsatser}
	\item Samuel Hammersberg väljs till vice mötesordförande.
\end{attsatser}

\subsubsection*{Beslut}

\begin{attsatser}
	\item attsatsen bifalles.
\end{attsatser}

\subsection{Val av mötessekreterare}

Mötessekreteraren har till uppgift att anteckna diskussioner, beslut, och eventuella reservationer under mötet.

\subsubsection*{Förslag till beslut}

\begin{attsatser}
	\item Sebastian Selander väljs till mötessekreterare.
\end{attsatser}

\subsubsection*{Beslut}

\begin{attsatser}
	\item attsatsen bifalles.
\end{attsatser}

\newpage

\subsection{Val av protokolljusterare}

Protokolljusterare har till uppgift att kontrollera att protokollet i slutändan reflekterar de faktiska besluten och diskussionerna som fördes under sammanträdet; samt agera rösträknare vid slutna omröstningar.
Utöver protokolljusterarna så ska mötesordförande och mötessekreteraren signera protokollet.
Vid Divisionsstämmans sammanträden ska det vara två justerare.
Mötesordförande och mötessekreteraren kan inte vara justerare.

\subsubsection*{Förslag till beslut}

\emph{Inga förslag från styrelsen innan mötet}

\subsubsection*{Beslut}
    \begin{attsatser}
        \item Tekla Siesjö väljs till protokolljusterare
        \item Kristoffer Gustafsson väljs till protokolljusterare
    \end{attsatser}

\newpage

\section{Rapporter}

\subsection{Styrelsen}

\subsubsection{Verksamhetsrapport}

Styrelsen har haft fyra möten sedan Divisionsstämmomötet den 15 september.
Protokollen från mötena finns tillgängliga för allmänheten på Styrelsens drive.
Länk till styrdokument och protokoll: \verb|https://drive.google.com/drive/folders/1o_lLM7g7ph-xdxgut3E_BU0ywichWYTX?usp=sharing|

Albin och Tekla har varit på institutionsråd.
Institutionsrådet är en samling personer från näringslivet och D\&ITs institutionsledning.
Informationen vi fick där var bland att man håller på att utrusta föreläsningssalarna med utrustning för att livestreama föreläsningarna.
Tanken är att man i framtiden ska ha hybridföreläsningar, där studenter kan delta antingen på plats eller på distans.
Det kommer att användas även under normala omständigheter, då man den senaste tiden har slagit ihop massa kurser för att kunna spara pengar.
Det har lett till att man har fått kurser som har fler kursdeltagare än vad det finns platser för i föreläsningssalarna.
Många kurser i framtiden kommer ha runt 400 kursdeltagare, och Chalmers har ingen föreläsningssal som rymmer fler än strax över 200.

Albin har löst och satt upp en Google Workspace instans för divisionen.
Det innebär att samtliga divisionsmedlemmar som är aktiva i någon kommitté eller i Styrelsen har möjlighet att få ett användarkonto med en @dvet.se mejladress.

Styrelsen har även köpt in en konferenstelefon för pengar vi har fått från Göta för just det ändamålet.
Det ligger i Styrelserummet och är tillgänglig där.

Styrelsen har även köpt RGB lampor och satt upp i Monaden.
Det arbetas på ett snyggt grafiskt gränssnitt så man enkelt ska kunna styra lampornas färg och styrka.
Länk till källkod: https://github.com/thowsen/belysning
Morgan är väldigt stolt över koden.
Alla som är intresserade och vill lära sig Javascript är välkomna att bidra till projektet!

Styrelsen har haft ett möte med Alex (PA) och Jonathan (SYV) den 22 september.
Den stora saken som diskuterades på mötet var att programplanen kommer förändras för DV'22 och framåt.
Den nya planeringen kommer inte påverka nuvarande studenter, så dessa kör enligt den gamla programplanen.
Linjär algebra kommer flyttas från LP 2 till LP 3, och analysen kommer flyttas från LP 3 till LP 4.
Anledningen är att matteinstitutionen har brist på personer som kan undervisa kurserna, så DV kommer i framtiden samläsa den tillsammans med D och IT.
Och för att alla tre programmens scheman ska gå ihop behöver kurserna flyttas till andra LP:n.
I samband med det kommer det öppnas upp ett hål i LP 2, och då Alex tycker att det är alldeles för tidigt att ha en valbar kurs så har han infört att GruDat (grundläggande datorteknik) kommer vara obligatoriskt.
Osäkert när grudat kommer läsas, men Alex försöker få den att hamna i LP 2.
Han vill att man även i fortsättningen läser diskret matte och Haskell första läsperioden, men det går just nu inte riktigt ihop med D:s och IT:s scheman.
Om han inte lyckas övertyga de andra så kommer GruDat att läsas i LP 1, och antingen Haskell eller diskret matte flyttas till LP 2.

Betygssystemet 3-4-5 kommer även börja tillämpas läsåret 22/23.
Betygssystemet tilläts efter ett rektorsbeslut i mars 2020, men det nya betygssystemet får endast börja användas efter att kursplaner har ändrats, och det tar ett par år.
Det är upp till kursansvariga att bestämma om de ska gå efter G-VG systemet eller 3-4-5 systemet för GU studenter, men 3-4-5 ska bli standard.
Alex Gerdes har pratat med och skickat in kursplansändringar för stort sett alla kurser som är obligatoriska för DV-programmet.
Han ser inte någon större anledningar till varför kursansvariga skulle vilja fortsätta köra på G-VG systemet, men möjligheten finns som sagt.
Det kommer inte finnas någon möjlighet att ''plussa'', vilket Chalmerister har.
Det innebär att om man ''endast'' får en 3:a på en salstenatenta så kan man skriva tentan igen (omtenta) för att höja sitt betyg.
Alex Gerdes tycker att systemet med att ''plussa'' är en styggelse, och inte borde tillåtas överhuvudtaget.

Inga andra saker som Gerdes vill förändra i dagsläget.

Det blivande sexmästeriet har arrangerat

%Lägg rapporter i handlingar

\begin{itemize}
    \item en chillfest; och
    \item en Halloweensittning.
\end{itemize}

Det blivande rustmästeriet och PR-kommittén har arrangerat

\begin{itemize}
    \item två spelkvällar;
    \item en filmkväll; och
    \item en städdag.
\end{itemize}

\newpage

\subsubsection{Ekonomi}

Ingående saldo för året 2021 var 1538kr
Saldo är 6732 för 11 november 2021.

Q1-Q2

Q1 förändring är -135kr från ingående 1538kr till 1403kr.\\
Q2 fick vi en sen insättning från Göta avseende 2020 års mottagning.  Tillsammans med insättning av tidigare års kontantkassa så ökade finanserna till 6290kr

Q3-Q4

Q3 är tyngd av mottagning men exklusive mottagning är resultatet -49kr.\\
Q4 har knappt börjat och i skrivande stund är den enda transaktionen ett inköp av konferenstelefon(högtalare/mikrofon kombo) vilket vi skall få igen av Göta.

Mottagningen.\\
Summan av mottagningen blev 10550kr och är alltså detsamma belopp som sattes in på kontot av Göta.

Så vad händer i Q4? Över lag är vi ganska nöjda med den ekonomiska utvecklingen. Bortsett från ingåendekassa från tidigare år, kontantinsättning och insättning av föregående års mottagning är förändringen -1237kr vilket är godkänt. Den egentliga siffrar är strax under -1000kr då vi väntar på insättningar.

Styrelsen har tidigare under året diskuterat ett mål på att ha tillräckligt med likvider för att täcka en mottagning på strax över 12 000kr. Detta för att undvika situationen som var i år, att medlemmarna ligger ute med pengar tills mottagningen är över och mottagningspengar kommit in från Göta. Med det sagt så kommer budget fortsatt vara relativt återhållsam.


\newpage
\section{Beslutsärenden}

Enligt Stadgan måste ändringar av Stadgan röstas igenom på två av Divisionsstämmans varandra följande möten.
Om en beslutpunkt innehåller ``första läsningen' innebär det att det är första gången beslutet tas upp för omröstning.
Om en beslutspunkt innehåller ``andra läsningen'' innebär det att beslutspunkten har röstats igenom förra stämmomötet, och beslutet behöver bekräftas för att gå igenom.

\subsection{Proposition: Inför ordlistan i Stadgan (andra läsningen)}

Det finns egentligen ingen poäng av att ha ordlistan som ett separat styrdokument skilt från Stadgan.
Stadgan är divisionens grundstomme, och det är alltid den man ska utgå ifrån.
För att göra saker och ting lättare för läsare av dokumentet önskar vi göra ordlistan till en del av Stadgan.

Syftet med ordlistan är att ha en central plats för styrdokument där man kan hitta exakta definitioner av visa termer som används i styrdokument.

\subsubsection*{Förslag till beslut}

\begin{attsatser}
	\item Styrdokumentet \emph{Ordlista för styr- och principdokument} införs som ett nytt första kapitel i stadgan, och försvinner som individuellt dokument.

	\item ändra 12 kap. 1 § första stycket i stadgan från

	\begin{displayquote}
		Dessa stadgar eller ordlistan för styr- och principdokument kan endast ändras genom beslut av divisionsstämman med två tredjedelars majoritet vid två på varandra följande sammanträden.
		Ett av divisionsstämmans sammanträden behöver vara ordinarie.
		Det måste vara minst 10 läsdagar mellan sammanträderna.
	\end{displayquote}

	till

	\begin{displayquote}
		Denna stadga kan endast ändras genom beslut av divisionsstämman med två tredjedelars majoritet vid två på varandra följande sammanträden.
		Ett av divisionsstämmans sammanträden behöver vara ordinarie.
		Det måste vara minst 10 läsdagar mellan mötena.
	\end{displayquote}

	\item ändra 4 § i \emph{Regler för dokumentsamlingen} från

	\begin{displayquote}
		Alla dokument ska skrivas med föreningens ordlista i åtanke.
		Ett ord definierat i ordlistan får inte användas med en annan definition än den som står i ordlistan.
	\end{displayquote}

	till

	\begin{displayquote}
        Alla dokument ska skrivas med stadgan i åtanke.
        Ett ord definierat i Stadgans ordlistan får inte användas med en annan definition än den som står i ordlistan.
    \end{displayquote}
\end{attsatser}

\subsubsection*{Beslut}
    \begin{attsatser}
        \item samtliga attsatser bifalles
    \end{attsatser}

\newpage
\subsection{Proposition: Ändra \emph{årsredovisning} till \emph{årsrapport} (andra läsningen)}

Enligt Stadgan så ska vi skapa en \emph{årsredovisning} av vår ekonomi.
Termen \emph{årsredovisning} har en juridisk innebörd i Sverige, och innebär att man måste följa visa precisa regler.
Vi har aldrig gjort en \emph{årsredovisning}, och kommer inte göra heller under överskådlig framtid.


%KANSKE ÄR GJORT NU?
Det vi däremot ska göra, som vi har tyvärr inte haft möjlighet att göra till detta möte, är att skapa en \emph{ekonomisk årsrapport} med all den information om ekonomin som ni behöver ha.
Självklart får ni grotta ner er i bokföringen om ni känner för det.

\subsubsection*{Förslag till beslut}

\begin{attsatser}
	\item 5 kap. 7 § första meningen i Stadgan ändras från

	\begin{displayquote}
        Senast sex månader efter räkenskapsårets början ska man på divisionsstämman redovisa divisionens årsredovisning samt revisorernas berättelse.
    \end{displayquote}

    till

    \begin{displayquote}
        Senast sex månader efter räkenskapsårets början ska man på Divisionsstämman redovisa divisionens ekonomiska årsrapport samt revisorernas berättelse.
    \end{displayquote}
\end{attsatser}

\subsubsection*{Beslut}
    \begin{attsatser}
        \item samtliga attsatser bifalles
    \end{attsatser}

\newpage
\subsection{Proposition: Ändra \emph{årsmöte} till \emph{Divisionsstämman} (andra läsningen)}

Divisionens högsta organ heter \emph{Divisionsstämman}, och inte \emph{årsmöte}.

\subsubsection*{Förslag till beslut}

\begin{attsatser}
	\item ändra 13 kap. 1 § i Stadgan från

	\begin{displayquote}
        Förslag om divisionens upplösning får behandlas endast på årsmöte.
    \end{displayquote}

	till

	\begin{displayquote}
		Förslag om divisionens upplösning får behandlas endast av Divisionsstämman.
	\end{displayquote}

	\item ändra 4 kap. 7 § i Stadgan från

	\begin{displayquote}
        Divisionens verksamhet och räkenskaper ska granskas av revisor utsedd av årsmötet.
    \end{displayquote}

	till

	\begin{displayquote}
		Divisionens verksamhet och räkenskaper ska granskas av revisor utsedd av Divisionsstämman.
	\end{displayquote}
\end{attsatser}

\subsubsection*{Beslut}
    \begin{attsatser}
        \item samtliga attsatser bifalles
    \end{attsatser}

\newpage
\subsection{Proposition: Häv icke genomförda beslut (andra läsningen)}

Styrelsebeslut och stämmobeslut gäller fram tills dess att de hävs.
Det kan medföra att beslut som fattades för flera år sedan, men som ännu inte verkställts, är tekniskt sett fortfarande giltiga.
För närvarande behöver varje ny styrelse gå igenom divisionens alla tidigare protokoll för att hitta eventuella beslut som de inte håller med om och häva dessa.
Vi föreslår att vi i stadgan inför att alla beslut behöver verkställas under det verksamhetsår där man fattade beslutet.
Om den nya styrelsen håller med ett tidigare icke verkställt beslut behöver de bara protokollföra att det tidigare beslutet fortfarande gäller.

\subsubsection*{Förslag till beslut}

\begin{attsatser}
	\item Under 5 kap. i Stadgan införa en ny paragraf med lydelsen

	\begin{displayquote}
		Samtliga icke verkställda beslut fattade av Divisionsstämman hävs efter det verksamhetsår under vilket beslutet fattades.
	\end{displayquote}

	\item Under 7 kap. i Stadgan införa en ny paragraf med lydelsen

	\begin{displayquote}
		Samtliga icke verkställda beslut fattade av Styrelsen hävs efter det verksamhetsår under vilket beslutet fattades.
	\end{displayquote}
\end{attsatser}

\subsubsection*{Beslut}
    \begin{attsatser}
        \item samtliga attsatser bifalles
    \end{attsatser}

\newpage
\subsection{Proposition: Ändra medlemsperiod (andra läsningen)}

Medlemsperioden är den period under vilket medlemskap i Datavetenskapsdivisionen gäller.
Idag är den från och med 1 oktober till och med 30 september.
Anledningen till att vi valde denna period från början är att Mecenats kårlegitimation uppdateras 1 oktober baserat på kårmedlemskap, även om kårmedlemskapet går ut tidigare.

Problemet är att nya studenter som går med i divisionen när de börjar studera behöver förnya sitt medlemskap kort efter de har gått med från första början.

Vi har pratat med IT-sektionsstyrelsen, och kommit fram till att vi bara behöver skicka över personnummer för att kontrollera kårmedlemskap.
Vi behöver då alltså inte själva kontrollera kårmedlemskapet, utan låta Göta göra det åt oss.
Vi kan därmed ha ha en medlemsperiod mer anpassad efter hur läsåret ser ut.

\subsubsection*{Förslag till beslut}

\begin{attsatser}
	\item för in följande i \emph{ordlistan}

	\begin{displayquote}
		\begin{description}
			\item[medlemsperiod] Perioden under vilket medlemskap i divisionen sträcker sig över.
		\end{description}
	\end{displayquote}

	\item ändra 3 kap. 3 § första stycket från

	\begin{displayquote}
		Medlemskap i divisionen sträcker sig från och med den första oktober till och med den sista september, hädanefter kallat ``medlemsperioden''.
	\end{displayquote}

	till

	\begin{displayquote}
		Medlemsperioden sträcker sig från och med den första juli till och med den sista juni.
	\end{displayquote}
\end{attsatser}

\subsubsection*{Beslut}
    \begin{attsatser}
        \item samtliga attsatser bifalles
    \end{attsatser}

\newpage
\subsection{Proposition: Ta bort valberedningen (andra läsningen)}

Valberedningen är det organ som har till uppgift att lägga fram förslag på personer man tror hade passat in bra på de förtroendeposter där man ska hålla inval.
Tanken är inte riktigt att valberedningen själv ska ta fram personerna som ska väljas, utan mer granska de personer som nuvarande förtroendevalda rekommenderar.

Styrelsen önskar ta bort Valberedningen då vi anser att nackdelarna överväger fördelarna med ett sådant organ.

En av de viktigaste fördelarna med Valberedningen som har lyfts fram är att den introducerar en mer opartiskhet i processen att rekommendera personer till stämman.

Det första problemet är att det behöves folk för att engagera sig i Valberedningen.
Det innebär att personer som annars hade haft tid och engagemang att organisera arrangemang och andra roliga saker behöver lägga den tiden för att ha möten och andra aktiviteter av byråkratisk karaktär.

Det andra problemet är att Valberedningen har stort sett varit icke-existerande det senaste decenniet.
Så Valberedningen är, och har varit det senaste tio åren, en pappersprodukt.

Därför tycker Styrelsen att det är rimligt att ta bort Valberedningen.

\subsubsection*{Förslag till beslut}

\begin{attsatser}
	\item ta bort ``valberedningen'' från listan i 2 kap. 1 § tredje stycket i stadgan.

	\item ta bort 8 kap. från stadgan.

	\item ta bort styrdokumentet \emph{Regler för valberedningen}.

	\item ta bort sista punkten från 1 § i styrdokumentet \emph{Regler för styrelsen}.

	\item ändra punkt 5 i 1 kap. 3 § andra stycket i styrdokumentet \emph{Regler för divisionssstämman} från

	\begin{displayquote}
		valberedningens eventuella förslag på personer till förtroendeuppdrag
	\end{displayquote}

	till

	\begin{displayquote}
		eventuella förslag på personer till förtroendeuppdrag
	\end{displayquote}

	\item ta bort punkt 1 och 2 i 1 § i styrdokumentet \emph{Regler för alla kommittéer}.
\end{attsatser}

\subsubsection*{Beslut}
    \begin{attsatser}
        \item samtliga attsatser bifalles
    \end{attsatser}

\newpage
\subsection{Proposition: Ta bort Talmanspresidiet (andra läsningen)}

Talmanspresidiet är det organ som har till uppgift att arrangera och leda Divisionsstämmans möten.
Tanken är att några få personer kan ta på sig en av Styrelsens regelbundna uppgifter som inte kräver att Styrelsen agerar som Styrelsen.
Man vill helt enkelt underlätta för Styrelsen så att några personer som annars inte hade varit aktiva i divisionen har möjlighet att göra några mindre arbetsuppgifter.

Styrelsen önskar ta bort Talmanspresidiet av ungefär samma skäl som vi önskar ta bort Valberedningen.

\subsubsection*{Förslag till beslut}

\begin{attsatser}
	\item ta bort ``talmanspresidiet'' från listan i 2 kap. 1 § tredje stycket i stadgan.

	\item ta bort 6 kap. från stadgan.

	\item ta bort 5 kap. 1 a § och 5 kap. 1 b § i Stadgan

	\begin{displayquote}
		Divisionsstämmans ordinarie sammanträden annonseras av talmansspresidiet.

		Divisionsstämman måste sammanträda under höstterminen innan nästkommande räkenskapsår börjar.

		Divisionsstämman måste sammanträda under vårterminen innan den första maj.

		Extra sammanträden kan kallas av styrelsen, talmanspresidiet, revisorerna eller minst 25 av divisionens medlemmar.
	\end{displayquote}

	och lägg till följande som nya stycken direkt under 5 kap. 1 §

	\begin{displayquote}
		Divisionsstämmans möten kan kallas av

		\begin{itemize}
			\item styrelsen;
			\item revisorerna; eller
			\item minsta alternativet av 10 personer eller 50 procent av medlemmar som är icke styrelsemedlemmar
		\end{itemize}

		Divisionsstämman måste hålla möte under höstterminen innan nästkommande räkenskapsår börjar.

		Divisionsstämman måste hålla möte under vårterminen innan 1 maj.
	\end{displayquote}

	\item ändra 5 kap. 6 § första stycket första meningen i Stadgan från

	\begin{displayquote}
		Efter divisionsstämmans sammanträde är avslutat har talmanspresidiet tjugo läsdagar på sig att publicera det justerade stämmoprotokollet.
	\end{displayquote}

	till

	\begin{displayquote}
		Efter divisionsstämmans sammanträde är avslutat har mötespresidiet tjugo läsdagar på sig att publicera det justerade stämmoprotokollet.
	\end{displayquote}

	\item ta bort ``talmanpresidiumet som helhet'' från listan i 11 kap. 3 § tredje stycket i Stadgan.

	\item ändra 11 kap. 2 b § i Stadgan från

	\begin{displayquote}
		Vid styrelsens avsättning ska en temporär styrelse väljas på innevarande sammanträde.
		Den temporära styrelsen tar över ordinarie styrelses befogenheter och skyldigheter tills dess en ny ordinarie styrelse har valts.
		Den temporära styrelsen eller talmanspresidiet ska annonsera ett extra sammanträde av divisionsstämman.
		Sammanträdet ska ske inom 15 läsdagar eller sista läsdag för terminen, vad som kommer först.
	\end{displayquote}

	till

	\begin{displayquote}
		Vid styrelsens avsättning ska en temporär styrelse väljas på innevarande sammanträde.
		Den temporära styrelsen tar över ordinarie styrelses befogenheter och skyldigheter tills dess en ny ordinarie styrelse har valts.
		Den temporära styrelsen ska annonsera ett extra sammanträde av divisionsstämman.
		Sammanträdet ska ske inom 15 läsdagar eller sista läsdag för terminen, vad som kommer först.
	\end{displayquote}

	\item ta bort 2 kap. från styrdokumentet \emph{Regler för divisionssstämman}.

	\item ändra samtliga referenser till talmanen i styrdokumentet \emph{Regler för divisionssstämman} till ``möteordförande''.

	\item ändra samtliga referenser till talmanspresidiet i styrdokumentet \emph{Regler för divisionssstämman} till ``mötespresidiet''.
\end{attsatser}


\subsubsection*{Beslut}
    \begin{attsatser}
        \item samtliga attsatser bifalles
    \end{attsatser}

\newpage
\subsection{Proposition: Mer flexibilitet i Dokumentsamlingen (andra läsningen)}

Det finns några paragrafer i divisionens regler som på ett väldigt detaljerat sätt beskriver hur man ska arbeta med officiella dokument.
De flesta av dessa regler är onödigt detaljerade, och förhindrar mer än vad de möjliggör.

Anledningen till att de kom till är därför att man ville få bukt på problemet att massor av styrdokument fanns på flera olika platser (både digitalt och fysiskt), och man inte visste vilka av dessa som fortfarande gällde.

Styrelsen anser att man bör ta bort dessa, för att de förhindrar att man hittar nya smarta lösnignar på problemet.
Det finns möjligheter i framtiden att hårdare beskriva hur dokumenten ska hanteras, men det bör inte göras nu.

Resterande regler för dokumentsamlingen passar bra in i Stadgan, och bör därmed införas där för se till att all information om något finns på ett ställe.

\subsubsection*{Förslag till beslut}

\begin{attsatser}
	\item ta bort styrdokumentet \emph{Regler för alla kommittéer}.

	\item ändra 2 § första stycket från styrdokumentet \emph{Regler för dokumentsamlingen}.

	\begin{displayquote}
		Samtliga dokument ska vara skrivna med \LaTeX~och använda föreningens dokumentklass och relevant dokumentmall.
		Dokumentklassen utvecklas och förvaltas av styrelsen.
	\end{displayquote}

	till

	\begin{displayquote}
		Samtliga officiella dokument ska använda divisionens dokumentmall.
	\end{displayquote}

	\item ta bort 2 § andra stycket från styrdokumentet \emph{Regler för dokumentsamlingen}.

	\item ta bort 3 § från styrdokumentet \emph{Regler för dokumentsamlingen}.

	\item flytta samtliga resterande paragrafer i \emph{Regler för dokumentsamlingen} till 12 kap. i Stadgan.

	\item ta bort dokumentet \emph{Regler för dokumentsamlingen}.
\end{attsatser}


\subsubsection*{Beslut}
    \begin{attsatser}
        \item samtliga attsatser bifalles
    \end{attsatser}

\newpage
\subsection{Proposition: Ersätt Dokumentsamlingen med ett Reglemente (andra läsningen)}

Vi tycker att det är enklare att hålla koll på två större dokument, än flera små dokument.
Därför föreslår vi att vi konsoliderar alla styrdokument som Divisionsstämman har beslutat om, förutom Stadgan, till ett dokument och kallar det \emph{Reglementet}.

\subsubsection*{Förslag till beslut}

\begin{attsatser}
	\item ändra 2 kap. 1 § andra stycket i Stadgan från

	\begin{displayquote}
		Samtliga organ behöver även följa de regler som definieras i styr- och principdokumenten publicerade i dokumentsamlingen.
		Undantag från divisionens styr- och principdokumenten publicerade i dokumentsamlingen kan endast fattas av det organ som antagit dokumentet, eller av ett organ överställt det organ som har antagit dokumentet.
		För att ett undantag ska kunna göras behöver man uppfylla samma krav som om man skulle ändra dokumentet.
	\end{displayquote}

	till

	\begin{displayquote}
		Samtliga organ behöver även följa divisionens styr- och principdokument som de berörs av.
		Undantag från divisionens styr- och principdokumenten kan endast fattas av det organ som antagit dokumentet, eller av ett organ överställt det organ som har antagit dokumentet.
		För att ett undantag ska kunna göras behöver man uppfylla samma krav som om man skulle ändra dokumentet.
	\end{displayquote}

	\item ändra 12 kap. 2 § i Stadgan från

	\begin{displayquote}
		Styrdokument eller principdokument får införas i dokumentsamlingen eller ändras endast efter divisionsstämman fattat beslut med två tredjedelars majoritet.

		För att ett dokument ska gälla krävs det att dokumentet är publicerat i divisionens dokumentsamling.
	\end{displayquote}

	till

	\begin{displayquote}
		Reglementet får endast ändras efter divisionsstämman fattat beslut med två tredjedelars majoritet.

		För att en regel ska gälla krävs det att dokumentet är publicerat i Reglementet.
	\end{displayquote}

	\item ändra 2 § första stycket i \emph{Regler för dokumentsamlingen} från

	\begin{displayquote}
		Dokumentsamlingen är är ett begrepp på det repository som förvarar alla publika officiella dokument som Datavetenskapsdivisionen antar.
		Syftet med dokumentsamlingen är att förvara alla dokument i ett läsbart format på ett och samma ställe.
	\end{displayquote}

	till

	\begin{displayquote}
		Reglementet ärr ett dokument som innehåller samtliga regler som Divisionsstämman har beslutat om.
		Samtliga regler ska vara i ett läsbart format på ett och samma ställe.
	\end{displayquote}

	\item flytta samtliga styrdokument som Divisionsstämman beslutat om till Reglementet.
\end{attsatser}

\subsubsection*{Beslut}
    \begin{attsatser}
        \item samtliga attsatser bifalles
    \end{attsatser}

\newpage
\subsection{Proposition: Minska antalet styrelseledamöter (andra läsningen)}

Tidigare var det ända sättet att bli aktiv i divisionen var att gå med i Styrelsen, och därmed gjorde Styrelsen allt jobb.
Det inkluderar att anordna Mottagningen, sittningar, sköta om Monaden, och viss representation mot universitet.
Följderna var att arbetsuppgifter som åligger Styrelsen att göra inte genomfördes, då man fokuserade på att till exempel anordna sittningar eller Mottagningen.
Detta ledde till en process att strukturera upp divisionen som började för några år sedan och fortsätter.
Visionen är att Styrelsen ska fokusera på sina styrelseuppdrag och se till att dessa sköts och inte försummas till förmån för andra aktiviteter.

Styrelsen vill tydliggöra detta genom att minska ner Styrelsen till endast de poster som krävs för att genomföra Styrelsens huvudsakliga uppgift.
Studiesociala aktiviteter kommer arrangeras av kommittéer, se propositionen \textit{Omstrukturera kommitéer och ta bort intressegrupper}.

\subsubsection*{Förslag till beslut}

\begin{attsatser}
	\item ta bort ``övriga styrelsemedlemmar'' från listan i 7 kap. 1 § i stadgan.
	\item ändra 7 kap. 3 § i stadgan från

	\begin{displayquote}
		Styrelsen ska högst bestå av sju medlemmar.
	\end{displayquote}

	till

	\begin{displayquote}
		Styrelsen kan högst bestå av fem medlemmar.
	\end{displayquote}

	\item ta bort 6 § från styrdokumentet \emph{Regler för styrelsen}.
\end{attsatser}

\subsubsection*{Beslut}
    \begin{attsatser}
        \item samtliga attsatser bifalles
    \end{attsatser}

\newpage
\subsection{Proposition: Omstrukturera kommittéer och ta bort intressegrupper (andra läsningen)}

Då Styrelsen fokusera på att genomföra styrelsearbete och de vanliga administrativa uppgifterna ligger det på kommittéerna att arrangera studiesociala aktiviteter och ansvarar för att det praktiska arbetet genomförs.

Styrelsen föreslår att divisionen ska ha en liknande modell på kommittéer som sektionsföreningar vid Stockholms universitets Studentkår.
Det innebär att Datavetenskapsdivisionen kommer ha flera kommittéer som har varsina arbetsområden.
Till varje kommitté ska Divisionsstämman välja en ordförande, som ska inneha posten under ett verksamhetsår.
Det är endast kommittéordföranden som väljs av Divisionsstämman.
Styrelsen kommer alltså inte ha makt att välja kommittéordförande.

Ordförande har är den som ansvarar för att kommitténs uppgifter genomförs; och välja in nya medlemmar.
Att vara ansvarig för att något sker betyder \textbf{inte} att man ska genomföra arbetet.
Ordförande är den som har till uppdrag att stämma av med styrelsen.
Utöver själva arbetsuppgifterna så ansvarar ordförande för att välja in nya medlemmar i kommittén.

Då vi gör kommittéer mycket mera flexibla försvinner syftet med intresseföreningar.
Därför finns det ingen anledning att ha kvar dem.
Det finns idag inga intresseföreningar.

\subsubsection*{Förslag till beslut}

\begin{attsatser}
	\item ta bort punkten ``antal förtroendevalda'' från listan i 9 kap. 3 § i stadgan.
	\item ändra 9 kap. 4 § från

	\begin{displayquote}
		Varje kommitté består av en ordförande och övriga förtroendevalda.

		Endast medlemmar kan väljas till förtroendeposter i en kommitté.
		Man kan endast vara ordförande för en kommitté i divisionen under den mandatperiod som man innehar posten.

		Divisionsstämman väljer ordförande och de övriga förtroendevalda till kommittéer.
		Valet av ordförande sker separat från valet av de övriga förtroendevalda.

		Vid skapande av en ny kommitté ska val av kommitténs ordförande och förtroendevalda införas i schemat för innevarande sammanträde.
	\end{displayquote}

	till

	\begin{displayquote}
		Varje kommitté består av en ordförande och övriga kommittémedlemmar.
		Endast divisionsmedlemmar kan bli medlemmar i kommittén.

		Kommittéordförande väljs av Divisionsstämman.
		Mandatperioden för kommittéordförande är densamma som divisionens verksamhetsår.

		Övriga medlemmar väljs in av kommittéordförande.
		Det finns ingen bestämd mandatperiod för övriga kommittémedlemmar.

		Vid skapande av en ny kommitté ska val av kommittéordförande införas i schemat för innevarande möte.
		Detta ska göras även om mötesschemat är fastställt tidigare under mötet.
	\end{displayquote}

	\item ta bort ``intressegrupper'' från listan i 2 kap. 1 § tredje stycket i stadgan.
	\item ta bort 10 kap. i stadgan.
\end{attsatser}

\subsubsection*{Beslut}
    \begin{attsatser}
        \item samtliga attsatser bifalles
    \end{attsatser}

\newpage

\subsection{Proposition: ny logotyp}

Den 29 oktober 2020 under styrelsemöte 4 beslutade styrelsen 2020 att den dåvarande logotypen inte längre skulle användas, på grund av att den bryter mot upphovsrättslagen.
Först och främst handlade det om att styrelsen inte vill ha en officiell logotyp som bryter mot upphovsrättslagen, och eventuellt universitetets regler för hur deras namn används.
Om våran logga bryter mot lagar, så kan det skapa problem näringslivspartners.
Sedan skulle nog inte Capcom, upphovsrättsinnehavaren av Mega Man, bli jätteglada om de fick reda på att vi använde Mega Man utan tillåtelse.

Vi diskuterade även detta under senaste divisionsstämma.

På grund av detta har vi, som ni flesta vet, satt upp de olika föreslagna logotyper i Monaden.
Ni hittar även dessa här: \verb|https://drive.google.com/drive/folders/1OLFXEvzHtClHOYIqKFPtRCFaB3OPxPBO?usp=sharing|
Vi har frågat samtliga upphovsmän om de vill ställa upp med sina förslag, och om de lämnar över upphovsrätten till divisionen om förslaget skulle väljas.

Styrelsen vill inte yrka på något specifikt logotypförslag.
Därför föreslår Styrelsen att man under stämmomötet diskuterar de olika förslagen.
Sedan ligger det på stämman att antingen

\begin{itemize}
    \item ta beslut om man ska gå vidare med ett eller flera av de befintliga förslagen;
    \item tillsätta en arbetsgrupp som har till uppgift att det till nästkommande möte finns fler och nya förslag; eller
    \item att punkten bordläggs till nästkommande möte (det vill säga, inget nytt kommer hända i frågan).
\end{itemize}

\subsubsection*{Förslag till beslut innan mötet}

\emph{Inga konkreta förslag från styrelsen innan mötet}

\subsubsection*{Förslag till beslut från stämman}

\begin{attsatser}
    \item en arbetsgrupp för att presentera logotyper skapas
    \item arbetsgruppen presenterar minst ett förslag till nästa divisionsstämmomöte
    \item styrelsen plockar fram riktlinjer för denna arbetsgrupp till och med 17 december
    \item inval av ansvariga sker under nuvarande divisionsstämma
\end{attsatser}

\subsubsection*{Diskussion}
   Alternativen som styrelsen presenterat förtydligas. Alternativet med arbetsgrupp diskuteras och förtydligas. Vad är finita automater och formella språk egentligen? Om en eventuell arbetsgrupp skulle tillsättas diskuteras hur riktlinjer för denna grupp skulle se ut och hur de bestäms. I en eventuell arbetsgrupp, vem får vara med? Måste man vara medlem i divisionen?

\subsubsection*{Beslut}
    \begin{attsatser}
        \item samtliga attsatser bifalles.
    \end{attsatser}

\newpage

\subsection{Val av ansvariga för logotyparbetsgrupp}
    Vad trevligt att vi äntligen fått logotypsvalet i gång! Nu behöver vi dock välja förtroendevalda för arbetsgruppen!

    \subsubsection*{Förslag till beslut}
        \begin{attsatser}
            \item \emph{inga förslag innan mötet}
        \end{attsatser}

    \subsubsection*{Nomineringar}
        \begin{attsatser}
            \item Josefin Kokkinakis väljs till ansvarig för logotypsarbetsgruppen
            \item Jeffrey Wolff väljs till ansvarig för logotypsarbetsgruppen
        \end{attsatser}

        \subsubsection*{Utjustering}
        Nominerade personerna justeras ut.

    \subsubsection*{Beslut}
        \begin{attsatser}
            \item Josefin Kokkinakis väljs till ansvarig för logotypsarbetsgruppen
            \item Jeffrey Wolff väljs till ansvarig för logotypsarbetsgruppen
        \end{attsatser}

    \subsubsection*{Injustering}
        De valda personerna justeras in.

\newpage

\subsection{Ajournering}
    Stämman beslutar för att ta paus klockan 18:40

\subsection{Mötet återupptas}
    Stämman beslultar för att återuppta mötet klockan 18:57

\newpage

\subsection{Proposition: Anpassa DVRKs regler till stadgeändringarna}

Den första attsatsen berör de nödvändiga ändringarna vi behöver göra för att anpassa DVRKs (mottagningskommittén) regler till stadgeändringarna i propositionen \emph{Omstrukturera kommittéer och ta bort intressegrupper}.

Syftet med den andra attsatsen handlar om att Styrelsen vill lämna det upp till DVRK att besluta hur de arbetar med IT-sektionens övergripande mottagningskommitté.

Styrelsen, genom

\emph{Albin Otterhäll\\Divisionsordförande}

\subsubsection*{Förslag till beslut}

\begin{attsatser}

\item 3 § i styrdokumentet \emph{Regler för DVRK} ändras från
    \begin{displayquote}
        Utöver ordförande ska kommittéen ha 0 – 5 förtroendevalda.

        Mandatperioden för samtliga förtroendevalda i kommittéen är densamma som kalenderåret.
    \end{displayquote}

    till

    \begin{displayquote}
        Mandatperioden för kommittémedlemmarna är densamma som kalenderår.
    \end{displayquote}

    \item 4 § i styrdokumentet \emph{Regler för DVRK}
        \begin{displayquote}
            DVRKs ordförande ska delta i kårens möten som behandlar mottagningen.
            Hen ska även sitta i IT-sektionens insparkskommitté.
        \end{displayquote}

        tas bort.
    \item beslut om ordförande gäller för verksamhetsåret 2022.

\end{attsatser}

\subsubsection*{Beslut}
    \begin{attsatser}
        \item samtliga attsatser bifalles
    \end{attsatser}




\subsection{Inval av ordförande till DVRK verksamhetsåret 2022}

    \subsubsection*{Nomineringar}

        \begin{itemize}
            \item  Kristoffer Gustafsson nominerar sig själv
        \end{itemize}

        \subsubsection*{Utjustering}
        Den nominerade justeras ut

        \subsubsection*{Beslut}
            \begin{attsatser}
                \item Kristoffer Gustafsson väljs till ordförande för DVRK
            \end{attsatser}

        \subsubsection*{Injustering}
        Den nominerade justeras in

\newpage

\subsection{Proposition: Ta bort LeK}

Syftet med LeK är detsamma som den föreslagna kommittén \emph{ConCats}.
Då ingen idag sitter i LeK; LeKs regler inte är anpassade efter de kommande stadgeändringarna; och personerna som är bakom motionen för att skapa ConCats önskar sitta i kommittén ser vi ingen poäng med att behålla LeK, som idag endast är en pappersprodukt.

Styrelsen, genom

\emph{Albin Otterhäll\\Divisionsordförande}

\subsubsection*{Förslag till beslut}

\begin{attsatser}
    \item Att styrdokumentet \emph{Regler för LeK} tas bort i sin helhet.
\end{attsatser}

\subsubsection*{Beslut}
    \begin{attsatser}
        \item samtliga attsatser bifalles
    \end{attsatser}



\subsection{Motion: Skapa festkommitté}

    Det är ingen hemlighet att Datavetenskapsdivisionen behöver en festkommitté ty festen är en viktig del för studentlivet.
    Därför är det inte så konstigt att det idag finns stort engagemang att både arrangera och delta i sådana fester. \\
    Halloweensittningen är ett bra exempel på detta då 36 personer deltog. \\
    Förutom det faktum att det är trevligt med festligheter, skapar evenemang även en sammanhållning inom divisionen vilket är viktigare än någonsin speciellt nu efter covid-19. \\
    Kommittén kommer att likna andra sexmästerier. \\
    Kommittén har för avseende att anordna sittningar, pubrundor och fester bland annat. \\
    För välmående av hela divisionen i framtiden så anser vi också att det är fundamentalt att vi har en festkommitté. \\
    Detta är såklart på grund av de tidigare nämnda fördelarna men också för att upprätthålla den yttre bilden av divisionen.

    \emph{Kristoffer Gustafsson\\DV'20}

    \emph{Anthon Wirback\\DV'21}

    \emph{Josefin Kokkinakis\\DV'21}

    \emph{Leo Mirzajanzadeh\\DV'21}

    \emph{Petter Blomkvist\\DV'18}

    \emph{Lukas Gartman\\DV'20}

    \emph{Lucas Gyllensvaan\\DV'21}


    \subsubsection*{Förslag från motionären}

    \begin{attsatser}
        \item Att i avsnittet med kommittéer i reglementet, eller som självständigt dokument i dokumentsamlingen, införa följande

        \begin{displayquote}
            \subsection*{Mega6}
            \begin{enumerate}[label=\arabic* §]
                \item Kommitténs namn är Mega6.

                \item Mega6 har till uppgift att anordna festligheter.

                \item Mandatperioden för kommittémedlemmarna bestäms internt inom kommittén.
            \end{enumerate}
        \end{displayquote}
    \end{attsatser}

    \subsubsection*{Styrelsens svar på motion}
    Styrelsen tycker att det är väldigt roligt att medlemmar är intresserade att starta upp ett sexmästeri!
    Vad är studentlivet utan festligheter?

    \subsubsection*{Förslag från styrelsen}
        \begin{attsatser}
            \item motionens samtliga attsatser bifalles
        \end{attsatser}


    \subsubsection*{Beslut}
        \begin{attsatser}
            \item samtliga attsatser bifalles
        \end{attsatser}

\subsection{Inval av ordförande till Mega6 verksamhetsåret 2021 och 2022}

    \subsubsection*{Nomineringar}

        \begin{itemize}
            \item Anthon Wirback nominerar sig själv
        \end{itemize}

        \subsubsection*{Utjustering}
        Den nominerade justeras ut

        \subsubsection*{Beslut}
            \begin{attsatser}
                \item Anthon Wirback väljs till ordförande för Mega6
            \end{attsatser}

        \subsubsection*{Injustering}
        Den nominerade justeras in

\newpage

\subsection{Motion: Skapa ConCats}

Datavetenskapsdivisionen behöver ConCats! ConCats har till uppgift att arrangera
evenemang i Monaden samt arbeta för att skapa en trivsam miljö. Exempel på detta är: \\
studiesociala aktivitetskvällar samt ansvar för divisionens lokal Monaden. \\
Monaden är en samlingspunkt för oss inom Datavetenskap och därför känns det viktigt att
skapa en social samt trevlig miljö där alla kan känna sig välkomna. Att arrangera
evengemang har stor betydelse för den sociala aspekten och ger större möjligheter att lära
känna medstudenter inom programmet. I förlängningen tror vi att en stärkt gemenskap kan
verka för att förenkla och förbättra studietiden.

    \subsubsection*{Förslag från motionären}

        \begin{attsatser}
        \item Att i avsnittet med kommittéer i reglementet, eller som självständigt dokument i dokumentsamlingen, införa följande

        \begin{displayquote}
            \subsection*{ConCats}
            \begin{enumerate}[label=\arabic* §]
                \item ConCats arbetar för upprätthålla en studiesocial miljö i och runt Monaden.

                \item Kommittémedlemmarna bestäms internt inom kommittén.

                \item ConCats är av styrelsen anförtrodd att ta hand om divisionslokalen Monaden.
            \end{enumerate}
        \end{displayquote}
    \end{attsatser}

    \subsubsection*{Styrelsens svar på motion}

Styrelsen ser verkligen ett behov av att ha en kommitté som ansvarar för att Monaden sköts om, och att det arrangeras aktiviteter för att stärka programtillhörigheten.
Därför tycker vi att det är väldigt roligt att det finns drivna personer som önskar starta upp ett rustmästeri och PR-kommitté!

Enligt Stadgan ska kommitténs namn finnas med i kommitténs regler.
Då de andra kommittéernas första paragrafer alla explicit nämner vad kommitténs namn är, så tycker vi att det är rimligt att det även står i denna kommitténs regler.

Olyckligtvis så missades det att föreslå att punkten som ställer kravet att kommittéernas regler ska specificera kommittémedlemmarnas mandatperiod ska tas bort.
Detta leder till att kommittéernas regler måste innehålla en sådan specifikation, även om det är helt onödigt då det redan kommer stå det i Stadgan i en annan paragraf.
Detta kan tyvärr inte åtgärdas nu, utan det får göras på kommande Divisionsstämmomöten.

Vi önskar även poängtera att det inte av Styrelsen som kommittén skulle få sitt eventuella ansvar för Monaden, utan från Divisionsstämman.
Styrelsen vill göra det tydligt att makten utgår ifrån er medlemmar, och inte från Styrelsen.
Därför önskar Styrelsen att den tredje föreslagna paragrafen inte tas med, och att ansvara för Monaden ska tolkas implicit som en del av motionärens första föreslagna paragraf (den andra paragrafen i Styrelsens yrkande).

        \subsubsection*{Förslag från styrelsen}

        \begin{attsatser}
        \item Att i avsnittet med kommittéer i reglementet, eller som självständigt dokument i dokumentsamlingen, införa följande

        \begin{displayquote}
            \subsection*{ConCats}
            \begin{enumerate}[label=\arabic* §]
                \item Kommitténs namn är ConCats.

                \item ConCats arbetar för upprätthålla en studiesocial miljö i och runt Monaden.

                \item Mandatperioden för kommittémedlemmarna bestäms internt inom kommittén.

                \item Kommittémedlemmarna bestäms internt inom kommittén.
            \end{enumerate}
        \end{displayquote}
    \end{attsatser}

        \subsubsection*{Ändringsyrkanden}
        \begin{attsatser}
            \item ändra den föreslagna 2 § i första attsatsen från

               \begin{displayquote}
                   ConCats arbetar för upprätthålla en studiesocial miljö i och runt Monaden.
               \end{displayquote}
                till
               \begin{displayquote}
                   ConCats har till uppgift att ansvara för att divisionens sektionslokal tas om hand, och för att skapa gemenskap på de program som divisionen riktar sig emot.
               \end{displayquote}

            \item ta bort den föreslagna 4 § i första attsatsen som lyder

                \begin{displayquote}
                    Kommittémedlemmarna bestäms internt inom kommittén.
                \end{displayquote}
            \item ändra \emph{ConCats} till \emph{Omega} i motionens första attsats

        \end{attsatser}

    \subsubsection*{Diskussion}
    Namnändringsförslaget diskuteras och anledningen till varför förslaget presenteras förtydligas
    ConCats förklarar varför och hur deras namn kom fram.

    \subsubsection*{Beslut}
        \begin{attsatser}
            \item första attsatsen från ändringsyrkandet bifalles.
            \item andra attsatsen från ändringsyrkandet bifalles.
            \item tredje attsatsen från ändringsyrkandet avslås.
            \item 1 § och 3 § i \emph{Förslag från styreslen} bifalles.
        \end{attsatser}

\newpage

\subsection{Inval av ordförande till ConCats verksamhetsåret 2021 och 2022}

\subsubsection*{Diskussion}
Hur stämman går till väga om ordförande för kommittén vill avgå mitt under ett verksamhetsår diskuteras.

    \subsubsection*{Nomineringar}

        \begin{itemize}
            \item Miranda Jernberg nominerar sig själv
        \end{itemize}

        \subsubsection*{Utjustering}
        Den nominerade justeras ut

        \subsubsection*{Beslut}
            \begin{attsatser}
                \item Miranda Jernberg väljs till ordförande för ConCats.
            \end{attsatser}

        \subsubsection*{Injustering}
        Den nominerade justeras in


\newpage

\subsection{Motion: Skapa arbetsmarknadskommitté}

Datavetenskapsdivisionen behöver en arbetsmarknadskommitté! En arbetsmarknadskommitté har till uppgift att arrangera evenemang tillsammans med arbetsgivare. Exempel på arrangemang är lunchföreläsningar; mingelkvällar; besök hos arbetsgivare; och allmän marknadsföring i våra kanaler.

Det finns idag massor av företag som skriker efter mjukvaruutvecklare, samtidigt som de flesta datavetarna är osäkra på vilka arbetsgivare som finns där ute. DVarms (Datavetenskaps arbetsmarknadskommitté) uppdrag kommer att vara att försöka hjälpa studenter och arbetsgivarna att hitta varandra.

Jag har hört att många undrar om vi kan driva en arbetsmarknadskommitté likt Chalmerssektionerna, då Chalmerssektionerna oftast har fler studenter än vad vi har på Datavetenskap.

Det är sant att vi har färre studenter än Datateknik (civilingenjör) och Informationsteknik. Samtidigt hade vi inte ett mycket mindre intag än Elektroteknik hade för bara två år sedan, och de kunde driva en arbetsmarknadskommitté. Och det finns även inget som hindrar DVarm att planera in sina arrangemang så att även studenter från Datateknik och Informationsteknik kan komma, arbetsgivare kommer inte direkt klaga om Chalmersstudenter närvarar på våra lunchföreläsningar.

Genom att erbjuda tjänster till arbetsgivare för att nå ut till studenter kan vi ta betalt, och för oss handla det om stora summor. Det är nästan uteslutande från arbetsmarknadskommittéerna som Chalmerssektionerna får sina pengar.

\emph{Albin Otterhäll\\DV'18; Divisionsordförande, Datavetenskapsdivisionen '19, '20, '21; Mottagningsansvarig, Datavetenskapsdivisionen '19; Eventansvarig, IT-sektionens mottagningskommitté '19; Sekreterare och förrådsansvarig, Recentiorskommittén '20}

    \subsubsection*{Diskussion}
    Vad en arbetsmarknadskommitté är förtydligas. Varför vi skulle vilja ha en arbetsmarknadskommitté förtydligas också.
    Pengar och roliga evenemang presenteras som två stora punkter en arbetsmarknadskommitté kan tillföra.

    \subsubsection*{Förslag från motionären}

        \begin{attsatser}
        \item Att i avsnittet med kommittéer i reglementet, eller som självständigt dokument i dokumentsamlingen, införa följande
        \begin{displayquote}
            \subsection*{DVarm}
            \begin{enumerate}[label=\arabic* §]
                \item Kommitténs namn är DVarm, vilket står för Datavetenskaps arbetsmarknadskommitté

                \item DVarm har till uppgift att arbeta för samarbeten mellan divisionen och arbetsmarknaden

                \item Mandatperioden för kommittémedlemmarna bestäms internt inom kommittén
            \end{enumerate}
        \end{displayquote}
    \end{attsatser}

    \subsubsection*{Styrelsens svar på motion}
    Styreslen tycker att en arbetskomitté hade varit toppen!
    Mer samarbeten med företag och roliga event sitter aldrig fel!

    Styrelsen, genom

    \emph{Samuel Hammersberg\\Vice divisionsordförande}

    \subsubsection*{Förslag från styrelsen}
        \begin{attsatser}
            \item motionens samtliga attsatser bifalles
        \end{attsatser}

    \subsubsection*{Beslut}
        \begin{attsatser}
            \item samtliga attsatser bifalles
        \end{attsatser}


\subsection{Inval av ordförande till DVArm verksamhetsåret 2022}

\subsubsection*{Diskussion}
Hur stämman går till väga om ordförande för kommittén vill avgå mitt under ett verksamhetsår diskuteras.

        \subsubsection*{Mötesordförandesöverlämning}
        Mötesordförande (Albin Otterhäll) överlämnar rollen som mötesordförande till Samuel Hammersberg

        \subsubsection*{Nomineringar}
            \begin{itemize}
                \item Albin Otterhäll nominerar sig själv
            \end{itemize}

        \subsubsection*{Utjustering}
        Den nominerade justeras ut

        \subsubsection*{Beslut}
            \begin{attsatser}
                \item Albin Otterhäll väljs till ordförande
            \end{attsatser}

        \subsubsection*{Injustering}
        Den nominerade justeras in

        \subsubsection*{Mötesordförandesöverlämning}
        Mötesordförande (Samuel Hammersberg) överlämnar rollen till Albin Otterhäll

\newpage

\subsection{Val av divisionsordförande}

Den nuvarande styrelsens mandatperiod tar slut den 31 december, och Albin önskar inte fortsätta att sitta i Styrelsen under verksamhetsåret 2022.
På grund av detta så behöver vi välja en ny divisionsordförande.

Styrelsen föreslår Samuel Hammersberg (DV'20) som ny divisionsordförande för verksamhetsåret 2022.
Han är idag vice divisionsordförande, och har visat att han är driven; ansvarsfull; och ledarkraftig.
Vi har fullt förtroende för honom, och är säkra på att divisionen kommer uppnå nya höjdpunkter med honom som ordförande.

Styrelsen, genom

\emph{Albin Otterhäll\\Divisionsordförande}

        \subsubsection*{Förslag till beslut}
        \begin{attsatser}
            \item Divisionsstämman väljer Samuel Hammersberg (DV'20) till divisionsordförande för verksamhetsåret 2022.
        \end{attsatser}

        \subsubsection*{Utjustering}
        Den nominerade justeras ut

        \subsubsection*{Beslut}
            \begin{attsatser}
                \item Samuel Hammersberg (DV'20) till divisionsordförande för verksamhetsåret 2022.
            \end{attsatser}

        \subsubsection*{Injustering}
        Den nominerade justeras in

\subsection{Inval till styrelsen verksamhetsår 2022}

Vid divisionsstämmans möte den 15 september beslutade divisionen att välja in Samuel Hammersberg (vice ordförande), Sebastian Selander (sekreterare), och Tekla Siesjö (SAMO) till styrelsen för verksamhetsåret 2021

På grund av att dessa personer endast röstades in för verksamhetsåret 2021 behöver vi under detta möte även rösta in styrelseledamöter för verksamhetsåret 2022.

Vi har ännu inte hittat en ersättningskandidat för vice ordföranderollen, och kan därav inte föreslå någon ännu.


        \subsubsection*{Förslag till beslut}
        \begin{attsatser}
            \item Sebastian Selander (DV'20); Tekla Siesjö (DV'20); Morgan Thowsen (DV'18) väljs in till styrelsen för verksamhetsåret 2022
        \end{attsatser}

        \subsubsection*{Mötessekreteraröverlämning}
        Mötessekreterare (Sebastian Selander) överlämnar posten till Jeffrey Wolff

        \subsubsection*{Utjustering}
        De nominerade justeras ut

        \subsubsection*{Beslut}
        \begin{attsatser}
            \item Sebastian Selander (DV'20); Tekla Siesjö (DV'20); Morgan Thowsen (DV'18) väljs in till styrelsen för verksamhetsåret 2022
        \end{attsatser}

        \subsubsection*{Injustering}
        De nominerade justeras in

        \subsubsection*{Mötessekreteraröverlämning}
        Mötessekreterare (Jeffrey Wolf) överlämnar posten till Sebastian Selander

\newpage

\subsection{Val of revisorer för verksamhetsåret 2021}
Vi behöver revisorer för 2022 så att stämman har koll på att styrelsen sköter sitt jobb!

        \subsubsection*{Beslut}
            \begin{attsatser}
                \item posten vakantsätts
            \end{attsatser}

\newpage

\section{Diskussioner}\label{sec:discussioner}

\subsection{Slaget om IT-sektionen}

IT-sektionsstyrelsen har satt upp en arbetsgrupp för att planera ''Slaget om IT-sektionen'' den 21 januari 2022.
Det ska vara en tävling mellan alla programmen under IT-fakulteten. \\
Huvudansvariga för arrangemanget är sektionsstyrelsen. \\
I dagsläget har inga specifika aktiviteter planerats, och vi har inte fått någon information om vilka som sitter i arbetsgruppen.
Men de har hyrt två sporthallar på Lindholmen från klockan 15:00 till klockan 17:30. \\
Styrelsen har inte fått mer information än detta.

Styrelsen under om någon är intresserad av att delta i arbetsgruppen. \\
Om någon är intresserad så är det bara att kontakta Styrelsen, så kan vi skicka kontaktuppgifterna till sektionsstyrelsen.

\newpage

\section{Avslutande av möte}

Mötet avslutades kl. 20:20

\stämmosignaturer

\end{document}
