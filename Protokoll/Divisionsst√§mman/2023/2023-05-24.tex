\documentclass[protokoll]{dvd}

\KOMAoptions{
	headwidth = 18cm,
	footwidth = 18cm,
}

\begin{document}


\title{Divisionsstämmans möte 2}
\subtitle{2023}
\author{Divisionsstämman}
\date{2023-05-24}

\textbf{Datum:} \csname @date\endcsname\\
\textbf{Tid:} 17:17\\
\textbf{Plats:} Monaden (ev. EA)\\
\textbf{Styrelsemedlemmar:}
\begin{närvarande_förtroendevalda}
\förtroendevald{Ordförande}{Samuel Hammersberg}{}
\förtroendevald{Vice ordförande}{\emph{Vakant}}{}
\förtroendevald{SAMO}{Tekla Siesjö}{}
\förtroendevald{Kassör}{Lukas Gartman}{}
\förtroendevald{Sekreterare}{\emph{Vakant}}{}
\end{närvarande_förtroendevalda}

\textbf{Närvarande övriga medlemmar:}
\begin{närvarande_medlemmar}
\medlem{Albin Otterhäll}
\medlem{Anthon Wirback}[lämnade vid punkt 4.9]
\medlem{Tim Persson}
\medlem{Emmie Berger}[lämnade vid punkt 4.10]
\medlem{Leo Mirzajanzadeh}[lämnade vid punkt 4.12]
\medlem{Lukas Gartman}
\medlem{William Bodin}
\medlem{Kristoffer Gustafsson}
\medlem{Samuel Hammersberg}
\medlem{Sebastian Selander}
\medlem{Tekla Siesjö}[lämnade vid punkt 4.10]
\medlem{Oscar Rei}[lämnade vid punkt 4.9]
\medlem{Sebastian Pålsson}[lämnade vid punkt 4.3]
\medlem{Nils Karl Teodor Lyrevik}
\medlem{Omar Ahmad}[lämnade vid punkt 3.3]
\medlem{Alexander Kjellberg}[lämnade vid punkt 3.2]
\medlem{Oliver Ivarsson}[lämnade vid punkt 3.2]
\medlem{Karl Nilsson}[lämnade vid punkt 4.3]
\medlem{Josefin Kokkinakis}[lämnade vid punkt 4.7]
\medlem{Alva Johansson}[lämnade vid punkt 4.7]
\end{närvarande_medlemmar}

\newpage
\section{Öppnande av möte}
Mötet öppnades av Samuel Hammersberg 17.17

\section{Formalia}

\subsection{Divisionsstämmans beslutbarhet}

6 kap. i stadgan definierar regler Divisionstämman.

Den 8e maj kallade styrelsen till divisionsstämma genom att skriva i discordservern \emph{MonadenDV};

Tyvärr kunde inte ett utskick ske över mail då divisionen har förlorat
access till studenternas maillistor då när Göteborgs Universitet
bytte över till Outlook, ändrades de även reglerna angående mailutskick
till maillistor. Styrelsen jobbar med att fixa access igen.

Dessa möteshandlingar skickades ut under måndagen den 22e maj i discordservern \emph{MonadenDV}.

\subsubsection*{Förslag till beslut}

\begin{attsatser}
    \item divisionsstämman har uppnått kraven i stadgan för att få hålla möte, och är därmed beslutbar.
\end{attsatser}

\subsubsection*{Beslut}

\begin{attsatser}
    \item attsatsen bifalles
\end{attsatser}

Attsatsen godkändes.

\subsection{Fastställande av mötesschema}

För att divisionsstämman ska kunna fatta ett beslut eller protokollföra en diskussion behöver punkten i mötesschemat där stämman ska fatta beslut vara inlagd eller föras in i mötesschemat senast vid den här punkten.

\subsubsection*{Förslag till beslut}

\begin{attsatser}
    \item införa fastställande av röstlängden som punkt 2.1 i dagordningen.
    \item införa motionen "uthyrning av skåp till divisionsmedlemmar" som punkt 4.11 i dagordningen.
\end{attsatser}

\subsubsection*{Beslut}

\begin{attsatser}
    \item attsatsen bifalles
	\item attsatsen bifalles
\end{attsatser}

\subsection{Val av mötesordförande}
Mötesordförande har till uppgift att leda Divisionsstämmans sammankomst.
Hen ansvarar för att mötesformalia sköts.
\subsubsection*{Förslag till beslut}
\begin{attsatser}
    \item Samuel Hammersberg väljs till mötesordförande.
\end{attsatser}
\subsubsection*{Beslut}
\begin{attsatser}
    \item attsatsen bifalles
\end{attsatser}

\subsection{Val av vice mötesordförande}
Vice mötesordförande hjälper mötesordförande med att hålla talarlistan, och att alla får komma till tals.
\subsubsection*{Förslag till beslut}
\begin{attsatser}
    \item Lukas Gartman väljs till vice mötesordförande
\end{attsatser}
\subsubsection*{Beslut}
\begin{attsatser}
    \item attsatsen bifalles
\end{attsatser}

\subsection{Val av mötessekreterare}
Mötessekreteraren har till uppgift att anteckna diskussioner, beslut, och eventuella reservationer under mötet.
\subsubsection*{Förslag till beslut}
\begin{attsatser}
    \item Tekla Siesjö väljs till mötessekreterare.
\end{attsatser}
\subsubsection*{Beslut}
\begin{attsatser}
    \item attsatsen bifalles
\end{attsatser}

\subsection{Val av protokolljusterare}
Protokolljusterare har till uppgift att kontrollera att protokollet i slutändan reflekterar de
faktiska besluten och diskussionerna som fördes under sammanträdet; samt agera rösträknare vid slutna omröstningar.
Utöver protokolljusterarna så ska mötesordförande och mötessekreteraren signera protokollet.
Vid Divisionsstämmans sammanträden ska det vara två justerare.
Mötesordförande och mötessekreteraren kan inte vara justerare.
\subsubsection*{Förslag till beslut}
\begin{attsatser}
    \item Tim Persson väljs in som protokolljusterare
    \item Kristoffer Gustavsson väljs in som protokolljusterare
\end{attsatser}
\subsubsection*{Beslut}
\begin{attsatser}
    \item samtliga attsatser bifalles
\end{attsatser}

\newpage

\section{Rapporter}
\subsection{Styrelsen}
\subsubsection*{Verksamhetsrapport}
Sedan sista stämman den 22e februari styrelsen haft 3 möten.
Protokollen från mötena finns (snart) tillgängliga för allmänheten på styrelsens drive.
Länk till styrdokument och protokoll:\\
\verb|https://drive.google.com/drive/folders/1o_lLM7g7ph-xdxgut3E_BU0ywichWYTX?usp=sharing|

\subsubsection*{Ordförande}
Det har varit ett roligt år här tillsammans på Datavetenskapsdivisionen.
Jag har personligen varit på institutionsråd och på programråd!
På instutionsrådet diskuterades saker såsom hur student satisfaction concerning
teaching, där jag rapporterade och skickade vidare den enkät jag hade skickat ut!

På programrådet diskuterades ett flertal punkter men primärt hur många som klarar
av och går ut ur våra program, balansen mellan kön, men även förslaget om att
lägga till en kurs vilket skulle gå över datavetenskapens historia. Detta var
ett förslag ett par av mötesdeltagarna tyckte väldigt mycket om!

Jag ser väldigt framemot kommande året, och hoppas på att vi kommer
starta det med en dunder mottagning!

Lycka till på tentorna och ha en trevlig sommar!

MvH

Samuel Hammersberg

\subsubsection*{SAMO}
Min sista rapport som SAMO i Datavetenskapsdivisionen.
I skrivande stund har vi från Roger fått 2.0 mail, innehållet i dessa
kommer utvecklas i punkten om väggen!

Utöver det så har medlemslistorna uppdaterats, elever pratats
med och Göta kommunicerats med! Knappar har felanmälts och mikrovågsugnar
har satts igång igen, protokoll har skrivits och möten har hållts i!

Det har varit kul, peace out $:victory\_sign:$

MvH

Tekla "$victory\_sign$" Siesjö

\subsubsection*{Ekonomi}
Det är med glädje jag presenterar min första rapport som ny kassör i Datavetenskapsdivisionen.
I skrivande stund har vi från Göta äskat 6,7tkr som gått till kaffe och material till Monaden
samt lite roliga evenemang med kommittéerna.

Utöver det som rapporterades under förra stämman så har det spenderats ytterligare 6,9tkr från
Götas mottagningsbudget till DVRKs aspning och kläder samt en filmkväll. Den budgeten är härmed stängd.

Bokförings- och fakturasystemet som föregående kassör (Morgan) satte upp behålls i ett år till
men kommer att behöva bytas ut till en gratisvariant vid årsskiftet då det nuvarande systemet
anses vara för kostsamt gentemot storleken på vår organisation.

Det har varit en långsam övergång som kassör med ett flertal besök hos banken utan lycka.
Det är i skrivande stund ett ärende som behandlas och vi kommer så småningom att kunna
bedriva verksamheten som vanligt. Jag vill passa på att tacka Morgan och Albin så mycket
för deras tid och energi som har lagts ned till att hjälpa mig under denna övergångsperiod!

MvH

Lukas Gartman

\newpage

\subsection{Verksamhetsrapporter av kommittéer}
\subsubsection*{DVRK}
Vi har sedan förra stämman påbörjat planeringen för höstens Inspark och har haft
stort fokus på att skapa samarbeten med andra föreningar under Göta Studentkår.
Vi har försökt att aktivt marknadsföra oss själva och göra oss synliga för att
andra studenter ska känna till DVRK och Datavetenskapliga programmet.
Vi höll även en sittning för att väcka intresse för att vara phadder
till hösten mottagning

- Tim “Båtsman” Persson, Ordförande DVRK'23

\subsubsection*{ConCats}
Vi har fortsatt jobba aktivt med att ta hand om monaden.
Som resultat av aspnings filmkvällen har vi nu vuxit med 4 nya medlemmar. 
Vi ska fortsätta växa under kommande år och under mottagningen har vi
bl.a planerat en ConCatsCup i slottskogen där reccarna kommer tävla för
ett pris och en lagbild att hänga upp i Monaden, går detta bra kan 
det bli en årlig tradition, något vi vill ha mer av här på DV. I framtiden
vill vi vara öppna till möjligheten att välkomna fler engelska talande studenter i vår kommitté

- Alex ":D" Lisborg, Ordförande ConCats'22

\subsubsection*{Femme++}
Femme++ hade tyvärr inga aktiviteter efter mottagningen under höstterminen 2022.
Men under våren 2023 har vi dragit igång! En mycket lyckad karaokekväll ute på stan och en mysig middag hemma hos Tekla.
Flera av medlemmarna besökte DataTjej-konferensen i mars (DataTjej är en förening för kvinnor och ickebinära inom IT och Data).
Vi har också skapat en grupp på fyra personer som kommer ta över efter Tekla och Emmie som snart slutar studera.
Vi har börjat planera mottagningen och är mycket taggade inför nästa år!

- Emmie Berger, Ordförande Femme++'22

\subsubsection*{DVArm}
Ingen rapport.

\subsubsection*{Mega6}
Muntlig verksamhetsrapport given av representant från kommittéen.

\subsubsection*{Uppdrag}
Det vi i studienämnden gjort är att medverka på de flesta kursnämndsmöten för DV23s obligatoriska kurser. Vi har under dessa möten fört egna protokoll, huvudsakligen för att protokollen som skrivs av Chalmers inte till fullo representerar åsikterna från studenterna på DV. Innan kursnämndsmötena så har vi pratat med studentrepresentant och några studenter som läst kursen under läsperioden för att få en helhetsbild och för att se till att all kritik förmedlas till ansvariga. Vi har även läst igenom de tidigare protokollen vi haft tillgång till för att under kursnämndsmötena kunnat diskutera med ansvariga huruvida klagomålen och lovade åtgärder från tidigare år åtgärdats. Utifrån anteckningar jag fört under tidigare kursnämndsmöten har vi även kunnat jämföra betygsfördelningen och ifrågasätta förändringar. Dessvärre kunde vi ej medverka på alla kursnämndsmöten då vi inte fick information om mötena i god tid. Efter kontakt med utbildningssekreteraren på Chalmers så har vi nu kommit med i utskickslistan inför kursnämndsmöten som hålls av Chalmers. Vår förhoppning är nu att vi kommer få information om datum för mötena i god tid och därmed även hinna prata med fler studenter från DV som läst kursen under den läsperioden.

Utöver ovanstående så har jag (Josefin) under årets gång medverkat på programråd och fört dialoger med examinatorer, programansvarig samt ordförande från Göta studentkårs IT-sektion.

Min största bedrift har varit ett extra inlagt examinationstillfälle som tidigare år saknats trots att vi enligt \textit{regeler för examination på grundnivå och avancerad nivå vid Göteborgs universitet} haft rätt till det.

Kommittén är dock relativt ny och så jag hoppas på att hinna med mer nästa verksamhetsår.

- Josefin Kokkinakis, Ordförande för Studienämnd

\subsubsection*{Flytten från Lindholmen}
Förra stämman diskuterades det om hur flytten från Lindholmen skulle gå till.
Under denna tiden har Samuel Hammersberg kontaktat både Alex Gerdes, vilket ansvarar för Datavetenskaps programmet
och Karl Nilsson, vilket är sektionsordförande för IT-sektionen på Göta.
Alex hade tyvärr inte några nyheter då det är för tidigt, och
Karl skickade ett långt protokoll som inkluderade en förstudie(sida 60+) om hur detta ska gå till!
Tyvärr har Samuel inte haft tid nog att gräva sig ner i dokumentet, men har skickat det
i discordservern \emph{MonadenDV}, för de som är sugna på att läsa!

\newpage

\section{Beslutsärenden}

Enligt Stadgan måste ändringar av Stadgan röstas igenom på två av Divisionsstämmans varandra följande möten.
Om en beslutpunkt innehåller ``första läsningen' innebär det att det är första gången beslutet tas upp för omröstning.
Om en beslutspunkt innehåller ``andra läsningen'' innebär det att beslutspunkten har röstats igenom förra stämmomötet, och beslutet behöver bekräftas för att gå igenom.

\subsection{Proposition: Uppdatera reglementet}
Just nu ligger lite saker i reglementet på fel plats och detta vill vi rätta till!
ConCats och DVArm verkar ha råkat mixrat ihop sina arbetsbördor!

\subsubsection*{Förslag till beslut}
\begin{attsatser}
    \item ändra kapitel 7s titel i \emph{Reglementet} från
    \begin{displayquote}Regler för ConCats\end{displayquote}
    till
    \begin{displayquote}Regler för DVArm\end{displayquote}

    \item ändra 7 § 1 \emph{Reglementet} från
    \begin{displayquote}
        Kommitténs namn är ConCats, vilket står för Datavetenskaps arbetsmarkands-kommitté.
    \end{displayquote}
    till
    \begin{displayquote}
        Kommitténs namn är DVArm, vilket står för Datavetenskaps arbetsmarkands-kommitté.
    \end{displayquote}

    \item ändra kapitel 8s titel i \emph{Reglementet} från
    \begin{displayquote}Regler för DVArm\end{displayquote}
    till
    \begin{displayquote}Regler för ConCats\end{displayquote}

\end{attsatser}

\subsubsection*{Beslut}
\begin{attsatser}
    \item samtliga attsatser bifalles
\end{attsatser}
Alla attsatser avslås då styrelsen har rätt att göra redaktionella ändringar i styrdokument som tagits 
fram av divisionsstämman, enligt punkt 10.3 i stadgan.

\subsection{Rivning av väggen}
Styrelsen fick i uppdrag att utreda huruvida rivning av väggen i monaden är möjligt,
detta behöver nu diskuteras efter att alla delgetts detaljerna kring situationen.
Detta inleddes av en muntlig rapport om situationen av Tekla Siesjö.
Styrelsen hade fått tillbaka en rittning av lokalen och stämman kunde då diskutera 
vilka väggar som skulle rivas. 

\subsubsection*{Förslag till beslut}
\begin{attsatser}
	\item \emph{inga förslag från styrelsen innan mötet}
\end{attsatser}
\subsubsection*{Förslag från stämman}
\begin{attsatser}
	\item styrelsen skickar in en offert för att riva köksväggen och mellanväggen
	\item styrelsen skickar in en offert för att riva mellanväggen
	\item styrelsen skickar in en offert för att riva köksväggen
\end{attsatser}
\subsubsection*{Beslut}
\begin{attsatser}
    \item samtliga attsatser från \emph{Förslag från stämman} bifalles
\end{attsatser}

\subsection{Motion: Arbetet för logga arbete}
Förra stämman så valdes det att förlänga arbetstiden för arbetsgruppen som fixar våran nya logga!
Idag är de här och ska presentera vad de har kommit fram till!
\subsubsection*{Förslag till beslut}
\begin{attsatser}
    \item välja den grafiska profil som arbetsgruppen tagit fram
\end{attsatser}
% \subsubsection*{Ändringsyrkanden}
% \begin{attsatser}
% 	\item att arbetet ska fortsätta och texten på loggan ska ändras för att bli mer symmetrisk 
% \end{attsatser}
\subsubsection*{Beslut}
\begin{attsatser}
	% \item attsatsen från ändringsyrkandet avslås
    \item attsatsen bifalles
\end{attsatser}

\subsection{Motion: Inval av ordförande till Mega6}
Avsaknaden av events från Mega6 under våren 2023 har inte undgått någon,
allra minst Mega6s medlemmar. Vi i Mega6 har haft flera bakslag, motgångar
och svårigheter under denna tid. Därför föreslår jag mig själv, KG,
till att väljas som ordförande för Mega6. Mitt mål är att mottagningen
och därefter aspningen utförs och att de når upp till förväntningarna.

- Kristoffer Gustafsson DV'20
\subsubsection*{Förslag till beslut}
\begin{attsatser}
    \item Kristoffer Gustafsson väljs till ordförande för Mega6.
\end{attsatser}
\subsubsection*{Beslut}
\begin{attsatser}
    \item attsatsen bifalles
\end{attsatser}

\subsection{Motion: Inval av ordförande till Femme++}
Det är viktigt att ha en plats där de som i nuläget tillhör dessa minoritetsgrupper
på DV kan uttrycka sin åsikt och prata om saker som de annars känner är svårt att prata om.

Det är flera personer som uttryckt att anledningen till att de inte tar del av studie-livet
på skolan är för de har svårt att hitta sin plats och uttrycka sig när de befinner sig i en minoritet.

Genom lite mindre sociala evenemang samt aktiviteter som workshops och självförsvar så hoppas
vi stärka dessa personers kraft att uttrycka sig även i andra sociala sammanhang eller bara
känna att de har någon plats att vara med på, i divisionen.

- Josefin Kokkinakis DV'21 \\
- Emmie Berger DV'20 \\
- Alva Johansson DV'21 \\
- Moa Ahlberg DV'21 \\
- Natalie Stein DV'21
\subsubsection*{Förslag till beslut}
\begin{attsatser}
    \item välja Josefin Kokkinakis som ordförande för Femme++ under verksamhetsåret 2023.
\end{attsatser}
\subsubsection*{Beslut}
\begin{attsatser}
    \item attsatsen bifalles
\end{attsatser}

\subsection{Motion: Byta namn på Uppdrag till Studienämnd}
Studienämnden är en ny kommitté som arbetar med att säkerställa studie -och utbildningskvalitén på programmet. Under en senare stämma valdes det att kommittén skulle namnges Uppdrag, och vi skulle vilja byta detta till Studienämnd. Vi anser att det är viktigt att kommitténs namn representerar det kommittén jobbar med, och det gör inte namnet Uppdrag i nuläget. Detta leder till att studenter som bara ser Uppdrag inte förstår vad kommittén gör vid första blick,
utan att det behövs en förklaring. Utöver ovanstående skäl var vi inte heller närvarande när beslutet togs och blev heller inte tillfrågade.

- Josefin Kokkinakis DV'21
\subsubsection*{Förslag till beslut}
\begin{attsatser}
	\item byta namnet på Uppdrag till Datavetenskaps studienämnd.
\end{attsatser}
\subsubsection*{Beslut}
\begin{attsatser}
    \item attsatsen bifalles
\end{attsatser}

\subsection{Motion: starta ny kommitté DV\_Ops}
Vill starta en ny kommitté med syfte att främja starkare kunskap och intresse kring
datorer och open source mjukvara. Kommittén kommer hålla diverse arrangemang med
syfte att upplysa och lära DVs studenter om användbara verktyg och generell kunskap
om datorer. Syftet är även att väcka ett större intresse för IT samt främja användandet
och kunskapen kring öppen mjukvara och göra detta tillgängligt för alla.
Kommittén ska även ansvara för divisionens IT. Då menas bl.a dvet.se och exempelvis Monadens Wi-Fi.

- Tim “Båtsman” Persson DV'22
\subsubsection*{Förslag till beslut}
\begin{attsatser}
    \item i avsnittet om kommittéer i reglementet, eller som självständigt dokument i dokumentsamlingen, införa följande

    \begin{displayquote}
        \subsection*{DV\_Ops}
        \begin{enumerate}[label=\arabic* §]
            \item Kommitténs namn är DV\_Ops.

            \item DV\_Ops arbetar med att främja och göra IT intresset tillgängligt för alla samt agera IT-Avdelning för divisionen och dess lokaler.

            \item Mandatperioden för kommittémedlemmarna bestäms internt inom kommittén.

            \item Kommittémedlemmarna bestäms internt inom kommittén.
        \end{enumerate}
    \end{displayquote}
\end{attsatser}

\subsubsection*{Beslut}
\begin{attsatser}
    \item attsatsen bifalles
\end{attsatser}

\subsection{Motion: Inval av DV\_Ops}
Jag föreslår mig själv som ordförande för DV\_Ops.

- Tim “Båtsman” Persson DV'22

\subsubsection*{Förslag till beslut}
\begin{attsatser}
    \item välja Tim "Båtsman" Persson till ordförande för DV\_Ops under verksamhetsåret 2023.
\end{attsatser}
\subsubsection*{Beslut}
\begin{attsatser}
    \item attsatsen bifalles
\end{attsatser}

\subsection{Motion: förtydliga kommittéers uppgift}
Önskar ett förtydligande av kommittéers uppdrag utöver mottagningen.
Vill att det ska finnas tydliga riktlinjer kring vad de individuella
kommittéernas uppdrag är utöver mottagningen. Detta för att försäkra att kommittéer uppfyller
sitt syfte och att aktiviteter kommer igång på riktigt.
Tanken är att detta ska träda i kraft vid respektive kommittés nästa verksamhetsår.

- Tim “Båtsman” Persson DV'22

\subsubsection*{Förslag till beslut}
\begin{attsatser}
    \item i avsnittet om kommittéer i reglementet, eller som självständigt dokument i dokumentsamlingen, införa följande

     \begin{displayquote}
        \subsection*{Verksamhetsplan för kommittéer}
        \begin{enumerate}[label=\arabic* §]
            \item varje ordförande ska presentera en verksamhetsplan för styrelsen senast 1 månad efter inval.
        \end{enumerate}
    \end{displayquote}
\end{attsatser}
\subsubsection*{Beslut}
\begin{attsatser}
    \item attsatsen bifalles
\end{attsatser}

\subsection{Mötseordning: ny mötessekreterare}
Tyvärr så måste mötessekreterare Tekla Siesjö lämna mötet,
och därav behöver mötet en ny mötessekreterare.
Nils Karl Teodor Lyrevik ställer upp som mötessekreterare.

\subsubsection*{Förslag till beslut}
\begin{attsatser}
	\item Nils Karl Teodor Lyrevik väljs in som mötessekreterare
\end{attsatser}
\subsubsection*{Beslut}
\begin{attsatser}
    \item attsatsen bifalles
\end{attsatser}

\subsection{Motion: skapa ny kommitté Mega 7}
Dear DVD, I hope this motion finds you in good health and high spirits. I am
writing to propose the establishment of a new committee named Mega 7, realm dedicated of MONADEN.
to the crucial \emoji{sunflower}\emoji{cherry-blossom}\emoji{rose} task of watering PLANTS in the enchanting
MONADEN, although a place of undeniable beauty, currently thrives on the presence of PLASTIC PLANTS. 
While these artificial botanical wonders serve their
purpose, they lack the vitality and essence that real PLANTS bring to any environment. Our
objective, power of \emoji{leaf-fluttering-in-wind}
through PLANTS the \emoji{leaf-fluttering-in-wind}
formation .
of Mega 7, is to infuse MONADEN with the life-giving, refreshing
Mega 7 shall not be burdened with the responsibility of acquiring new PLANTS, for MONADEN
already boasts a mesmerizing collection of artificial flora. Instead, this committee will focus its
cared efforts for onwith the diligent the utmost watering love and and devotion.
nurturing \emoji{sweat-droplets}\emoji{green-heart}
of these PLASTIC PLANTS, ensuring that they are
It is crucial to recognize the profound significance of PLANTS in our lives. PLANTS are not
very merely air decorative we breathe, ornaments purifying but it and essential bestowing components upon usof the the gift natural of OXYGEN.
world. They \emoji{blowfish}
PLANTS contribute also to the play a pivotal role in maintaining BIODIVERSITY, serving as habitats for countless creatures and
fostering the delicate balance of ECOSYSTEMS.
Moreover, PLANTS possess an unparalleled ability to
soothe and invigorate our souls. Their vibrant colors,
delicate fragrances, and gentle rustling in the wind
offer solace, tranquility, and an escape from the hustle
and bustle of our daily lives. By creating Mega 7, we
affirm our dedication to preserving this essential
connection has predominantly with nature, embraced even in the an artificial.
environment \emoji{leaf-fluttering-in-wind}\emoji{sparkler}
that
Through the diligent care and attention lavished upon
MONADEN's PLASTIC PLANTS, Mega 7 will breathe life into the very heart of this remarkable
realm. We shall strive to create an oasis, where the harmony between artificial and organic elements
this is seamlessly magical place.
woven \emoji{seedling}\emoji{star-struck}
together, nourishing not only the PLANTS but also the souls of all who inhabit
In conclusion, I urge you to consider the creation of Mega 7, recognizing the vital role that
PLANTS play in our lives and embracing the opportunity to nurture and cherish them, even in a
PLANTS world dominated and ensure by plastic that MONADEN facsimiles. thrives Let us come with the together love and to celebrate care it truly the deserves.
remarkable \emoji{leaf-fluttering-in-wind}\emoji{rainbow}
power of
Thank you for your attention, and I eagerly await your favorable response.
Warm regards
\emoji{exploding-head}\emoji{smiling-face-with-hearts}William Bodin.

\subsubsection*{Förslag till beslut}
\begin{attsatser}
    \item i avsnittet med kommittéer i reglementet, eller som självständigt dokument i dokumentsamlingen, införa följande.

    \begin{displayquote}
        \subsection*{Mega 7}
        \begin{enumerate}[label=\arabic* §]
            \item Kommitténs namn är Mega7.

            \item Mega 7 vattnar Monadens blommor.

            \item Mandatperioden för kommittémedlemmarna bestäms internt inom kommittén.

            \item Kommittémedlemmarna bestäms internt inom kommittén.
        \end{enumerate}
    \end{displayquote}
\end{attsatser}
\subsubsection*{Beslut}
\begin{attsatser}
    \item attsatsen bifalles
\end{attsatser}

\subsection{Motion: Inval av Mega 7}
Dear [Committee/Board Name],
I hope this message finds you well. I am writing to follow up on the motion
for the creation of the Mega 7 Committee, proposed earlier. It is with great
this enthusiasm esteemed that committee.
I now put \emoji{party-popper}\emoji{sunflower}\emoji{star}
forth a motion to appoint Will as the President of
Will's passion for PLANTS and dedication to their well-being make him an ideal candidate for the
role. His extensive knowledge in horticulture, coupled with his unwavering commitment to
nurturing success.
\emoji{leaf-fluttering-in-wind}\emoji{hibiscus}\emoji{leaf-fluttering-in-wind}
the green treasures of our world, will undoubtedly propel Mega 7 to great heights of
Will outstanding leadership qualities, coupled with his ability to inspire and engage others, make
him the perfect candidate to guide the committee in its mission. His infectious enthusiasm and
harmonious genuine love MONADEN.
for PLANTS\emoji{seedling}\emoji{sparkler}\emoji{leaf-fluttering-in-wind}
will serve as a guiding light, illuminating the path towards a thriving and
In his role as President, Will will not only oversee the watering and care of the existing PLASTIC
PLANTS in MONADEN but also foster a culture of appreciation and reverence for the invaluable
role that PLANTS play in our lives. His exceptional organizational skills and unwavering
commitment excellence.
\emoji{sweat-droplets}\emoji{green-heart}\emoji{deciduous-tree}
will ensure that the committee operates smoothly and efficiently, always striving for
By appointing Will as the President of Mega 7, we embrace a leader who embodies the spirit of our
dedication, mission. His and presence a deep will appreciation invigorate for the the committee, remarkable infusing power of it with PLANTS.
an extra \emoji{leaf-fluttering-in-wind}\emoji{rainbow}\emoji{smiling-face-with-hearts}
dose of passion,
I kindly request your favorable consideration and support for this motion, as I firmly believe that
flourishing Will's leadership paradise will that elevate celebrates Mega and 7 to cherishes new heights the and wonders allow of us nature.
to make \emoji{leaf-fluttering-in-wind}\emoji{seedling}\emoji{hibiscus}
MONADEN a
Thank you for your attention, and I eagerly await your positive response.
Warm regards
\emoji{exploding-head}\emoji{smiling-face-with-hearts}, William Bodin
\subsubsection*{Förslag till beslut}
\begin{attsatser}
    \item välja William "Hilliam" Bodin till ordförande för Mega7 under verksamhetsåret 2023.
\end{attsatser}
\subsubsection*{Beslut}
Under personvalet blev det fyra som röstade för attsatsen och fyra som röstade mot. 
Enligt stadgan (6.5 §) ska då detta beslut väljas genom lotten, och detta gjordes genom att singla slant.
Det valdes att klave skulle representera att attsatsen skulle bifallas och krona för avslagning av attsatsen.
Myntet landade på krona.
\begin{attsatser}
    \item attsatsen avslås
\end{attsatser}

\subsection{Motion: uthyrning av skåp till divisionsmedlemmar}

Vi har idag skåp i Monaden, och vi vill få personer som medlemar i divisionen. 
Varför inte göra det kostnadsfritt för medlemmar, och låta icke-medlemmar betala?

- Albin "Slaget" Otterhäll DV'18

\subsubsection*{Förslag till beslut}
\begin{attsatser}
	\item ConCats tilldelas ansvaret för de mindre skåpen i Monaden
	\item divisionsmedlemmar får kostnadsfritt hyra tilldelat skåp (i mån av plats) i Monaden
	\item divisionsmedlemmar får förtur framför icke-medlemmar vid skåpstilldelning
	\item ConCats sätter ett uthyrningspris på skåpen för icke-medlemmar som gör det rationellt för icke-medlemmar att bli medlemmar
	\item Styrelsen till nästa stämma införskaffar eller lånar peruker som ska bäras av samtliga medlemmar i Styrelsen från stämmans öppnande till stämmand avslutande
\end{attsatser}
\subsubsection*{Ändringsyrkanden}
\begin{attsatser}
	\item ändra attsatser från 
	\begin{enumerate}
		\item[\textbf{1. Att}] ConCats tilldelas ansvaret för de mindre skåpen i Monaden
		\item[\textbf{2. Att}] divisionsmedlemmar får kostnadsfritt hyra tilldelat skåp (i mån av plats) i Monaden
		\item[\textbf{3. Att}] divisionsmedlemmar får förtur framför icke-medlemmar vid skåpstilldelning
		\item[\textbf{4. Att}] ConCats sätter ett uthyrningspris på skåpen för icke-medlemmar som gör det rationellt för icke-medlemmar att bli medlemmar
		\item[\textbf{5. Att}] Styrelsen till nästa stämma införskaffar eller lånar peruker som ska bäras av samtliga medlemmar i Styrelsen från stämmans öppnande till stämmand avslutande
	\end{enumerate} till
	\begin{enumerate}
		\item[\textbf{1. Att}] ConCats tilldelas ansvaret för de mindre skåpen i Monaden
		\item[\textbf{2. Att}] divisionsmedlemmar får kostnadsfritt hyra tilldelat skåp (i mån av plats) i Monaden
		\item[\textbf{3. Att}] divisionsmedlemmar får förtur framför icke-medlemmar vid skåpstilldelning
		\item[\textbf{4. Att}] ConCats sätter ett uthyrningspris på skåpen för icke-medlemmar som gör det rationellt för icke-medlemmar att bli medlemmar
		\item[\textbf{5. Att}] Styrelsen till nästa stämma införskaffar eller lånar peruker som ska bäras av samtliga medlemmar i Styrelsen från stämmans öppnande till stämmand avslutande
		\item[\textbf{6. Att}] införa krav på medlemskap för att få tillgång till ett skåp
	\end{enumerate}
\end{attsatser}

\subsubsection*{Beslut}
\begin{attsatser}
    \item attsatsen från ändringsyrkandet bifalles
	\item första attsatsen avslås
	\item andra attsatsen avslås
	\item tredje attsatsen bifalles
	\item fjärde attsatsen avslås
	\item femte attsatsen bifalles
	\item sjätte attsatsen bifalles
\end{attsatser}
\newpage

\section{Avslutande av möte}

Mötet beräknas avslutas 20.47

\stämmosignaturer

\end{document}
