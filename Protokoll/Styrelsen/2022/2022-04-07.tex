\documentclass[protokoll]{dvd}
\usepackage{graphicx}

\KOMAoptions{
    headwidth = 18cm,
    footwidth = 18cm,
}

\begin{document}

\title{Styrelsemöte 4}
\subtitle{2022}
\author{Styrelsen}
\date{2022-04-07}

\textbf{Datum:} \csname @date\endcsname\\
\textbf{Tid:} 11:00\\
\textbf{Plats:} Styrelserummet och Discord\\
\textbf{Styrelsemedlemmar:}
\begin{närvarande_förtroendevalda}
    \förtroendevald{Ordförande}{Samuel Hammersberg}{}
    \förtroendevald{Kassör}{Morgan Thowsen}{}
    \förtroendevald{Vice ordförande}{Vakant}{}
    \förtroendevald{Sekreterare}{Sebastian Selander}{}
    \förtroendevald{SAMO}{Tekla Siesjö}{}
\end{närvarande_förtroendevalda}

% \textbf{Övriga medlemmar:}

\section{Öppnande av möte}

Mötet beräknas öppnas av Samuel Hammersberg kl 11.00

\section{Runda bordet}

Runda bordet innebär att varje person berättar hur de känner sig.
Man kan till exempel berätta att man är stressad på grund av en inlämning, irriterad på sin granne, eller bara väldigt glad därför att man ligger i fas med plugget.

\newpage

\section{Formalia}

\subsection{Styrelsens beslutbarhet}

\blockquote[7 kap. 5 \S~första stycket i stadgan][]{%
    Styrelsen är endast beslutsmässig då samtliga styrelsemedlemmar har fått kallelsen till styrelsemötet och minst hälften av styrelsemedlemmarna är närvarande.
    Ordförande eller vice ordförande måste vara närvarande när beslut tas.
}

\subsubsection{Beslut}

\begin{attsatser}
    \item Styrelsen har uppnått kraven i 7 kap. 5 § första stycket i stadgan och är därmed beslutbar.
\end{attsatser}

\subsection{Fastställande av mötesschema}

För att styrelsen ska kunna fatta ett styrelsebeslut eller protokollföra en diskussion behöver punkten i mötesschemat där styrelsen ska fatta beslut vara inlagd eller föras in i mötesschemat senast vid den här punkten.

\subsubsection{Beslut}

\begin{attsatser}
    \item mötesschemat fastställs utan några förändringar.
\end{attsatser}

\subsection{Val av protokolljusterare}

Protokolljusterare har till uppgift att kontrollera att protokollet i slutändan reflekterar de faktiska besluten och diskussionerna som fördes under mötet.
Utöver protokolljusteraren så ska mötesordförande och mötessekreteraren signera protokollet.
Vid styrelsemöten ska det endast vara en justerare.
Mötesordförande och mötessekreteraren kan inte vara justerare.

\subsubsection{Beslut}
\begin{attsatser}
    \item Morgan Thowsen väljs till protokolljusterare
\end{attsatser}

\newpage

\section{Rapport}

\subsection{Styrelseövergripande}

\subsubsection{Bankkonto öppnat}
Bankkonto är öppnat och pengarna är överförda. Göta stänger ner sitt gamla konto.

\subsubsection{Ny ordförande till DVRK}
Vi har nu gjort ett utskick om att divisionen söker efter ny ordförande till DVRK då den tidigare avgått.

\subsubsection{Girls code club}
Girls Code Club är ett tillfälle för tjejer att lära sig koda under sommaren. Vi har blivit kontaktade av Ana Bove som önskar vår hjälp. Vi har ett möte bokat med henne under påsken för att se vad vi kan göra!

\subsection{Ordförande}

\subsubsection{Kontakta Gerdes om arbetsmarknadsinformation}
Samuel har bokat möte med Jonathan Klingberg m.m. angående arbetsmarknad.

\subsection{Kassör}

\subsubsection{Äskat för bankkonto}
Har äskat pengar för att öppna bankkontot (1000kr). Äskan har blivit godkänd

\subsubsection{Äskat för kläder till Mega6}
Har skickat in en äskan för kläder (2000kr) till Mega6. Äskan har blivit godkänd

\subsection{Sekreterare}

\subsubsection{Kontaktinformation}
Har börjat sammanställa dokument med kontaktinformation för framtida styrelser samt enklare att har allt samlat på ett ställe!

\subsection{SAMO}

\subsubsection{Elproblem}
Meddelat att vi har problem med statisk elektricitet. Vi har fått svar att de inte hittade någonting.

\subsubsection{Funktionsanpassning}
Har återigen skickat iväg mail om funktionsanpassning i Monaden till Roger Johansson.

\newpage

\section{Beslutsärenden}

\subsection{Rättighetsbytande för bankkonto}

\includegraphics[scale=0.5]{/home/sebastian/Documents/LaTeX/dvet/styrmote-2022-04-07/konto.png}

\subsection{Bokföringssystem}
Dags att besluta om ett bokföringssystem som DVArm kan använda!
Anledning till varför vi vill ha bokföringssystem är för att vi ska bli näringsdrivande.
Göta har erbjudit sig ta hand om fakturering av företag m.m, Göta kan sedan i sin tur "donera" pengarna till oss.

På grund av att vi inte är näringsdrivande tjänar vi inget till att skaffa ett bokföringssystem.

    \subsubsection*{Förslag till beslut}
    \begin{attsatser}
    \item \emph{Inga förslag} 
    \end{attsatser}

    \subsubsection*{Beslut}
    \begin{attsatser}
    \item vi inte skaffar något bokföringssystem
    \end{attsatser}

\newpage

\section{Diskussion}

\subsection{Skrivarkvot}
Bör vi ordna en satt skrivarkvot för sekreterare per år? Annars måste sekreteraren står för det själv.

\subsection{Orbi till mottagningen}
Göta har erbjudit att vi kan använda Orbi för vår mottagning om vi vill. Orbi kan vara ett bra alternativ då vi förhoppningsvis kan nå fler nya studenter under mottagningen.
Styrelsen ska inleda en diskussion med DVRK för att se vad de tycker.

\subsection{Ansökan om föreningsstatus}
Dags att ansöka om ny föreningsstatus. Ansökan öppnas 18 april. Det vi i styrelsen behöver göra är att ordna ett antal dokument och texter för Göta.

\newpage

\section{Avslutande av möte}

\subsection{Arkiverade punkter}
\begin{itemize}
    \item Domäner
    \item Bokföringssystem
    \item Äskan
\end{itemize}

\subsection{Mötesutvärdering}

\subsection{Nästa möte}
21 april 11.00 - 12.00

\subsection{Mötets avslutande}

Mötet avslutades 12.28

\styrelsesignaturer

\end{document}
