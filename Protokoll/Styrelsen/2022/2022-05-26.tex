\documentclass[protokoll]{dvd}

\KOMAoptions{
    headwidth = 18cm,
    footwidth = 18cm,
}

\begin{document}

\title{Styrelsemöte 8}
\subtitle{2022}
\author{Styrelsen}
\date{2022-05-26}

\textbf{Datum:} \csname @date\endcsname\\
\textbf{Tid:} 11:00\\
\textbf{Plats:} Styrelserummet och Discord\\
\textbf{Styrelsemedlemmar:}
\begin{närvarande_förtroendevalda}
    \förtroendevald{Ordförande}{Samuel Otterhäll}{Ja}
    \förtroendevald{Kassör}{Morgan Thowsen}{Ja}
    \förtroendevald{Vice ordförande}{Vakant}{}
    \förtroendevald{Sekreterare}{Sebastian Selander}{Ja}
    \förtroendevald{SAMO}{Tekla Siesjö}{Ja}
\end{närvarande_förtroendevalda}

% \textbf{Övriga medlemmar:}

\section{Öppnande av möte}

Mötet öppnades av Samuel Hammersberg kl 11.00

\section{Runda bordet}

Runda bordet innebär att varje person berättar hur de känner sig.
Man kan till exempel berätta att man är stressad på grund av en inlämning, irriterad på sin granne, eller bara väldigt glad därför att man ligger i fas med plugget.

\section{Formalia}

\subsection{Styrelsens beslutbarhet}

\blockquote[7 kap. 5 \S~första stycket i stadgan][]{%
    Styrelsen är endast beslutsmässig då samtliga styrelsemedlemmar har fått kallelsen till styrelsemötet och minst hälften av styrelsemedlemmarna är närvarande.
    Ordförande eller vice ordförande måste vara närvarande när beslut tas.
}

\subsubsection*{Beslut}

\begin{attsatser}
    \item Styrelsen har uppnått kraven i 7 kap. 5 § första stycket i stadgan och är därmed beslutbar.
\end{attsatser}

\subsection{Fastställande av mötesschema}

För att styrelsen ska kunna fatta ett styrelsebeslut eller protokollföra en diskussion behöver punkten i mötesschemat där styrelsen ska fatta beslut vara inlagd eller föras in i mötesschemat senast vid den här punkten.

\subsubsection*{Förslag}

\begin{attsatser}
    \item mötesschemat fastställs utan några förändringar.
\end{attsatser}

\subsection{Val av protokolljusterare}

Protokolljusterare har till uppgift att kontrollera att protokollet i slutändan reflekterar de faktiska besluten och diskussionerna som fördes under mötet.
Utöver protokolljusteraren så ska mötesordförande och mötessekreteraren signera protokollet.
Vid styrelsemöten ska det endast vara en justerare.
Mötesordförande och mötessekreteraren kan inte vara justerare.

\subsubsection*{Förslag}
\begin{attsatser}
    \item Morgan Thowsen väljs till protokolljusterare
\end{attsatser}

\section{Rapport}

\subsection{Styrelseövergripande}

    \subsubsection*{Divisionsstämma}
    Årets divisionsstämma har genomförts nyligen. Vi var 13 personer (inklusive styrelsen) av 30 medlemmar.
    Under mötet bestämde vi äntligen vad för officiell logga vi ska ha.
    Divisionsstämmorapporten kommer upp på divisionens drive inom kort.

\newpage

\subsection{Ordförande}

    \subsubsection*{Institutionsråd}
    Institutionsrådet har varit och det var inte direkt relevant för oss.
    Har även talat med PDave vad gäller git/latex/m.m-workshops.
    Han undrade om någon av oss var intresserade. 

    \subsubsection*{Studienämnd}
    Alex Gerdes har pratat med potentiella ordförande för UPPDRAG och det verkar som att intresset ökat igen.

\subsection{Kassör}

    \subsubsection*{DVRK}
    Alla fakturor som rör mottagningen ska gå via DVRK och deras kassör, inte styrelsens kassör

    \subsubsection*{Mottagningsmöte}
    Schemat för introduktionskurserna är satt. Det är fortfarande oklart hur äskning ska ske till institutionen.
    De förväntar sig fakturor men det är inte alltid säkert att det är något vi kan lösa.

    \subsubsection*{Retroaktiv äskan}
    En retroaktiv äskan har gjorts på allting som vi betalt under läsåret.
    Det verkar som att institutionen är positivt inställd till detta, vilket skulle innebära att vi inte har spenderat en enda krona under läsåret.

    \subsubsection*{Linode}
    Vi kommer flytta över vår hosting, inklusive discordbotten till Linode då de erbjuder gratis hosting.
    Det innebär att vi kommer behöva skicka en rapport på vad pengarna går åt. Vi anar att det inte kräver värst mycket arbete.

% \subsection{Sekreterare}

\subsection{SAMO}

    \subsubsection*{Verksamhetsrapport}
    En verksamhetsrapport ska skickas in till Göta över vad dom är gjort under året. Den arbetas på för tillfället.

% \section{Beslutspunkter}

\section{Diskussion}

    \subsection*{Girls Code Club}
    Vi har ännu inte hittat någon som är intresserad av att arrangera ett event för Girls Code Club.
    Vi kan göra ett utskick till men vi har inte fått någon ytterligare information vad gäller tid, datum eller pengar så det är svårt för oss att arbeta på det.


\section{Avslutande av möte}
Mötet avslutades kl. 12.08

% \subsection{Mötesutvärdering}

\subsection{Nästa möte}
Efter sommaren

\subsection{Mötets avslutande}

\styrelsesignaturer

\end{document}
