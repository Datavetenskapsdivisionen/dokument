\documentclass[protokoll]{dvd}

\KOMAoptions{
    headwidth = 18cm,
    footwidth = 18cm,
}

\begin{document}

\title{Styrelsemöte 3}
\subtitle{2021}
\author{Styrelsen}
\date{2021-09-21}

\textbf{Datum:} \csname @date\endcsname\\
\textbf{Tid:} 12:00\\
\textbf{Plats:} Styrelserummet\\
\textbf{Styrelsemedlemmar:}
\begin{närvarande_förtroendevalda}
    \förtroendevald{Ordförande}{Albin Otterhäll}{}
    \förtroendevald{Kassör}{Morgan Thowsen}{}
    \förtroendevald{Vice ordförande}{Samuel Hammersberg}{}
    \förtroendevald{Sekreterare}{Sebastian Selander}{}
    \förtroendevald{SAMO}{Tekla Siesjö}{}
\end{närvarande_förtroendevalda}
\textbf{Övriga medlemmar:}

\section{Öppnande av möte}

Mötet beräknas öppnas av Albin Otterhäll klockan 12:00.

\section{Runda bordet}

Runda bordet innebär att varje person berättar hur de känner sig.
Man kan till exempel berätta att man är stressad på grund av en inlämning, irriterad på sin granne, eller bara väldigt glad därför att man ligger i fas med plugget.

\section{Formalia}

\subsection{Styrelsens beslutbarhet}

\blockquote[7 kap. 5 \S~första stycket i stadgan][]{%
    Styrelsen är endast beslutsmässig då samtliga styrelsemedlemmar har fått kallelsen till styrelsemötet och minst hälften av styrelsemedlemmarna är närvarande.
    Ordförande eller vice ordförande måste vara närvarande när beslut tas.
}

Albin kollade vilken tid övriga medlemmar hade möjlighet att ha möte, för att sedan kalla till mötet.
Torsdagen den 16 september skrev Samuel tid och datum på Styrelsens Discordserver.

\subsubsection*{Förslag till beslut}

\begin{attsatser}
    \item Styrelsen har uppnått kraven i 7 kap. 5 § första stycket i stadgan och är därmed beslutbar.
\end{attsatser}

\subsection{Fastställande av mötesschema}

För att styrelsen ska kunna fatta ett styrelsebeslut eller protokollföra en diskussion behöver punkten i mötesschemat där styrelsen ska fatta beslut vara inlagd eller föras in i mötesschemat senast vid den här punkten.

\subsubsection*{Förslag till beslut}

\begin{attsatser}
    \item Mötesschemat fastställs utan några förändringar.
\end{attsatser}

\subsection{Val av mötessekreterare}

Sebastian blev invald i Styrlsen med tanknen att han ska bli divisionens sekreterare.
Om han blir vald kommer han vara mötessekreterare by default vid framtida möten.

\subsubsection*{Förslag till beslut}

\begin{attsatser}
    \item Sebastian Selander väljs till mötessekreterare.
\end{attsatser}

\subsection{Val av protokolljusterare}

Protokolljusterare har till uppgift att kontrollera att protokollet i slutändan reflekterar de faktiska besluten och diskussionerna som fördes under mötet.
Utöver protokolljusteraren så ska mötesordförande och mötessekreteraren signera protokollet.
Vid styrelsemöten ska det endast vara en justerare.
Mötesordförande och mötessekreteraren kan inte vara justerare.

\subsubsection*{Förslag till beslut}

\emph{Inga förslag}

\section{Rapporter}

\subsection{Styrelsen som helhet}

\subsubsection{Låsa in brädspel}

Det har kommit till styrelsens kännedom att flera personer har ``lånat hem'' brädspel från divisionens spelhylla, utan att informera om detta.
Det har även tidigare förekommit stölder av brädspel.
I dagsläget är endast TV-spelen inlåsta i föreningsskåpen i Monaden.

Vid styrelsemöte 2 be

Ingen har gått igenom brädspelen för att välja ut de stöldbegärliga spelen.
Ingen har åtagit sig att göra arbetet.

\subsubsection{Divisionsstämmans extramöte den 15 september}

Stämman genomförde sitt möte den 15 september enligt plan i Monaden.
Mötet gick bra.
Sebastian och Samuel lagade en väldigt god linsgryta som mötesmat.
Det blev mycket över.
Samtliga av Styrelsens propositioner röstades igenom, med vissa ändringar.
Sebastian, som var mötessekreterare, håller på att slutföra mötesprotokollet.

\subsection{Ordförande}

\subsubsection{Prata med DRust angående pant}

Albin ska prata med DRust om huruvida vi kan komma upp med en gemensam lösning på hur vi ska lösa pant från Monaden.
Städarna plockade bort den tidigare pantlådan någon gång 2020 därför att de hade fått order om det från någon arbetsmiljöenhet som klagade på lukten.
Albin misstänker att de stora svarta (överfyllda) pantpåsarna som förvarades i Monaden var den stora boven, men det inte spelar någon roll nu.
Vi har fått till oss från städarna att vi kan spara pant i Monaden om vi tömmer panten 1--2 gånger per vecka.
Det viktiga är att det inte luktar i Monaden.

\begin{description}[style=multiline, widest=00.00, align=left, leftmargin=2.5cm]
    \item[2021-08-25] Albin har skrivit till DRust och frågat om vi kan göra en deal med panten.

    \item[2021-09-20] Fick svar från Sjöcrona (ordförande för DRust'21) att de kan förvara och panta panten, men att de inte har en nyckel de kan ge ut till deras förråd.
    Albin informerade Sjöcrona att det handlar om ungefär en liten soppåse i veckan.
    Albin föreslåg att man kan sätta upp något form av schema där antingen vi lämnar panten i Basen, eller att DRust kommer och hämtar panten i Monaden.
    Sjöcrona ska diskutera det med hans fellow DRustare på deras onsdagsmöte.
\end{description}

\subsubsection{Större kök till Monaden}

Albin har åtagit sig att prata med Roger Johannsson, vice prefekt för grundutbildningen vid D\&IT, angående att Monaden behöver ett större kök.
Vice prefekten är den som är ytterst ansvarig för Monaden på universitetets sida.

\begin{description}[style=multiline, widest=00.00, align=left, leftmargin=2.5cm]
    \item[2021-09-21] Albin har inte haft möjlighet att prata med Roger angående Monadens kök.
\end{description}

\subsubsection{Dispositionsavtal}

\emph{Denna punkt är nerärvd från den tidigare Styrelsen.}

Roger Johannsson, vice prefekt för grundutbildningen vid D\&IT, har lagt det på verksamhetsstödet att ta fram ett dispositionsavtal av Monaden.
Dispositionsavtal sätter reglerna för hur Monaden får användas.
Samtliga sektionslokaler, både på GU och Chalmers, har ett dispositionsavtal.

\begin{description}[style=multiline, widest=00.00, align=left, leftmargin=2.5cm]
    \item[2021-09-20] Skrev mejl och frågade om en statusuppdatering.

    \item[2021-09-21] Fick svar från Roger om att en tjänsten som administrativchef håller på att tillsättas, och att han hoppas på att saker kommer börja rulla igång efter det.
\end{description}

% \subsection{Vice ordförande}

\subsection{Kassör}

\subsubsection{Bankkonto hos Swedbank}

Swebank bekräftade att de fått våran ansökan om föreningskonto hos dem den 16 augusti.
Ingen ny information sedan dess.
Beräknad behandlingstid är två månader.

\begin{description}[style=multiline, widest=00.00, align=left, leftmargin=2.5cm]
    \item[2021-08-16] Swedbank skickar en skriftlig bekräftelse via mejl att de har mottagit våran ansökan.

    \item[2021-09-21] Ingen uppdatering i ärendet.
\end{description}

\subsubsection{Köpa partylampor till Monaden}

Partylamporna är inköpta.

\subsubsection*{Förslag till beslut}

\begin{attsatser}
    \item klarskriva ärendet.
\end{attsatser}

\subsubsection{Köpa HDMI-sladd}

Morgan har köpt en HDMI-sladd.
Sladden som går mellan Raspberry Pi:n och TV:n är utbytt mot den nya sladden.
Den gamla sladden är nu inkopplad i datorskärmen.

\subsubsection*{Förslag till beslut}

\begin{attsatser}
    \item klarskriva ärendet.
\end{attsatser}

\subsubsection{Nytt skåpskontrakt}

Ett nytt skåpskontrakt är framtaget, och skåpen har börjat delas ut.
Runt fjorton pers har fått skåp.

\subsubsection*{Förslag till beslut}

\begin{attsatser}
    \item klarskriva ärendet.
\end{attsatser}

% \subsection{Sekreterare}

% \subsection{SAMO}

\section{Beslutsärenden}

Per capsulam beslut är beslut fattade av styrelsen utanför möten.
Dessa beslut behöver bekräftas på nästkommande styrelsemöte.

\subsection{Per Capsulam: Inköp av RGB-lampor från IKEA}

Vid Styrelsemöte 2 beslutade Styrelsen att det skulle införskaffas ``partylampor'' till Monaden.
Anledningen är att samtliga gamla lampor är utbrända.
Morgan tog fram fyra alternativ och postade en omröstning på Styrelsens Discordserver.
Alternativet med 11 RGB lampor; en fjärrkontroll; och en gateway från IKEA fick samtliga röster.

\subsubsection*{Förslag till beslut}

\begin{attsatser}
    \item bekräfta per capsulam beslutet att köpa in 11 RGB lampor; en fjärrkontroll; och en gateway från IKEA till Monaden.
\end{attsatser}

\subsection{Val av poster}

Samuel Hammersberg blev nominerad med tanken att han skulle bli vice ordförande.

Tekla Siesjö blev nominerad med tanken att han skulle bli SAMO.

Sebastian Selander blev nominerad med tanken att han skulle bli vice ordförande.

\subsubsection*{Förslag till beslut}

\begin{attsatser}
    \item Samuel Hammersberg väljs till vice ordförande.
    \item Tekla Siesjö väljs till SAMO.
    \item Sebastian Selander väljs till sekreterare.
\end{attsatser}

\section{Diskussioner}\label{sec:discussioner}

\subsection{Läsning av ekonomiska regler}

Divisionen har ekonomiska regler som dess medlemmar behöver följa.
Därför ska samtliga nya styrelseledamöter läsa igenom reglerna.

\subsection{Läsning av tidigare protokoll}

Beslut från tidigare möten under nuvarande verksamhetsår är relevanta för de nya styrelsemedlemmarna.

\subsection{Hur ska vi strukturera upp styrelsearbete?}

Hur ska Styrelsen strukturera upp sitt arbete?

\subsection{Kommittéer i framtiden}

Albin och Morgan har diskuterat hur de tänker vilka kommittéer som ska finnas, och hur de ska fungera.
Tankarna är att vi ska utgå utifrån de olika ``typer'' av människor vi har i föreningen, så att nästan alla alltid har en kommitté som skulle kunna intressera dem.
Idéen är att divisionen ska ha
\begin{itemize}
    \item en kommitté som ska agera rustmästeri och PR-kommitté(-ish) (programrådskommitté), rustmästerier har till uppgift att ta hand om sektionslokalerna, PR-kommittéer (programrådskommittéer) har ansvaret att ``bilda gemenskap på sektionerna'';

    \item en festkommitté som främst anordnar fester och liknande aktiviteter (eventuellt pubbar i framtiden);

    \item en recentiorskommitté som har till uppgift att koordinera Mottagningen; samt

    \item en arbetsmarkandskommitté som har till uppgift att samarbeta med företag.
\end{itemize}

Det är kotym på Chalmers att PR-kommittéerna anordnar pubbar, men den eventuella framtida uppgiften tänker vi att festkommittéen kan ta på sig.
Vi tänker att våran motsvarighet till PR-kommitté kommer fokusera mera på chill-aktiviteter.
Vi har inga konkreta förslag på kommitténamn (förutom på recentiorskommittéen), och anser att kommittéerna får bestämma själva över eventuella kommittékläder och profilering.

Kommittéordföranden bör gå i sitt andra år, eller senare för att undvika att krångel med nya inval.
Programmet har ganska stora avhopp efter LP 2 och under sommaren efter första året.
De nya studenterna kan då söka till kommittéerna som vanliga kommittéemedlemmar.
Då endast kommittéordförande väljs av stämman, kan endast kommittéordförande utföra inval av nya studenter.
Därför kan en eventuell vice ordförande inte ta över om en ordförande avgår.

\section{Avslutande av möte}

\subsection{Mötesutvärdering}

\subsection{Nästa möte}

\subsection{Mötets avslutande}

Mötet beräknas avslutas klockan 13:00.

\styrelsesignaturer

\end{document}
