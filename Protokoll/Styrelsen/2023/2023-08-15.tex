\documentclass[protokoll]{dvd}

\KOMAoptions{
    headwidth = 18cm,
    footwidth = 18cm,
}

\begin{document}

\title{Styrelsemöte 4}
\subtitle{2023}
\author{Styrelsen}
\date{2023-08-15}

\textbf{Datum:} \csname @date\endcsname\\
\textbf{Tid:} 15:17\\
\textbf{Plats:} Monaden\\
\textbf{Styrelsemedlemmar:}
\begin{närvarande_förtroendevalda}
    \förtroendevald{Ordförande}{Samuel Otterhäll}{Ja}
    \förtroendevald{Vice ordförande}{Vakant}{Nej}
    \förtroendevald{Kassör}{Lukas Gartman}{Ja}
    \förtroendevald{SAMO}{Tekla Siesjö}{Nej}
    \förtroendevald{Sekreterare}{Vakant}{Nej}
\end{närvarande_förtroendevalda}

\textbf{Övriga medlemmar:} \\
    \medlem{William Bodin}


\section{Öppnande av möte}

Mötet öppnades av Samuel Hammersberg kl 15.18

\section{Runda bordet}

Runda bordet innebär att varje person berättar hur de känner sig.
Man kan till exempel berätta att man är stressad på grund av en inlämning, irriterad på sin granne, eller bara väldigt glad därför att man ligger i fas med plugget.

\section{Formalia}

\subsection{Styrelsens beslutbarhet}

\blockquote[7 kap. 5 \S~första stycket i stadgan][]{%
    Styrelsen är endast beslutsmässig då samtliga styrelsemedlemmar har fått kallelsen till styrelsemötet och minst hälften av styrelsemedlemmarna är närvarande.
    Ordförande eller vice ordförande måste vara närvarande när beslut tas.
}

\subsubsection*{Förslag}

\begin{attsatser}
    \item Styrelsen har uppnått kraven i 7 kap. 5 § första stycket i stadgan och är därmed beslutbar.
\end{attsatser}
\subsubsection*{Beslut}
\begin{attsatser}
    \item förslaget till beslut bifalles
\end{attsatser}


\subsection{Fastställande av mötesschema}

För att styrelsen ska kunna fatta ett styrelsebeslut eller protokollföra en diskussion behöver punkten i mötesschemat där styrelsen ska fatta beslut vara inlagd eller föras in i mötesschemat senast vid den här punkten.

\subsubsection*{Förslag}

\begin{attsatser}
    \item mötesschemat fastställs utan några förändringar.
\end{attsatser}
\subsubsection*{Beslut}
\begin{attsatser}
    \item förslaget till beslut bifalles
\end{attsatser}

\subsection{Val av mötessekreterare}
Eftersom att styrelsen för nuvarande inte har en sekreterare tar Lukas Gartman på sig jobbet.
\subsubsection*{Förslag}
\begin{attsatser}
    \item Lukas Gartman väljs till mötessekreterare
\end{attsatser}
\subsubsection*{Beslut}
\begin{attsatser}
    \item förslaget till beslut bifalles
\end{attsatser}

\subsection{Val av protokolljusterare}

Protokolljusterare har till uppgift att kontrollera att protokollet i slutändan reflekterar de faktiska besluten och diskussionerna som fördes under mötet.
Utöver protokolljusteraren så ska mötesordförande och mötessekreteraren signera protokollet.
Vid styrelsemöten ska det endast vara en justerare.
Mötesordförande och mötessekreteraren kan inte vara justerare.

\subsubsection*{Förslag}
\begin{attsatser}
    \item William Bodin väljs till protokolljusterare
\end{attsatser}
\subsubsection*{Beslut}
\begin{attsatser}
    \item förslaget till beslut bifalles
\end{attsatser}

\section{Rapport}
\subsection{Ordförande}
Samuel Hammersberg har sammarbetat med Lukas Gartman för att försöka göra klart allt arbete gentemot Swedbank och bank access.
Han har även försökt få tag i maillistor för divisionen, men har ej fått respons från ansvariga.
Han har också arrangerat inköp av overraller för intresserade divisionsmedlemmar. 

\subsection{Kassör}
Lukas Gartman har sammarbetat med Samuel Hammersberg för att försöka göra klart allt arbete gentemot Swedbank och bank access.
Har har skött allmän ekonomi, och sett till att fakturor har blivit betalda med hjälp av tidigare kassör.
Han har även diskuterat med programansvarig Alex Gerdes om möjlighet för milersättning m.m för folk som arrangerar mottagningen.
Han har också haft möte med tidigare kassör och ekonomiansvarige för mottagningen för att klargöra hur bokföring och liknande sköts. 
Han har dessutom mottagit och hanterat äskningar gentemot Göta Studentkår, bland annat; 
1200kr för ett game jam, 1928,41kr för diverse studiemedel, 85,50kr för kaffe, 1499,87kr för mer studiemedel, 
650kr för dekoration för vår sektionslokal, 100kr för diskmedel och plastbestik, 375,15kr för en avslutningsmiddag och 871,43kr för en filmkväll.

\newpage

\section{Beslutspunkter}

\subsubsection*{Komplettering av beslut på stämma}
Under stämman den 24:e november 2022 valdes Lukas Gartman in som kassör för divisionen.
Det beslutades även då att föreningen representeras gentemot Swedbank av Samuel Hammersberg och Lukas Gartman med \textbf{två i förening}.
Detta var ett misstag och styrelsen vill korrigera detta, då enligt stadgan ska detta vara \textbf{var för sig}.
Styrelsen vill korrigera detta misstaget då Samuel Hammersberg och Lukas Gartman ska representera divisionen
\textbf{var för sig}.

\subsubsection*{Förslag till beslut:}
\begin{attsatser}
    \item Samuel Hammersberg och Lukas Gartman representerar divisionen \textbf{var för sig}
\end{attsatser}

\subsubsection*{Beslut}
\begin{attsatser}
    \item förslaget till beslut bifalles
\end{attsatser}

\section{Avslutande av möte}

\subsection{Nästa möte}
Styrelsen har preliminärt bokat in 31:a 10:00 för nästa möte.

\subsection{Mötets avslutande}
Mötet avslutades kl. 15.50

\styrelsesignaturer

\end{document}
