\documentclass[protokoll]{dvd}

\KOMAoptions{
    headwidth = 18cm,
    footwidth = 18cm,
}

\begin{document}

\title{Styrelsemöte 2}
\subtitle{2023}
\author{Styrelsen}
\date{2023-01-31}

\textbf{Datum:} \csname @date\endcsname\\
\textbf{Tid:} 14:15\\
\textbf{Plats:} Styrelserummet och Discord\\
\textbf{Styrelsemedlemmar:}
\begin{närvarande_förtroendevalda}
    \förtroendevald{Ordförande}{Samuel Otterhäll}{Ja}
    \förtroendevald{Vice ordförande}{Vakant}{Ja}
    \förtroendevald{Kassör}{Lukas Gartman}{Ja}
    \förtroendevald{SAMO}{Tekla Siesjö}{Ja}
    \förtroendevald{Sekreterare}{Vakant}{Ja}
\end{närvarande_förtroendevalda}

\section{Öppnande av möte}

Mötet öppnades av Samuel Hammersberg kl 14.15

\section{Runda bordet}

Runda bordet innebär att varje person berättar hur de känner sig.
Man kan till exempel berätta att man är stressad på grund av en inlämning, irriterad på sin granne, eller bara väldigt glad därför att man ligger i fas med plugget.

\section{Formalia}

\subsection{Styrelsens beslutbarhet}

\blockquote[7 kap. 5 \S~första stycket i stadgan][]{%
    Styrelsen är endast beslutsmässig då samtliga styrelsemedlemmar har fått kallelsen till styrelsemötet och minst hälften av styrelsemedlemmarna är närvarande.
    Ordförande eller vice ordförande måste vara närvarande när beslut tas.
}

\subsubsection*{Beslut}

\begin{attsatser}
    \item Styrelsen har uppnått kraven i 7 kap. 5 § första stycket i stadgan och är därmed beslutbar.
\end{attsatser}

\subsection{Fastställande av mötesschema}

För att styrelsen ska kunna fatta ett styrelsebeslut eller protokollföra en diskussion behöver punkten i mötesschemat där styrelsen ska fatta beslut vara inlagd eller föras in i mötesschemat senast vid den här punkten.

\subsubsection*{Förslag}

\begin{attsatser}
    \item mötesschemat fastställs utan några förändringar.
\end{attsatser}

\subsection{Val av protokolljusterare}

Protokolljusterare har till uppgift att kontrollera att protokollet i slutändan reflekterar de faktiska besluten och diskussionerna som fördes under mötet.
Utöver protokolljusteraren så ska mötesordförande och mötessekreteraren signera protokollet.
Vid styrelsemöten ska det endast vara en justerare.
Mötesordförande och mötessekreteraren kan inte vara justerare.

\subsubsection*{Förslag}
\begin{attsatser}
    \item Lukas Gartman väljs till protokolljusterare
\end{attsatser}

\section{Rapport}

% \subsection{Styrelseövergripande}

% \newpage

\subsection{Ordförande}
Det har gått framåt lite med loggan, inga färdiga resultat ännu.
Samuel ska ha ett möte med Karl gällande studienämnden, de har haft problem med tillgång till gamla studieenkäter 
Karl kände att instutionen skött det dåligt med överlämningen av resurser(kursenkäter/mötesprotokoll) 
till studienämnden, mötet kommer uppdaga mer om det här. 

\subsection{Kassör}
Samuel och Lukas har varit på banken och fixat bankdosor och skrivit in sig på kontot, 
Albin och Morgan ska föra över ägandeskapet till Samuel och Lukas.
Alla har börjat smått rapportera hur packat det är på lunchtid i Monaden, men det kan bli bättre.

% \subsection{Sekreterare}

\subsection{SAMO}
Mailat om rivning av väggen, jag ska kontakta CFAB
Mail har kommi tillbaka om dörröppnarknappen, det eskaleras inte till akademiska hus, de tycker att den fungerar som den ska men har bett oss kontakta igen om något går fel 
Ska gå igenom medlemslistan på nytt inför föreningsansökan, dubbelkolla att alla är medlemmar och lägga till nya från formuläret 

\section{Beslutspunkter}


\subsection{Stämman}
DVRK vill ha en ny stämma för att välja en ny ordförande, helst under andra halvan av februari,
i nuläget ser vi inte ytterligare punkter som borde tas upp på den.
Samuel fixar announcement om att ha stämma och att folk får komma in med punkter till stämman.

\subsection{Förslag}
\begin{attsatser}
  \item Nästa stämma hålls den 22a februari
\end{attsatser}

\section{Diskussion}

\subsection{Äskningar}
Två äskningar från concats, Lukas kontaktar Morgan och Karl om hur dessa ska hanteras gentemot Göta.

\subsection{Ekonomikontot}
Lukas kontaktar Morgan och Albin om överlämning

\section{Avslutande av möte}
Mötet avslutades kl. 15.16

% \subsection{Mötesutvärdering}

\subsection{Nästa möte}
Nästa möte hålls Tisdag den 7/2 kl 11.15

% \subsection{Mötets avslutande}

\styrelsesignaturer

\end{document}
