\documentclass[protokoll]{dvd}

\KOMAoptions{
    headwidth = 18cm,
    footwidth = 18cm,
}
\usepackage{graphicx}
\begin{document}

\title{Styrelsemöte 6}
\subtitle{2023}
\author{Styrelsen}
\date{2023-12-08}


\textbf{Datum:} \csname @date\endcsname\\
\textbf{Tid:} 12:13\\
\textbf{Plats:} Styrelserummet\\
\textbf{Styrelsemedlemmar:}
\begin{närvarande_förtroendevalda}
    \förtroendevald{Ordförande}{Samuel Hammersberg}{Ja}
    \förtroendevald{Vice ordförande}{Tim Persson}{Ja}
    \förtroendevald{Kassör}{Lukas Gartman}{Ja}
    \förtroendevald{SAMO}{Josefin Kokkinakis}{Ja, kom in 12:26}
    \förtroendevald{Sekreterare}{Gustav Dalemo}{Ja}
\end{närvarande_förtroendevalda}

%\textbf{Övriga medlemmar:} \\
%    \medlem{Ingen}

\section{Öppnande av möte}

Mötet öppnades av Samuel Hammersberg kl 12:13

\section{Runda bordet}

Runda bordet innebär att varje person berättar hur de känner sig.
Man kan till exempel berätta att man är stressad på grund av en inlämning, irriterad på sin granne, eller bara väldigt glad därför att man ligger i fas med plugget.

\section{Formalia}

\subsection{Styrelsens beslutbarhet}

\blockquote[7 kap. 5 \S~första stycket i stadgan][]{%
    Styrelsen är endast beslutsmässig då samtliga styrelsemedlemmar har fått kallelsen till styrelsemötet och minst hälften av styrelsemedlemmarna är närvarande.
    Ordförande eller vice ordförande måste vara närvarande när beslut tas.
}

\subsubsection*{Förslag}

\begin{attsatser}
    \item Styrelsen har uppnått kraven i 7 kap. 5 § första stycket i stadgan och är därmed beslutbar.
\end{attsatser}
\subsubsection*{Beslut}
\begin{attsatser}
    \item förslaget till beslut bifalles
\end{attsatser}


\subsection{Fastställande av mötesschema}

För att styrelsen ska kunna fatta ett styrelsebeslut eller protokollföra en diskussion behöver punkten i mötesschemat där styrelsen ska fatta beslut vara inlagd eller föras in i mötesschemat senast vid den här punkten.

\subsubsection*{Förslag}

\begin{attsatser}
    \item mötesschemat fastställs utan några förändringar.
\end{attsatser}
\subsubsection*{Beslut}
\begin{attsatser}
    \item förslaget till beslut bifalles
\end{attsatser}

\subsection{Val av mötessekreterare}
Gustav Dalemo tar sig an uppdraget
\subsubsection*{Förslag}
\begin{attsatser}
    \item Gustav Dalemo väljs till mötessekreterare
\end{attsatser}
\subsubsection*{Beslut}
\begin{attsatser}
    \item förslaget till beslut bifalles
\end{attsatser}

\subsection{Val av protokolljusterare}

Protokolljusterare har till uppgift att kontrollera att protokollet i slutändan reflekterar de faktiska besluten och diskussionerna som fördes under mötet.
Utöver protokolljusteraren så ska mötesordförande och mötessekreteraren signera protokollet.
Vid styrelsemöten ska det endast vara en justerare.
Mötesordförande och mötessekreteraren kan inte vara justerare.

\subsubsection*{Förslag}
\begin{attsatser}
    \item Tim Persson väljs till protokolljusterare
\end{attsatser}
\subsubsection*{Beslut}
\begin{attsatser}
    \item förslaget till beslut bifalles
\end{attsatser}

\section{Rapport}
\subsection{Ordförande}
Samuel ska skriva ihop en enkät till alla elever om vad dem tycker om utbildningen. Skicka ut den till alla kandidat, ADS,  CS elever.

\subsection{Kassör}
Lukas är inte helt färdig med bokföringen. Ska be om en crashcourse i bokföring av Götas ekonomiansvarig.
Eskning: Femme++ en tacomiddag, men gick tyvärr inte pga Götas budget.
Eskning: Datatjej, Gerdes har ännu inte gett svar.
Oscar Rei ska se om pengar över från insparken kan få användas och inte brinna inne. Vi vill använda pengarna till allmän studentnytta.

\subsection{(Blivande) Vice-ordförande}
Tim har ännu inget att rapportera.
Har en diskussionspunkt att ta upp senare.

\subsection{(Blivande) SAMO}
Josefin tänkte ha Femme++ träffen ändå, trots att eskningen inte gick igenom.
Kandidatprojektsansökan har strulat. Vi har helt andra kriterier än chalmers.

\subsection{(Blivande) Mötessekreterare}
Har haft lite ideer på förbättringar kring protokolltemplates. Postade i discord-kanalen.

\newpage

\section{Beslutspunkter}

\subsubsection*{Logga på marken}
Vi har fått tillåtelse att måla vår logga på marken vid trapporna. Oklart om det fortfarande gäller. Vi bör få det förnyat.
Kanske skicka ett mail igen och kolla om det fortfarande är okej?
\subsubsection*{Förslag till beslut:}
\begin{attsatser}
    \item Samuel Hammersberg fortsätter arbetet.
\end{attsatser}

\subsubsection*{Beslut}
\begin{attsatser}
    \item Attsatsen bifalles
\end{attsatser}


\subsubsection*{Julfika}
Det har ju blivit lite tradition. Mysigt! Lucia låter rimligt.
\subsubsection*{Förslag till beslut:}
\begin{attsatser}
    \item Vi håller det 2023-12-13 preliminärt kl 15-17.
\end{attsatser}

\subsubsection*{Beslut}
\begin{attsatser}
    \item Attsatsen bifalles
\end{attsatser}

\subsubsection*{Tystnadsplikt och hantering av incidenter}
I nuläget när det har hänt incidenter har Samuel tagit upp det med SAMO och eventuellt senare tagits upp med resten av styrelsen men det finns inga regler kring hur vi ska hantera det. 
Kommer det en student med något så ska det inte skrivas i discordchatten och heller inte diskuteras utanför styrelsen.
\subsubsection*{Förslag till beslut:}
\begin{attsatser}
    \item Josefin skriver ihop ett dokument med ett förslag på hur vi ska hantera incidenter till nästa möte.
    \item Josefin föreslår att medverka på ett övningspass på kursen där incedenter eventuellt förekommer.
    \item Personen som en utsatt har kontaktat bör vara den som forsätter kommunikationen med den personen.
\end{attsatser}

\subsubsection*{Beslut}
\begin{attsatser}
    \item Attsatserna bifalles
\end{attsatser}

\subsubsection*{Delegering av roller}
På senaste stämman gjorde endast fyllnads val till styerlsemedlemskap av Tim,Josefin och Gustav men rollerna ska sättas nu.
\subsubsection*{Förslag till beslut:}
\begin{attsatser}
    \item Josefin Kokkinakis väljs som SAMO
    \item Tim Persson väljs till vice-ordförande
    \item Gustav Dalemo väljs som sekreterare.
\end{attsatser}

\subsubsection*{Beslut}
\begin{attsatser}
    \item Attsatserna bifalles
\end{attsatser}


\section{Diskussionspunkter}

\subsubsection*{Vilka är styrelsen}
Skulle vara bra i framtiden att göra det tydligt med vilka vi i styrelsen är och vad vi gör.

\subsubsection*{Redovisning av insparken}
DVRK har inte ännu gjort så mycket men kommer fixa ihop ett dokument snart.
Se till att alla som ska närvara på mötet kommer på mötet, och att dokumentet är formellt skrivet.


\newpage
\section{Avslutande av möte}

\subsection{Nästa möte}
Inget förslag.

\subsection{Mötets avslutande}
Mötet avslutades kl. 13:13

\styrelsesignaturer

\end{document}
