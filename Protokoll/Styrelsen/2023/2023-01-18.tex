\documentclass[protokoll]{dvd}

\KOMAoptions{
    headwidth = 18cm,
    footwidth = 18cm,
}

\begin{document}

\title{Styrelsemöte 1}
\subtitle{2023}
\author{Styrelsen}
\date{2023-01-18}

\textbf{Datum:} \csname @date\endcsname\\
\textbf{Tid:} 11:45\\
\textbf{Plats:} Styrelserummet och Discord\\
\textbf{Styrelsemedlemmar:}
\begin{närvarande_förtroendevalda}
    \förtroendevald{Ordförande}{Samuel Otterhäll}{Ja}
    \förtroendevald{Vice ordförande}{Vakant}{Ja}
    \förtroendevald{Kassör}{Lukas Gartman}{Ja}
    \förtroendevald{SAMO}{Tekla Siesjö}{Ja}
    \förtroendevald{Sekreterare}{Vakant}{Ja}
\end{närvarande_förtroendevalda}

\textbf{Övriga medlemmar:} \\
    \medlem{Sebastian Selander}
    \medlem{Morgan Thowsen}


\section{Öppnande av möte}

Mötet öppnades av Samuel Hammersberg kl 12.03

\section{Runda bordet}

Runda bordet innebär att varje person berättar hur de känner sig.
Man kan till exempel berätta att man är stressad på grund av en inlämning, irriterad på sin granne, eller bara väldigt glad därför att man ligger i fas med plugget.

\section{Formalia}

\subsection{Styrelsens beslutbarhet}

\blockquote[7 kap. 5 \S~första stycket i stadgan][]{%
    Styrelsen är endast beslutsmässig då samtliga styrelsemedlemmar har fått kallelsen till styrelsemötet och minst hälften av styrelsemedlemmarna är närvarande.
    Ordförande eller vice ordförande måste vara närvarande när beslut tas.
}

\subsubsection*{Beslut}

\begin{attsatser}
    \item Styrelsen har uppnått kraven i 7 kap. 5 § första stycket i stadgan och är därmed beslutbar.
\end{attsatser}

\subsection{Fastställande av mötesschema}

För att styrelsen ska kunna fatta ett styrelsebeslut eller protokollföra en diskussion behöver punkten i mötesschemat där styrelsen ska fatta beslut vara inlagd eller föras in i mötesschemat senast vid den här punkten.

\subsubsection*{Förslag}

\begin{attsatser}
    \item mötesschemat fastställs utan några förändringar.
\end{attsatser}

\subsection{Val av protokolljusterare}

Protokolljusterare har till uppgift att kontrollera att protokollet i slutändan reflekterar de faktiska besluten och diskussionerna som fördes under mötet.
Utöver protokolljusteraren så ska mötesordförande och mötessekreteraren signera protokollet.
Vid styrelsemöten ska det endast vara en justerare.
Mötesordförande och mötessekreteraren kan inte vara justerare.

\subsubsection*{Förslag}
\begin{attsatser}
    \item Tekla Siesjö väljs till protokolljusterare
\end{attsatser}

\section{Rapport}

\subsection{Styrelseövergripande}

\subsubsection*{DVRK mottagningsrapport}
    DVRK har skickat in en rapport över hur mottagningen. Då är det bara att vänta och vad
    institutionen tycker.

\newpage

% \subsection{Ordförande}

% \subsection{Kassör}

% \subsection{Sekreterare}

\subsection{SAMO}
    \subsubsection*{Dörrknapp}
    Dörrknappen är anmäld trasig. Knappen arbetar som om det vore sina sista dagar, och det hade
    varit trevligt om det inte var så.

\section{Beslutspunkter}

\section{Diskussion}

\subsubsection*{Sekreterare och Kassör}
Vi diskuterar lite över vad sekreterare och kassör gör då dessa roller byts ut.

\section{Avslutande av möte}
Mötet avslutades kl. 12.45

\subsection{Mötesutvärdering}

\subsection{Nästa möte}

\subsection{Mötets avslutande}

\styrelsesignaturer

\end{document}
