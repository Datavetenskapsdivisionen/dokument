% !TeX program = lualatex
\documentclass[protokoll]{dvd}

\KOMAoptions{
    headwidth = 18cm,
    footwidth = 18cm,
}
\usepackage{graphicx}
\begin{document}

\title{Styrelsemöte 8?}
\subtitle{2024}
\author{Styrelsen}
\date{2024-09-23}


\textbf{Datum:} \csname @date\endcsname\\
\textbf{Tid:} 16:31\\
\textbf{Plats:} Styrelserummet\\
\textbf{Styrelsemedlemmar:}
\begin{närvarande_förtroendevalda}
    \förtroendevald{Ordförande}{Samuel Hammersberg}{Ja}
    \förtroendevald{Vice ordförande}{Tim Persson}{Ja}
    \förtroendevald{Kassör}{Lukas Gartman}{Ja}
    \förtroendevald{SAMO}{Josefin Kokkinakis}{Ja}
    \förtroendevald{Sekreterare}{Gustav Dalemo}{Nej}
\end{närvarande_förtroendevalda}

%\textbf{Övriga medlemmar:}
%\medlem{Ingen}
\section{Öppnande av möte}

Mötet öppnades av Samuel Hammersberg kl 16:31

\section{Runda bordet}

Runda bordet innebär att varje person berättar hur de känner sig.
Man kan till exempel berätta att man är stressad på grund av en inlämning, irriterad på sin granne, eller bara väldigt glad därför att man ligger i fas med plugget.

\section{Formalia}

\subsection{Styrelsens beslutbarhet}

\blockquote[7 kap. 5 \S~första stycket i stadgan][]{%
    Styrelsen är endast beslutsmässig då samtliga styrelsemedlemmar har fått kallelsen till styrelsemötet och minst hälften av styrelsemedlemmarna är närvarande.
    Ordförande eller vice ordförande måste vara närvarande när beslut tas.
}

\subsubsection*{Förslag}

\begin{attsatser}
    \item Styrelsen har uppnått kraven i 7 kap. 5 § första stycket i stadgan och är därmed beslutbar.
\end{attsatser}
\subsubsection*{Beslut}
\begin{attsatser}
    \item förslaget till beslut bifalles
\end{attsatser}


\subsection{Fastställande av mötesschema}

För att styrelsen ska kunna fatta ett styrelsebeslut eller protokollföra en diskussion behöver punkten i mötesschemat där styrelsen ska fatta beslut vara inlagd eller föras in i mötesschemat senast vid den här punkten.

\subsubsection*{Förslag}

\begin{attsatser}
    \item mötesschemat fastställs utan några förändringar.
\end{attsatser}
\subsubsection*{Beslut}
\begin{attsatser}
    \item förslaget till beslut bifalles
\end{attsatser}

\subsection{Val av mötessekreterare}
Tim Persson tar sig an uppdraget
\subsubsection*{Förslag}
\begin{attsatser}
    \item Tim Persson väljs till mötessekreterare
\end{attsatser}
\subsubsection*{Beslut}
\begin{attsatser}
    \item förslaget till beslut bifalles
\end{attsatser}

\subsection{Val av protokolljusterare}

Protokolljusterare har till uppgift att kontrollera att protokollet i slutändan reflekterar de faktiska besluten och diskussionerna som fördes under mötet.
Utöver protokolljusteraren så ska mötesordförande och mötessekreteraren signera protokollet.
Vid styrelsemöten ska det endast vara en justerare.
Mötesordförande och mötessekreteraren kan inte vara justerare.

\subsubsection*{Förslag}
\begin{attsatser}
    \item Josefin Kokkinakis väljs till protokolljusterare
\end{attsatser}
\subsubsection*{Beslut}
\begin{attsatser}
    \item förslaget till beslut bifalles
\end{attsatser}

\section{Rapport} 
\subsection{Ordförande}
Ordförande har fört dialog med PA angående värvning av studenterepresentanter till programråd. 

Ordförande har även haft kontakt med IT-Sek styrelsen för framtida samarbeten. Ex hackathons via DV\_Ops.  

\subsection{Kassör}
Kassör har arbetat nära med DVRKs kassör under insparken.

\subsection{Vice-ordförande}
Vice-ordförande kommitte möte. 
Vi beslutade gensamt asp-info möte den 2 Okt 17:17 i Monaden.
Vi utvärderade även insparken och är mycket nöjda.

Mötet diskuterade även möjligheten av att Vice-Ordförande ska tillvidare agera som PR för föreningen.

\subsection{SAMO}
SAMO har haft en väldigt \textit{hands on} roll under insparken. 

\subsection{Mötessekreterare}

Ej närvarande.


\newpage


\section{Diskussionspunkter}

\subsection*{Hemligstämplade protokoll}

I syfte av att skydda identiteter vid incidentshantering ser styrelsen syfte att göra någon förändring i hur dessa incidenter protokollförs.

Vi föreslår en separat pärm för rapport för incidenter och tystnadsplikt för nuvarande och kommande styrelsemedlemmar.

\subsection*{Styrelsemerch}

Vi diskuterade möjligheten för styrelsen att skaffa merch.

\subsection*{Måla märket i backen}
Utred möjligheten att prata ihop oss med SGMatte och SGFysik angående att måla tillsamans.

Vi väljer att inte fortsätta arbetet med att måla märket för närvarande och bordslägger punkten. 
\newpage

\section{Beslutspunkter}
\subsection*{Inför tystnadsplikt för Styrelsen}
Styrelsen ska vid nästa stämma presentera en proposition för att införa tystnadsplikt för styrelsemedlemmar.

\subsection{Förslag}
\begin{attsatser}
  \item Josefin presenterar en proposition vid nästa stämma för att införa tystnadsplikt för styrelsemedlemmar.
\end{attsatser}
\subsection{Beslut}
Attsats 1 bifalles.

\subsection*{Äskan, glas till Monaden}
\textit{"Hejsan käraste styrelse.
Vi i ConCats skulle vilja äska pengar för glas till monaden i form av Kårkällarens gamla glas. Vi fick ett preliminärt innoficiellt pris på 30kr per glas efter DVRK x Hum6 kårkällarkväll. Ungefär 15-20 glas (eller ungeför en back?) av den större varianten är vad vi tänker oss.
puss och kram
Minaj"}


Styrelsen anser att det varken finns behov eller plats för nya glas till Monaden.
\subsection{Förslag}
\begin{attsatser}
    \item Äskan godkänns.
\end{attsatser}
\subsection{Beslut}
Attsats 1 avslås.
\subsection*{}

\subsection*{Nästa stämma}

Nästa stämma ska tar plats 17:17 den 15 Oktober

\subsection{Förslag}
\begin{attsatser}
    \item Nästa stämma tar plats 17:17 den 15 Oktober.
\end{attsatser}
\subsection{Beslut}
Attsats 1 bifalles.
\subsection*{}

\subsection*{Ovvekrokar i styrelse}
Just nu hänger majoriteten av föreningens medlemmars overaller bara i hallen i Monaden!
I och med att nästan alla overall-bärande medlemmar har access till styrelserummet hade det även passat att hänga upp overallerna där.


\subsection{Förslag}
\begin{attsatser}
    \item Att Lucas tar reda på hur mycket detta kostar och presenterar det på nästa styrelsemöte.
\end{attsatser}
\subsection{Beslut}
Attsats 1 bifalles.

\subsection*{Lathund till motion}

Som ny student kan det vara svårt att veta hur man skickar in en motion till divisionsstämman.
Det hade varit bra om en lathund fanns på föreningens wiki, och kanske med exempel på hur olika motioner kan se ut, om man till
exempel vill starta en kommittée eller att styrelsen ska plantera nytt gräss på innegården.

\subsection{Förslag}
\begin{attsatser}
    \item Att Samuel skriver ihop en guide och lägger upp på wikin tills nästa styrelsemöte.
\end{attsatser}
\subsection{Beslut}
Attsats 1 bifalles.

\subsection*{Enkät för program satisfaction och insparks utvärdering}
Bra att ta temperaturen på hur våra studenter upplever programmet och hur insparken upplevdes.

\subsection{Förslag}
\begin{attsatser}
    \item Att Tim skapar en enkät för att att få reda på hur våra studenter upplever utbildningen och insparken inför näst-nästa styrelsemöte.
\end{attsatser}
\subsection{Beslut}
Attsats 1 bifalles.

\subsection*{Ringklocka för dörren till Monaden}
Ibland är det svårt för folk att märka att folk behöver komma in i Monaden, och då hade kanske en ringklocka passat!
\subsection{Förslag}
\begin{attsatser}
    \item Lucas undersöker en lösning till nästkommande styrelsemöte.
    \item Tim kontaktar Akademiska hus angående installation till nästkommande styrelsemöte.
\end{attsatser}
\subsection{Beslut}
Attsats 1 och 2 bifalles.

\subsection*{Styrelse aspning}
Styrelsen behöver nytt folk med nya idéer då vi alla börjar bli gamla o senila. Förslår att vi i styrelsen håller någon form av aspning för de folket som är intresserade av styrelsen.

\subsection{Förslag}
\begin{attsatser}
    \item Styrelsen deltar på det gemensamma asp-info mötet 2 Oktober.
    \item Styrelsen arrangerar minst ett asp-arr.
\end{attsatser}
\subsection{Beslut}
Attsats 1 och 2 bifalles.


\subsection*{Föreningens närvaro online.}
Datavetenskapsdivisionen existerar inte online utanför dvet.se. Utåt är även igenkänningen av Datavetenskapsdivisionen. Detta vill vi ändra på.
\subsection{Förslag}
\begin{attsatser}
    \item Tim upprättar närvaro på Instagram. 
\end{attsatser}
\subsection{Beslut}
Attsats 1 bifalles.

\subsection*{Representanter för programråd}
Alex vill ha någon/några för N1COS som enbart är studenter, och Nils hade velat ha någon mer N2COS student (helst en internationell student då).
Mötet lär vara i början av LP2.
\subsection{Förslag}
\begin{attsatser}
    \item Samuel tar på sig arbetet till nästkommande styrelsemöte.
\end{attsatser}
\subsection{Beslut}
Attsats 1 bifalles.

\subsection*{MatNatSex lån}
Vi har fått mail från MatNatSex.
\textit{Hej, vi i MatNat letar efter lokal till en av våra sittningar som ska hållas i november och fick höra att monaden kanske skulle kunna vara ett alternativ. Skulle det vara möjligt att få låna lokalen och går det i så fall att hålla interna sittningar där eller måste det vara externa?
Med vänlig hälsning
Carolinn "Quercus" Bergstad, Källarmästare}

Det är endast lämpligt för utlån där Datavetenskapsdivisionens medlemmar får närvara.
\subsection{Förslag}
\begin{attsatser}
    \item Tim svarar på mailet. 
\end{attsatser}
\subsection{Beslut}
Attsats 1 bifalles.

\subsection*{Access till återvinningsrummet}
Vi ska få 5st personer som har access till återvinningsrummet.

\subsection{Förslag}
\begin{attsatser}
    \item Kristoffer Gustavsson ska få tillgång återvinningsrummet.
    \item Ida Vranvuk ska få tillgång återvinningsrummet.
    \item Nikhil Olsson Mukhopadhyay ska få tillgång återvinningsrummet.
    \item Tim Persson ska få tillgång återvinningsrummet.
    \item Lukas Gartman ska få tillgång återvinningsrummet.
\end{attsatser}
\subsection{Beslut}
Attsats 1 bifalles.

\subsection*{Rivning av väggen}
Styrelsen drev tidigare en undersökning om att försöka riva väggen mellan lilla rummet och stora rummet i Monaden.
Idag har inte mycket arbete skett primärt på grund av byrokrati på instutionen och livet.

\subsection{Förslag}
\begin{attsatser}
    \item Att bordslägga punkten till nästkommande styrelsemöte.
\end{attsatser}
\subsection{Beslut}
Attsats 1 bifalles.

\subsection*{Äskan, ConCats Wii kontroller}
\textit{Hejsan käraste styrelse
Vi I ConCats skulle vilja eska för fler wii controller. Specifikt två nya kontroller tillsammans med två nunchucks.
För tillfället har vi endast två stycken viket gör att ungängen som involverar Wiispel, exempelvis ConCats spelkvällar blir svåra. Vi har löst detta genom att ConCats medlemmar tar med sig controllers som lånas ut under kvällen.
Vi har sett ett stort intresse för just Wii spel och anser därför att fler wii controller är en sund investering i studentlivet och det sociala umgänget innom DVD.
Vi ber därför om 1200kr för inköp av 2 kontroller samt 2 nunchucks vilket är nödvändigt för många spel. Vi Hoppas att inte hela summan kommer att krävas då denna siffra valdes med lite marginal i åtanke då Wii kontroller inte längre produceras.
Vi hoppas ni delar ConCats vision om en glatt umgänge där alla (4 pers) kan spela mario kart tillsammans efter en hård dag av pluggade!}

Styrelsen anser att detta är en orimlig summa för inköpet.
\subsection{Förslag}
\begin{attsatser}
    \item Äskan godkänns.
\end{attsatser}
\subsection{Beslut}
Attsats 1 avslås.


\subsection*{Kandidater för nästa möte}
Styrelsen behöver nya medlemmar. 

\subsection{Förslag}
\begin{attsatser}
    \item Samtliga styrelsemedlemmar tar fram ett förslag tills nästkommande möte som presenteras internt.
\end{attsatser}
\subsection{Beslut}
Attsats 1 bifalles.
\newpage
\section{Avslutande av möte}

\subsection{Nästa möte}
2024-10-01

\subsection{Mötets avslutande}
Mötet avslutades kl 19:54 

\styrelsesignaturer

\end{document}
