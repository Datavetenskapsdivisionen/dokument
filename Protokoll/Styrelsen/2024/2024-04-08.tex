% !TeX program = lualatex
\documentclass[protokoll]{dvd}

\KOMAoptions{
    headwidth = 18cm,
    footwidth = 18cm,
}
\usepackage{graphicx}
\begin{document}

\title{Styrelsemöte 6}
\subtitle{2024}
\author{Styrelsen}
\date{2024-04-08}


\textbf{Datum:} \csname @date\endcsname\\
\textbf{Tid:} 12.10\\
\textbf{Plats:} Styrelserummet\\
\textbf{Styrelsemedlemmar:}
\begin{närvarande_förtroendevalda}
\förtroendevald{Ordförande}{Samuel Hammersberg}{Ja}
\förtroendevald{Vice ordförande}{Tim Persson}{Ja}
\förtroendevald{Kassör}{Lukas Gartman}{Ja}
\förtroendevald{SAMO}{Josefin Kokkinakis}{Nej}
\förtroendevald{Sekreterare}{Gustav Dalemo}{Ja}
\end{närvarande_förtroendevalda}

%\textbf{Övriga medlemmar:}
%\medlem{Ingen}
\section{Öppnande av möte}

Mötet öppnades av Samuel Hammersberg kl 12.10

\section{Runda bordet}

Runda bordet innebär att varje person berättar hur de känner sig.
Man kan till exempel berätta att man är stressad på grund av en inlämning, irriterad på sin granne, eller bara väldigt glad därför att man ligger i fas med plugget.

\section{Formalia}

\subsection{Styrelsens beslutbarhet}

\blockquote[7 kap. 5 \S~första stycket i stadgan][]{%
    Styrelsen är endast beslutsmässig då samtliga styrelsemedlemmar har fått kallelsen till styrelsemötet och minst hälften av styrelsemedlemmarna är närvarande.
    Ordförande eller vice ordförande måste vara närvarande när beslut tas.
}

\subsubsection*{Förslag}

\begin{attsatser}
    \item Styrelsen har uppnått kraven i 7 kap. 5 § första stycket i stadgan och är därmed beslutbar.
\end{attsatser}
\subsubsection*{Beslut}
\begin{attsatser}
    \item förslaget till beslut bifalles
\end{attsatser}


\subsection{Fastställande av mötesschema}

För att styrelsen ska kunna fatta ett styrelsebeslut eller protokollföra en diskussion behöver punkten i mötesschemat där styrelsen ska fatta beslut vara inlagd eller föras in i mötesschemat senast vid den här punkten.

\subsubsection*{Förslag}

\begin{attsatser}
    \item mötesschemat fastställs utan några förändringar.
\end{attsatser}
\subsubsection*{Beslut}
\begin{attsatser}
    \item förslaget till beslut bifalles
\end{attsatser}

\subsection{Val av mötessekreterare}
Gustav Dalemo tar sig an uppdraget
\subsubsection*{Förslag}
\begin{attsatser}
    \item Gustav Dalemo väljs till mötessekreterare
\end{attsatser}
\subsubsection*{Beslut}
\begin{attsatser}
    \item förslaget till beslut bifalles
\end{attsatser}

\subsection{Val av protokolljusterare}

Protokolljusterare har till uppgift att kontrollera att protokollet i slutändan reflekterar de faktiska besluten och diskussionerna som fördes under mötet.
Utöver protokolljusteraren så ska mötesordförande och mötessekreteraren signera protokollet.
Vid styrelsemöten ska det endast vara en justerare.
Mötesordförande och mötessekreteraren kan inte vara justerare.

\subsubsection*{Förslag}
\begin{attsatser}
    \item Tim Persson väljs till protokolljusterare
\end{attsatser}
\subsubsection*{Beslut}
\begin{attsatser}
    \item förslaget till beslut bifalles
\end{attsatser}

\section{Rapport}
\subsection{Ordförande}
Har varit slö på att fixa trycket till overallen, men
ska se till att det sker under sommaren.
Har även varit i diskussion med IT-sektionens ordförande
om att vi ska vara med på att hålla i ett event under
veckan innan Valborg. Det lär förmodligen bli ett krök
i Himlabacken.

\subsection{Kassör}
Redovisning för revisor gick bra, och har fått godkänt äskan för kostnaden av Swedbank.
Har även fått in en äskning från Femme++.

\subsection{Vice-ordförande}
Inget att rapportera.

\subsection{SAMO}
Ej närvarande.

\subsection{Mötessekreterare}
Inget att rapportera.

\newpage
\section{Beslutspunkter}

\subsubsection*{Femme++ middag}
Femme++ önskar att hålla i en middag och Josefin Kokkinakis har
äskat 2005kr för detta.
\subsubsection*{Förslag till beslut:}
\begin{attsatser}
    \item godkänna äskan på 2005kr.
\end{attsatser}

\subsubsection*{Beslut}
\begin{attsatser}
    \item Attsatsen bifalles
\end{attsatser}

\newpage
\section{Avslutande av möte}

\subsection{Nästa möte}
2024-05-01

\subsection{Mötets avslutande}
Mötet avslutades kl. 13.00

\styrelsesignaturer

\end{document}
