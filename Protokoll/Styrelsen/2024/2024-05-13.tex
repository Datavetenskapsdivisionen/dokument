% !TeX program = lualatex
\documentclass[protokoll]{dvd}

\KOMAoptions{
    headwidth = 18cm,
    footwidth = 18cm,
}
\usepackage{graphicx}
\begin{document}

\title{Styrelsemöte 7}
\subtitle{2024}
\author{Styrelsen}
\date{2024-05-13}


\textbf{Datum:} \csname @date\endcsname\\
\textbf{Tid:} 11.58\\
\textbf{Plats:} Styrelserummet\\
\textbf{Styrelsemedlemmar:}
\begin{närvarande_förtroendevalda}
\förtroendevald{Ordförande}{Samuel Hammersberg}{Ja}
\förtroendevald{Vice ordförande}{Tim Persson}{Ja}
\förtroendevald{Kassör}{Lukas Gartman}{Ja}
\förtroendevald{SAMO}{Josefin Kokkinakis}{Ja}
\förtroendevald{Sekreterare}{Gustav Dalemo}{Ja}
\end{närvarande_förtroendevalda}

%\textbf{Övriga medlemmar:}
%\medlem{Ingen}
\section{Öppnande av möte}

Mötet öppnades av Samuel Hammersberg kl 11.58

\section{Runda bordet}

Runda bordet innebär att varje person berättar hur de känner sig.
Man kan till exempel berätta att man är stressad på grund av en inlämning, irriterad på sin granne, eller bara väldigt glad därför att man ligger i fas med plugget.

\section{Formalia}

\subsection{Styrelsens beslutbarhet}

\blockquote[7 kap. 5 \S~första stycket i stadgan][]{%
    Styrelsen är endast beslutsmässig då samtliga styrelsemedlemmar har fått kallelsen till styrelsemötet och minst hälften av styrelsemedlemmarna är närvarande.
    Ordförande eller vice ordförande måste vara närvarande när beslut tas.
}

\subsubsection*{Förslag}

\begin{attsatser}
    \item Styrelsen har uppnått kraven i 7 kap. 5 § första stycket i stadgan och är därmed beslutbar.
\end{attsatser}
\subsubsection*{Beslut}
\begin{attsatser}
    \item förslaget till beslut bifalles
\end{attsatser}


\subsection{Fastställande av mötesschema}

För att styrelsen ska kunna fatta ett styrelsebeslut eller protokollföra en diskussion behöver punkten i mötesschemat där styrelsen ska fatta beslut vara inlagd eller föras in i mötesschemat senast vid den här punkten.

\subsubsection*{Förslag}

\begin{attsatser}
    \item mötesschemat fastställs utan några förändringar.
\end{attsatser}
\subsubsection*{Beslut}
\begin{attsatser}
    \item förslaget till beslut bifalles
\end{attsatser}

\subsection{Val av mötessekreterare}
Gustav Dalemo tar sig an uppdraget
\subsubsection*{Förslag}
\begin{attsatser}
    \item Gustav Dalemo väljs till mötessekreterare
\end{attsatser}
\subsubsection*{Beslut}
\begin{attsatser}
    \item förslaget till beslut bifalles
\end{attsatser}

\subsection{Val av protokolljusterare}

Protokolljusterare har till uppgift att kontrollera att protokollet i slutändan reflekterar de faktiska besluten och diskussionerna som fördes under mötet.
Utöver protokolljusteraren så ska mötesordförande och mötessekreteraren signera protokollet.
Vid styrelsemöten ska det endast vara en justerare.
Mötesordförande och mötessekreteraren kan inte vara justerare.

\subsubsection*{Förslag}
\begin{attsatser}
    \item Tim Persson väljs till protokolljusterare
\end{attsatser}
\subsubsection*{Beslut}
\begin{attsatser}
    \item förslaget till beslut bifalles
\end{attsatser}

\section{Rapport}
\subsection{Ordförande}
Inget riktigt att raportera, då de möten om campusflytten har blivit inställda, och
inga andra möten gentemot instutionen eller skola har skett.

\subsection{Kassör}
Har fått in ett par äskningar som behövs beslutas om på mötet.

\subsection{Vice-ordförande}
Inget att rapportera.

\subsection{SAMO}
Inget att rapportera.

\subsection{Mötessekreterare}
Inget att rapportera.

\newpage


\section{Diskussionspunkter}

\subsection*{Avslutningskalas}
Styrelsen vill planera och arrangera ett avslutningskalas för studenterna på
Datavetenskap, Computer Science och Applied Data Science.
Styrelsen planerar på att införskaffa mat för grillspet och glass till efterrätt.
Styrelsen siktar på att circa 40 personen kommer.

\section{Beslutspunkter}

\subsubsection*{ConCats studiefika}
ConCats planerar att arrangera en pluggstuga och vill bjuda på fika då.
Daniell Cole har äskat för 300kr.
\subsubsection*{Förslag till beslut:}
\begin{attsatser}
    \item godkänna äskan på 300kr.
\end{attsatser}

\subsubsection*{Beslut}
\begin{attsatser}
    \item Attsatsen bifalles
\end{attsatser}

\subsubsection*{Fixa grillen}
Tim Persson vill ta på sig att rusta upp grillen, och har därav äskat 235kr
för att köpa material för att fixa grillen.
\subsubsection*{Förslag till beslut:}
\begin{attsatser}
    \item godkänna äskan på 235kr.
\end{attsatser}

\subsubsection*{Festivalbord}
Tim Persson har föreslaggit att köpa in festivalbord som kan användas för
aktiviter utomhus för 3990kr.
\subsubsection*{Förslag till beslut:}
\begin{attsatser}
    \item godkänna äskan på 3990kr.
\end{attsatser}

\subsubsection*{Beslut}
\begin{attsatser}
    \item Attsatsen bifalles
\end{attsatser}

\subsubsection*{Avslutningskalas}
Som tidigare nämnt vill styrelsen arrangera ett avslutningskalas och planerar
att äska instutionen 4000kr för detta.
\subsubsection*{Förslag till beslut:}
\begin{attsatser}
    \item godkänna äskan till instutionen för 4000kr.
\end{attsatser}

\subsubsection*{Beslut}
\begin{attsatser}
    \item Attsatsen bifalles
\end{attsatser}

\subsubsection*{Monaden material}
Under verksamhetsåret 2023/2024 så har föreningen fått en äsknings budget av Göta Studentkår
på circa 1000kr. Lukas Gartman har äskat för att använda dessa pengar för att köpa
in kaffe, te och diskmedel.
\subsubsection*{Förslag till beslut:}
\begin{attsatser}
    \item godkänna äskan till Göta Studenkår för 1100kr.
\end{attsatser}

\subsubsection*{Beslut}
\begin{attsatser}
    \item Attsatsen bifalles
\end{attsatser}

\newpage
\section{Avslutande av möte}

\subsection{Nästa möte}
2024-08-09

\subsection{Mötets avslutande}
Mötet avslutades kl. 13.16

\styrelsesignaturer

\end{document}
