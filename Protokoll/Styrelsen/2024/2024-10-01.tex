% !TeX program = lualatex
\documentclass[protokoll]{dvd}

\KOMAoptions{
    headwidth = 18cm,
    footwidth = 18cm,
}
\usepackage{graphicx}
\begin{document}

\title{Styrelsemöte 9}
\subtitle{2024}
\author{Styrelsen}
\date{2024-10-01}


\textbf{Datum:} \csname @date\endcsname\\
\textbf{Tid:} 11:13\\
\textbf{Plats:} Styrelserummet\\
\textbf{Styrelsemedlemmar:}
\begin{närvarande_förtroendevalda}
    \förtroendevald{Ordförande}{Samuel Hammersberg}{Ja}
    \förtroendevald{Vice ordförande}{Tim Persson}{Ja}
    \förtroendevald{Kassör}{Lukas Gartman}{Ja}
    \förtroendevald{SAMO}{Josefin Kokkinakis}{Ja}
    \förtroendevald{Sekreterare}{Gustav Dalemo}{Nej}
\end{närvarande_förtroendevalda}

%\textbf{Övriga medlemmar:}
%\medlem{Ingen}
\section{Öppnande av möte}

Mötet öppnades av Samuel Hammersberg kl 11:13

\section{Runda bordet}

Runda bordet innebär att varje person berättar hur de känner sig.
Man kan till exempel berätta att man är stressad på grund av en inlämning, irriterad på sin granne, eller bara väldigt glad därför att man ligger i fas med plugget.

\section{Formalia}

\subsection{Styrelsens beslutbarhet}

\blockquote[7 kap. 5 \S~första stycket i stadgan][]{%
    Styrelsen är endast beslutsmässig då samtliga styrelsemedlemmar har fått kallelsen till styrelsemötet och minst hälften av styrelsemedlemmarna är närvarande.
    Ordförande eller vice ordförande måste vara närvarande när beslut tas.
}

\subsubsection*{Förslag}

\begin{attsatser}
    \item Styrelsen har uppnått kraven i 7 kap. 5 § första stycket i stadgan och är därmed beslutbar.
\end{attsatser}
\subsubsection*{Beslut}
\begin{attsatser}
    \item förslaget till beslut bifalles
\end{attsatser}


\subsection{Fastställande av mötesschema}

För att styrelsen ska kunna fatta ett styrelsebeslut eller protokollföra en diskussion behöver punkten i mötesschemat där styrelsen ska fatta beslut vara inlagd eller föras in i mötesschemat senast vid den här punkten.

\subsubsection*{Förslag}

\begin{attsatser}
    \item mötesschemat fastställs utan några förändringar.
\end{attsatser}
\subsubsection*{Beslut}
\begin{attsatser}
    \item förslaget till beslut bifalles
\end{attsatser}

\subsection{Val av mötessekreterare}
Tim Persson tar sig an uppdraget
\subsubsection*{Förslag}
\begin{attsatser}
    \item Tim Persson väljs till mötessekreterare
\end{attsatser}
\subsubsection*{Beslut}
\begin{attsatser}
    \item förslaget till beslut bifalles
\end{attsatser}

\subsection{Val av protokolljusterare}

Protokolljusterare har till uppgift att kontrollera att protokollet i slutändan reflekterar de faktiska besluten och diskussionerna som fördes under mötet.
Utöver protokolljusteraren så ska mötesordförande och mötessekreteraren signera protokollet.
Vid styrelsemöten ska det endast vara en justerare.
Mötesordförande och mötessekreteraren kan inte vara justerare.

\subsubsection*{Förslag}
\begin{attsatser}
    \item Josefin Kokkinakis väljs till protokolljusterare
\end{attsatser}
\subsubsection*{Beslut}
\begin{attsatser}
    \item förslaget till beslut bifalles
\end{attsatser}

\section{Rapport} 
\subsection{Ordförande}

Ordförande skapade mall för motioner. Fört vidare arbete med studentrepresentanter. I skrivande stund saknas en representant från N1COS. Skall gå på programråd nästa vecka.

\subsection{Kassör}
Kassör har begärt offert hos teknologtryck för styrelsemerch.
Uttrett vad för ringklocka vi kan ha. Tittat efter krokar till styrelserummet.

\subsection{Vice-ordförande}
Upprättat närvaro på instagram. Arbetat med studentenkät. 
Koordinerat med kommittér för aspning. 

\subsection{SAMO}
Skrivit propositioner för tystnadsplikts avtal. Fört över incidenthanteringen till LaTeX.


\subsection{Mötessekreterare}

Ej närvarande.


\newpage


\section{Diskussionspunkter}
\subsection*{Styrelesaspning}
Styrelsen föreslår att vi preliminärt har styrelsen bokat torsdag den 7e November för ett asp-arr. Vi planerar en middag med asparna på Ölstugan tullen.

\subsection*{Styrelsemerch}
Offert begärd av Teknologtryck. Bordslägger till nästa möte.
Vi vill även ha tygmärken. Detta utreder vi också.

\subsection*{ConCats: Matförsäljning vid arr}
Styrelsen utreder möjligheter och svarar på ConCats mail. 

\section{Besultspunkter}

\subsection{Krokpunkten}
Styrelsen sätter ihop en inköpslista och äskar till instutionen.
\begin{attsatser}
    \item Styrelsen sätter ihop inköpslista och presenterar på nästa möte.
\end{attsatser}
\subsubsection{Beslut}
\begin{attsatser}
    \item Attsats 1 godkänns.
\end{attsatser}

\subsection{Ringklocka}
Styrelsen sätter ihop en inköpslista och äskar till instutionen.
\begin{attsatser}
    \item Styrelsen sätter ihop inköpslista och presenterar på nästa möte.
\end{attsatser}
\subsubsection{Beslut} 
\begin{attsatser}     
\item Attsats 1 bifalles. 
\end{attsatser}

\subsection{Stämmapunkter, modernisera Mega7}
Styrelsen tycker att det är dags att avveckla Mega7 eftersom de EJ uppfyllt uppdrag enligt reglementet.
\begin{attsatser}
    \item Styrelsen skriver proposition.
\end{attsatser}
\subsubsection{Beslut} 
\begin{attsatser}     
\item Attsats 1 bifalles. 
\end{attsatser}

\subsection{Stämmapunkter, tystnadsplikt}
Styrelsen vill införa tystnadsplikt för hantering av person-incidenter för att bevara integritet.
\begin{attsatser}
    \item Josefin skriver proposition.
\end{attsatser}
\subsubsection{Beslut} 
\begin{attsatser}
\item Attsats 1 bifalles. 
\end{attsatser}

\subsection{KG: Nya däck x 4 + 100}
Vagnen som används för inköp är trasig. Krävs pengar för att reparera denna. 
\begin{attsatser}
    \item Äskan godkänns.
\end{attsatser}
\subsubsection{Beslut} 
\begin{attsatser}     
\item Attsats 1 bifalles. 
\end{attsatser}

\subsection{ConCats: Wii äskning}
Concats äskar 700kr för att köpa in Wii kontroller. 
\begin{attsatser}
    \item Äskan godkänns.
\end{attsatser}
\subsubsection{Beslut} 
\begin{attsatser}     
\item Attsats 1 bifalles. 
\end{attsatser}

\subsection{ConCats: motion}

\begin{attsatser}
    \item Godtycklig styrelsemedlem besvarar motionen inför stämman.
\end{attsatser}
\subsubsection{Beslut} 
\begin{attsatser}     
\item Attsats 1 godkänns.
\end{attsatser}

\subsection{ConCats: motion}

Hatt till styrelsen!
Vi anser att ordförande för styrelsen, för att framföra ett godtagbart utseende, bör bära propellerhatt. Storlek på propellern bör reflektera innehavarens ego, magnituden, och därmed propellerstorleken, bör bestämmas av resterande styrelsemedlemmar. Vi anser även att det är lämpligt att denne ersätter sin peruk med denna hatt, om och endast om peruk och hatt inte kan bäras samtidigt. 

- ConCats
- Nikhil "Minaj"
- Daniell "Casilda"
- Martin "Marabou"

Attsatser:
    * Att ändra regelverket sådant att ordföranden för styrelsen ska iklädas en propellerhatt i kombination eller istället för den nuvarande peruken.
    * Att denna hatt överförs till nästa ordförande under aspningen för att visa på överförandet av styrandet.


\begin{attsatser}
    \item Samuel besvarar motionen inför stämman.
\end{attsatser}
\subsubsection{Beslut} 
\begin{attsatser}     
\item Attsats 1 bifalles. 
\end{attsatser}

\newpage
\section{Avslutande av möte}

\subsection{Nästa möte}
2024-10-10 - 17:00

\subsection{Mötets avslutande}
Mötet avslutades kl 12:47 

\styrelsesignaturer

\end{document}