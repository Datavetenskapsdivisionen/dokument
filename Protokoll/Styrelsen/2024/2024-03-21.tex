% !TeX program = lualatex
\documentclass[protokoll]{dvd}

\KOMAoptions{
    headwidth = 18cm,
    footwidth = 18cm,
}
\usepackage{graphicx}
\begin{document}

\title{Styrelsemöte 5}
\subtitle{2024}
\author{Styrelsen}
\date{2024-03-21}


\textbf{Datum:} \csname @date\endcsname\\
\textbf{Tid:} 12:06\\
\textbf{Plats:} Styrelserummet\\
\textbf{Styrelsemedlemmar:}
\begin{närvarande_förtroendevalda}
    \förtroendevald{Ordförande}{Samuel Hammersberg}{Ja}
    \förtroendevald{Vice ordförande}{Tim Persson}{Ja}
    \förtroendevald{Kassör}{Lukas Gartman}{Ja}
    \förtroendevald{SAMO}{Josefin Kokkinakis}{Ja, lämnade 13:00}
    \förtroendevald{Sekreterare}{Gustav Dalemo}{Ja}
\end{närvarande_förtroendevalda}

%\textbf{Övriga medlemmar:}
%\medlem{Ingen}
\section{Öppnande av möte}

Mötet öppnades av Samuel Hammersberg kl 12:06

\section{Runda bordet}

Runda bordet innebär att varje person berättar hur de känner sig.
Man kan till exempel berätta att man är stressad på grund av en inlämning, irriterad på sin granne, eller bara väldigt glad därför att man ligger i fas med plugget.

\section{Formalia}

\subsection{Styrelsens beslutbarhet}

\blockquote[7 kap. 5 \S~första stycket i stadgan][]{%
    Styrelsen är endast beslutsmässig då samtliga styrelsemedlemmar har fått kallelsen till styrelsemötet och minst hälften av styrelsemedlemmarna är närvarande.
    Ordförande eller vice ordförande måste vara närvarande när beslut tas.
}

\subsubsection*{Förslag}

\begin{attsatser}
    \item Styrelsen har uppnått kraven i 7 kap. 5 § första stycket i stadgan och är därmed beslutbar.
\end{attsatser}
\subsubsection*{Beslut}
\begin{attsatser}
    \item förslaget till beslut bifalles
\end{attsatser}


\subsection{Fastställande av mötesschema}

För att styrelsen ska kunna fatta ett styrelsebeslut eller protokollföra en diskussion behöver punkten i mötesschemat där styrelsen ska fatta beslut vara inlagd eller föras in i mötesschemat senast vid den här punkten.

\subsubsection*{Förslag}

\begin{attsatser}
    \item mötesschemat fastställs utan några förändringar.
\end{attsatser}
\subsubsection*{Beslut}
\begin{attsatser}
    \item förslaget till beslut bifalles
\end{attsatser}

\subsection{Val av mötessekreterare}
Gustav Dalemo tar sig an uppdraget
\subsubsection*{Förslag}
\begin{attsatser}
    \item Gustav Dalemo väljs till mötessekreterare
\end{attsatser}
\subsubsection*{Beslut}
\begin{attsatser}
    \item förslaget till beslut bifalles
\end{attsatser}

\subsection{Val av protokolljusterare}

Protokolljusterare har till uppgift att kontrollera att protokollet i slutändan reflekterar de faktiska besluten och diskussionerna som fördes under mötet.
Utöver protokolljusteraren så ska mötesordförande och mötessekreteraren signera protokollet.
Vid styrelsemöten ska det endast vara en justerare.
Mötesordförande och mötessekreteraren kan inte vara justerare.

\subsubsection*{Förslag}
\begin{attsatser}
    \item Josefin Kokkinakis väljs till protokolljusterare
\end{attsatser}
\subsubsection*{Beslut}
\begin{attsatser}
    \item förslaget till beslut bifalles
\end{attsatser}

\section{Rapport} 
\subsection{Ordförande}
Har inte fått svar om festanmälan. Så länge är det bara att fortsätta festanmäla som vanligt.
SAMO kommer bli inbjuden till instutitionsråd.

\subsection{Kassör}
Vi har fått in pengarna för datatjej. Väntar på svar om Swedbank årskostnad.
Ska förbereda en redovisning för vår revisor.

\subsection{Vice-ordförande}
Började arbetet med att måla loggan. Har endast fått två svar men vill ha fler svar. Lyfter detta på nästa kommittemöte. 
Har börjat att ansluta oss till Orbi-pay.

\subsection{SAMO}
Kursutvärderingsfika på tisdag!

\subsection{Mötessekreterare}
Inget att rapportera.

\newpage


\section{Diskussionspunkter}

\subsection*{Loggan}
Vi vill ta upp detta som en diskussionspunkt på nästa stämma. Tim formulerar en punkt om detta.
Tim kontaktar studentkåren.

\subsection*{Monadenenkät}
Inte mycket arbete har gjorts sen senaste. Vi tar upp detta på stämman och Gustav formulerar en punkt om detta.

\subsection*{Ordförande}
Vice-ordförande är just nu inte helt övertygad om han klarar att ta sig an uppdraget.
Skulle det finnas en annan kandidat kan vice hjälpa till med överlämnandet.
En valberedning diskuterades. Vi tar upp detta som en punkt på stämman.
Tim och Samuel tar gemensamt ansvar för detta.

\subsection*{Stämma}
Vi har fått in tre motioner, samtliga om byte av ordförande.
Vi behöver rapporter från samtliga kommiteer, har i skrivande stund endast fått in från DVRK.
Gustav blir mötessekretare (förmodligen).
Incidenthantering ska in i reglementet, Samuel förbereder detta.
Varje styrelsemedlem ska ha en rapport.


\newpage
\section{Avslutande av möte}

\subsection{Nästa möte}
2024-03-25

\subsection{Mötets avslutande}
Mötet avslutades kl. 13.16

\styrelsesignaturer

\end{document}
