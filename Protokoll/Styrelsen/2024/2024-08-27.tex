% !TeX program = lualatex
\documentclass[protokoll]{dvd}

\KOMAoptions{
    headwidth = 18cm,
    footwidth = 18cm,
}
\usepackage{graphicx}
\begin{document}

\title{Styrelsemöte 7}
\subtitle{2024}
\author{Styrelsen}
\date{2024-08-27}


\textbf{Datum:} \csname @date\endcsname\\
\textbf{Tid:} 15:00
\textbf{Plats:} Styrelserummet\\
\textbf{Styrelsemedlemmar:}
\begin{närvarande_förtroendevalda}
    \förtroendevald{Ordförande}{Samuel Hammersberg}{Ja}
    \förtroendevald{Vice ordförande}{Tim Persson}{Ja}
    \förtroendevald{Kassör}{Lukas Gartman}{Ja}
    \förtroendevald{SAMO}{Josefin Kokkinakis}{Ja}
    \förtroendevald{Sekreterare}{Gustav Dalemo}{Ja}
\end{närvarande_förtroendevalda}

%\textbf{Övriga medlemmar:}
%\medlem{Ingen}
\section{Öppnande av möte}

Mötet öppnades av Samuel Hammersberg kl 15:05

\section{Runda bordet}

Runda bordet innebär att varje person berättar hur de känner sig.
Man kan till exempel berätta att man är stressad på grund av en inlämning, irriterad på sin granne, eller bara väldigt glad därför att man ligger i fas med plugget.

\section{Formalia}

\subsection{Styrelsens beslutbarhet}

\blockquote[7 kap. 5 \S~första stycket i stadgan][]{%
    Styrelsen är endast beslutsmässig då samtliga styrelsemedlemmar har fått kallelsen till styrelsemötet och minst hälften av styrelsemedlemmarna är närvarande.
    Ordförande eller vice ordförande måste vara närvarande när beslut tas.
}

\subsubsection*{Förslag}

\begin{attsatser}
    \item Styrelsen har uppnått kraven i 7 kap. 5 § första stycket i stadgan och är därmed beslutbar.
\end{attsatser}
\subsubsection*{Beslut}
\begin{attsatser}
    \item förslaget till beslut bifalles
\end{attsatser}


\subsection{Fastställande av mötesschema}

För att styrelsen ska kunna fatta ett styrelsebeslut eller protokollföra en diskussion behöver punkten i mötesschemat där styrelsen ska fatta beslut vara inlagd eller föras in i mötesschemat senast vid den här punkten.

\subsubsection*{Förslag}

\begin{attsatser}
    \item mötesschemat fastställs utan några förändringar.
\end{attsatser}
\subsubsection*{Beslut}
\begin{attsatser}
    \item förslaget till beslut bifalles
\end{attsatser}

\subsection{Val av mötessekreterare}
Gustav Dalemo tar sig an uppdraget
\subsubsection*{Förslag}
\begin{attsatser}
    \item Gustav Dalemo väljs till mötessekreterare
\end{attsatser}
\subsubsection*{Beslut}
\begin{attsatser}
    \item förslaget till beslut bifalles
\end{attsatser}

\subsection{Val av protokolljusterare}

Protokolljusterare har till uppgift att kontrollera att protokollet i slutändan reflekterar de faktiska besluten och diskussionerna som fördes under mötet.
Utöver protokolljusteraren så ska mötesordförande och mötessekreteraren signera protokollet.
Vid styrelsemöten ska det endast vara en justerare.
Mötesordförande och mötessekreteraren kan inte vara justerare.

\subsubsection*{Förslag}
\begin{attsatser}
    \item Josefin Kokkinakis väljs till protokolljusterare
\end{attsatser}
\subsubsection*{Beslut}
\begin{attsatser}
    \item förslaget till beslut bifalles
\end{attsatser}

\section{Rapport}
\subsection{Ordförande}
Har kommit i mer kontakt med tryckeriet, väntar på en korrektur. Inga kommande möten på ett tag så inte mycket att rapportera.

\subsection{Kassör}
Allting går väldigt smidigt detta året med mottagning. Det går väldigt bra och Hannes sköter sig exemplariskt.

\subsection{Vice-ordförande}
Ska försöka få ordning på märket på backen, annars inte mycket att rapportera.

\subsection{SAMO}
Tänkte skicka ut inlägg om pedagogiskt stöd då många kanske inte vet att det finns.

\subsection{Mötessekreterare}
Har skrivit ett litet latex-relaterat script för att skapa nya protokoll.
Underlättar förhoppningsvis för framtida sekreterare.

\newpage

\section{Beslutspunkter}

%----------------------------------------------------------------------------------------------------------------------------------------------
\subsubsection*{Incidentshanteringsdokumentet ska bli del av styrdokumenten}

\subsubsection*{Förslag till beslut:}
\begin{attsatser}
    \item Att incidentshanteringsdokumentet blir en del av styrdokumenten
\end{attsatser}

\subsubsection*{Beslut}
\begin{attsatser}
    \item Attsatserna bifalles, läggs in på wikin.
\end{attsatser}
%----------------------------------------------------------------------------------------------------------------------------------------------
\subsubsection*{Hantering vid uthyrning av material till andra föreningar}

\subsubsection*{Förslag till beslut:}
\begin{attsatser}
    \item Uthyrning av material till andra föreningar ska gå genom styrelsen och via uthyrningskontrakt som ska färdigställas.
\end{attsatser}

\subsubsection*{Beslut}
\begin{attsatser}
    \item Attsatserna bifalles
\end{attsatser}

%----------------------------------------------------------------------------------------------------------------------------------------------
\subsubsection*{Ska vår inventeringslista delas med andra IT-program?}

\subsubsection*{Förslag till beslut:}
\begin{attsatser}
    \item Nej, inventeringslistan ska inte delas med andra IT-program.
\end{attsatser}

\subsubsection*{Beslut}
\begin{attsatser}
    \item Attsatserna bifalles
\end{attsatser}
%----------------------------------------------------------------------------------------------------------------------------------------------

\section{Diskussionspunkter}

%----------------------------------------------------------------------------------------------------------------------------------------------
\subsection*{Påminna om phadderkontrakt}
Styrelsen vill påminna om att phadderkontrakt ska följas samt påminna om alkoholpolicy.

%----------------------------------------------------------------------------------------------------------------------------------------------
\subsection*{Nya styrelsemedlemmar}
Vi vill börja leta efter nya styrelsemedlemmar då det börjar bli dags. Vi vill göra det tydligt för kommande medlemmar om vad rollerna innebär och ordna en arbetsbeskrivning.
Kul med rekryteringsplancher att sätta upp i Monaden.

%----------------------------------------------------------------------------------------------------------------------------------------------
\subsection*{Inkludera komitteordföranden på styrelsemötena?}
Vi diskuterar om vi ska inkludera kommittéordföranden på styrelsemötena för att få en bättre överblick över vad som händer i föreningen. Tanken är att de inte ska vara ledamöter, utan endast medverkande under diskussion både på besluts- och diskussionspunkter.
Vi ligger upp mötena i eventkalendern.

%----------------------------------------------------------------------------------------------------------------------------------------------
\subsection*{Datum för nästa stämma}
Preliminärt så har vi det ungefär en vecka efter finalsittningen. Diskuteras mer på nästa möte.

%----------------------------------------------------------------------------------------------------------------------------------------------

\newpage
\section{Avslutande av möte}

\subsection{Nästa möte} Preliminärt onsdag 4 september kl 17:00


\subsection{Mötets avslutande}
Mötet avslutades kl. 16:19

\styrelsesignaturer

\end{document}
