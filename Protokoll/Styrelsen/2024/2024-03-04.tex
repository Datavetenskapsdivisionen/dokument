% !TeX program = lualatex
\documentclass[protokoll]{dvd}

\KOMAoptions{
    headwidth = 18cm,
    footwidth = 18cm,
}
\usepackage{graphicx}
\begin{document}

\title{Styrelsemöte 4}
\subtitle{2024}
\author{Styrelsen}
\date{2024-03-04}


\textbf{Datum:} \csname @date\endcsname\\
\textbf{Tid:} 12:00\\
\textbf{Plats:} Styrelserummet\\
\textbf{Styrelsemedlemmar:}
\begin{närvarande_förtroendevalda}
    \förtroendevald{Ordförande}{Samuel Hammersberg}{Ja}
    \förtroendevald{Vice ordförande}{Tim Persson}{Ja}
    \förtroendevald{Kassör}{Lukas Gartman}{Ja på länk}
    \förtroendevald{SAMO}{Josefin Kokkinakis}{Ja}
    \förtroendevald{Sekreterare}{Gustav Dalemo}{Ja}
\end{närvarande_förtroendevalda}

%\textbf{Övriga medlemmar:}
%\medlem{Ingen}
\section{Öppnande av möte}

Mötet öppnades av Samuel Hammersberg kl 12:00

\section{Runda bordet}

Runda bordet innebär att varje person berättar hur de känner sig.
Man kan till exempel berätta att man är stressad på grund av en inlämning, irriterad på sin granne, eller bara väldigt glad därför att man ligger i fas med plugget.

\section{Formalia}

\subsection{Styrelsens beslutbarhet}

\blockquote[7 kap. 5 \S~första stycket i stadgan][]{%
    Styrelsen är endast beslutsmässig då samtliga styrelsemedlemmar har fått kallelsen till styrelsemötet och minst hälften av styrelsemedlemmarna är närvarande.
    Ordförande eller vice ordförande måste vara närvarande när beslut tas.
}

\subsubsection*{Förslag}

\begin{attsatser}
    \item Styrelsen har uppnått kraven i 7 kap. 5 § första stycket i stadgan och är därmed beslutbar.
\end{attsatser}
\subsubsection*{Beslut}
\begin{attsatser}
    \item förslaget till beslut bifalles
\end{attsatser}


\subsection{Fastställande av mötesschema}

För att styrelsen ska kunna fatta ett styrelsebeslut eller protokollföra en diskussion behöver punkten i mötesschemat där styrelsen ska fatta beslut vara inlagd eller föras in i mötesschemat senast vid den här punkten.

\subsubsection*{Förslag}

\begin{attsatser}
    \item mötesschemat fastställs utan några förändringar.
\end{attsatser}
\subsubsection*{Beslut}
\begin{attsatser}
    \item förslaget till beslut bifalles
\end{attsatser}

\subsection{Val av mötessekreterare}
Gustav Dalemo tar sig an uppdraget
\subsubsection*{Förslag}
\begin{attsatser}
    \item Gustav Dalemo väljs till mötessekreterare
\end{attsatser}
\subsubsection*{Beslut}
\begin{attsatser}
    \item förslaget till beslut bifalles
\end{attsatser}

\subsection{Val av protokolljusterare}

Protokolljusterare har till uppgift att kontrollera att protokollet i slutändan reflekterar de faktiska besluten och diskussionerna som fördes under mötet.
Utöver protokolljusteraren så ska mötesordförande och mötessekreteraren signera protokollet.
Vid styrelsemöten ska det endast vara en justerare.
Mötesordförande och mötessekreteraren kan inte vara justerare.

\subsubsection*{Förslag}
\begin{attsatser}
    \item Josefin Kokkinakis väljs till protokolljusterare
\end{attsatser}
\subsubsection*{Beslut}
\begin{attsatser}
    \item förslaget till beslut bifalles
\end{attsatser}

\section{Rapport}
\subsection{Ordförande}
Kontaktade PAn från SEM om hur nöjda deras studenterna var för deras utbildning. Inget svar än.
Kontaktade också angående festanmälan. Inget svar än.

Var på programråd. Det gick ok. De nämnde att man kanske borde ändra kurser relaterat till computer science mastern. Algoritmer har ovanligt mycket antagningskrav. Det är ett problem att studenter inte kan läsa den kursen.

Våra matematikkurser togs upp. Bland annat att Koen ska hålla i analyskursen, det var ett förslag. Mer realistiskt är att köpa kurserna från en annan instutition.

Det diskuterades också om hur man får in fler "rätt typ" av sökande till programmet. Det är bra att göra reklam för programmet, på t.ex. gymnasieskolor.


\subsection{Kassör}
Inget att rapportera.

\subsection{Vice-ordförande}
Har mailat angående orbi-anslutning. Har haft ett krismöte med DVRK. Edward ska avsättas som ordförande och Ida ska tillsättas.
Har skickat ut frågan om det finns intresserade för en arbetsgrupp om att måla loggan på marken.

\subsection{SAMO}
Det var data-tjej i fredags. Det var lite föreläsare och fika. Ingen kvalite på något av dem tyvärr. Det var "helt sådär ändå".
Första gången var det jättebra, men efter det har det inte varit lika bra. Hoppas att ettorna tyckte annorlunda. Trevligt att mingla.

Det var också öppet hus i torsdags på humanisten. Det gick bra! 20 pers kom dit ungefär.

\subsection{Mötessekreterare}
Inget att rapportera.

\newpage


\section{Diskussionspunkter}

\subsection*{Plan för upprustning av Monaden}
Kan vara bra att göra en enkät om detta. Tim för anteckningar om punkter som kan förbättras.

\subsection*{Orbi}
Det diskuterades hurvida ett separat konto för Orbi-betalningar är bra eller inte med hänsyn till bokföring.
Kassör anser det vara enklare med samma konto för att undvika komplexitet.

\subsection*{Städ}
Vi ska ta reda på vilket datum Concats har bokat för vårstäd och vi ska lägga till styrelserummet.

\subsection*{Incidenthantering}
Samuel har läst och den verkar toppen. Redo att skickas vidare.

\newpage
\section{Avslutande av möte}

\subsection{Nästa möte}
2024-03-19

\subsection{Mötets avslutande}
Mötet avslutades kl. 13:00

\styrelsesignaturer

\end{document}
