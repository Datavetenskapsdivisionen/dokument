% !TeX program = lualatex
\documentclass[protokoll]{dvd}

\KOMAoptions{
    headwidth = 18cm,
    footwidth = 18cm,
}
\usepackage{graphicx}
\begin{document}

\title{Styrelsemöte 1}
\subtitle{2024}
\author{Styrelsen}
\date{2024-05-07}


\textbf{Datum:} \csname @date\endcsname\\
\textbf{Tid:} 12:10\\
\textbf{Plats:} Styrelserummet\\
\textbf{Styrelsemedlemmar:}
\begin{närvarande_förtroendevalda}
    \förtroendevald{Ordförande}{Samuel Hammersberg}{Ja}
    \förtroendevald{Vice ordförande}{Tim Persson}{Ja}
    \förtroendevald{Kassör}{Lukas Gartman}{Ja på länk}
    \förtroendevald{SAMO}{Josefin Kokkinakis}{Ja}
    \förtroendevald{Sekreterare}{Gustav Dalemo}{Ja}
\end{närvarande_förtroendevalda}

%\textbf{Övriga medlemmar:}
%\medlem{Ingen}
\section{Öppnande av möte}

Mötet öppnades av Samuel Hammersberg kl 12:01

\section{Runda bordet}

Runda bordet innebär att varje person berättar hur de känner sig.
Man kan till exempel berätta att man är stressad på grund av en inlämning, irriterad på sin granne, eller bara väldigt glad därför att man ligger i fas med plugget.

\section{Formalia}

\subsection{Styrelsens beslutbarhet}

\blockquote[7 kap. 5 \S~första stycket i stadgan][]{%
    Styrelsen är endast beslutsmässig då samtliga styrelsemedlemmar har fått kallelsen till styrelsemötet och minst hälften av styrelsemedlemmarna är närvarande.
    Ordförande eller vice ordförande måste vara närvarande när beslut tas.
}

\subsubsection*{Förslag}

\begin{attsatser}
    \item Styrelsen har uppnått kraven i 7 kap. 5 § första stycket i stadgan och är därmed beslutbar.
\end{attsatser}
\subsubsection*{Beslut}
\begin{attsatser}
    \item förslaget till beslut bifalles
\end{attsatser}


\subsection{Fastställande av mötesschema}

För att styrelsen ska kunna fatta ett styrelsebeslut eller protokollföra en diskussion behöver punkten i mötesschemat där styrelsen ska fatta beslut vara inlagd eller föras in i mötesschemat senast vid den här punkten.

\subsubsection*{Förslag}

\begin{attsatser}
    \item mötesschemat fastställs utan några förändringar.
\end{attsatser}
\subsubsection*{Beslut}
\begin{attsatser}
    \item förslaget till beslut bifalles
\end{attsatser}

\subsection{Val av mötessekreterare}
Gustav Dalemo tar sig an uppdraget.
\subsubsection*{Förslag}
\begin{attsatser}
    \item Gustav Dalemo väljs till mötessekreterare
\end{attsatser}
\subsubsection*{Beslut}
\begin{attsatser}
    \item förslaget till beslut bifalles
\end{attsatser}

\subsection{Val av protokolljusterare}

Protokolljusterare har till uppgift att kontrollera att protokollet i slutändan reflekterar de faktiska besluten och diskussionerna som fördes under mötet.
Utöver protokolljusteraren så ska mötesordförande och mötessekreteraren signera protokollet.
Vid styrelsemöten ska det endast vara en justerare.
Mötesordförande och mötessekreteraren kan inte vara justerare.

\subsubsection*{Förslag}
\begin{attsatser}
    \item Josefin Kokkinakis väljs till protokolljusterare 
\end{attsatser}
\subsubsection*{Beslut}
\begin{attsatser}
    \item förslaget till beslut bifalles
\end{attsatser}

\section{Rapport}
\subsection{Ordförande}
Jag höll ett möte med Mikaela och Petrus, den nya ordföranden för IT-sektionen. 
Vi diskuterade behovet av att etablera god kommunikation med våra studentrepresentanter. 
Det finns intresse för gemensamma evenemang både internt och med IT-sektionen. 
Vi förväntas hamna under NatSek (Naturvetarsektionen), som har en starkt centraliserad mottagning. 
Vi måste framföra vårt önskemål om att själva få planera vår mottagning för att bevara vår autonomi. 
Vi bör vara förberedda på eventuella budgetförändringar i samband med denna övergång. 
Det är också känt att IT-fakulteten förväntas avvecklas under året, vilket kan påverka vår budget framöver. 
Fortsättning följer...

\subsection{Kassör}
Jag har skickat in tre äskningar till institutionen och väntar på svar. Äskningarna avser stöd för ConCats pluggkväll för analys, för inköp till grillkvällar samt för två "festivalbord". Motiveringen är att vi har brist på utomhusmöbler och "festivalbord" skulle möjliggöra sittplatser och dukning vid evenemang som grillkvällar och liknande utomhusevent.

\subsection{Vice-ordförande}
Förnyat föreningsstatus. Försökte dra folk på valborg så de skulle veta var de skulle vara.

\subsection{SAMO}
Har medverkat på UGAIT-möten, annars inget att rapportera.

\subsection{Mötessekreterare}
Inget att rapportera.


\newpage

\section{Beslutspunkter}

\subsubsection*{Recceguiden, när vem skriver?}

\subsubsection*{Förslag till beslut:}
\begin{attsatser}
    \item Samuel Hammersberg tar på sig att skriva den, deadline 17/5-24.
\end{attsatser}

\subsubsection*{Beslut}
\begin{attsatser}
    \item Attsatserna bifalles
\end{attsatser}


\section{Diskussionspunkter} 

\subsection*{Backen}
Vi har ett starkt team bestående av 4-5 personer som är intresserade av att delta i målningsprojektet, men ingen vill ta på sig ansvaret. Vi diskuterade möjligheten att uppmuntra deltagande genom att mer aktivt engagera personer i projektet. Projektet innebär att måla en logga på marken och eventuellt skaffa den nödvändiga färgen.

\subsection*{Ryggtryck}
Vi har trycket, behöver bara koordinera ihop alla. Deadline bör vara innan insparken.

\subsection*{Avslutningscermoni}
Det kommer att arrangeras en avslutningsfest för examenstagare av DV. Vi behöver planera detta noggrant och bör ha ett dedikerat möte för att diskutera arrangemanget. Preliminärt datum för festen är den 5 juni 2024 klockan 17:00 och den riktar sig främst till kandidat- och masterexamenstagare. Alla som tar examen eller är intresserade är välkomna att delta! Vi bokar ett möte för att planera detta den 13 maj 2024 mellan klockan 11:15 och 13:15.

\subsection*{Användning av entrén}
Flytta ovala bordet! Flytta TVn till en bättre plats. Forstätta diskutera avändning av utrymmet. Det finns en del skräp som ska bort.

\subsection*{GCC, vilka kan vara med och vad?}
Den 14/6-24 ska de låna Monaden. Någon gång efter detta tillfälle vill de att vi håller något event efter 15:00-tiden.
Concats är inblandade eftersom det kanske blir brädspelskväll eller liknande.

\subsection*{Mottagningsevent, vad vill vi göra?}
Lunch med styrelsen, korv med bröd eller hamburgare. Vi ska också ha en trivia-night.

\subsection*{Användning av Götapengar}
Ordförande och kassör godkänner att ungefär 1000kr andvänds från Göta-budget för diskmedel, kaffe och kaffefilter.

\newpage
\section{Avslutande av möte}

\subsection{Nästa möte} 
2024-05-13

\subsection{Mötets avslutande}
Mötet avslutades kl. 12:59

\styrelsesignaturer

\end{document}
